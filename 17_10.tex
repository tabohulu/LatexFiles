\documentclass[20 pts]{article}
\usepackage{xeCJK}
\usepackage{amsfonts}
\usepackage{amssymb}
\usepackage{amsmath}
\usepackage{bm}
\setCJKmainfont{SimSun}
\title{Braided Convolutional Codes With Sliding Window Decoding} 
\author{Kwame Ackah Bohulu}
\date{17-10-2017}
\begin{document}
\maketitle

\newpage
 \section{Abstract}
 この論文には、編み畳み込み符号を復号するのに使用する目新しいスライディングウィンドウ復号アルゴリズムを紹介する。このアルゴリズムはBCJR	アルゴリズムに基づいている。復号待ち時間と性能のトレードオフを調査し、復号の複雑さを減らすため、デコーディングウインドウに使用する均一と不均一なメッセージの受けまわしスケジュールと早期停止ルールを提案する。スライディングウィンドウ復号アルゴリズムのパラメータ、ウインドウの大きさとメッセージの受けまわしスケジュールをうまく選べるため、密度展開分析を行う。母符号より高いレートを得るため、定期的なパンクチュアリングを使用する。シミュレーション結果で明らかになたことは、不均一なメッセージの受けまわしスケジュールと定期的なパンクチュアリングを使用すると、幅広いレートでAWGNチャネル容量に近い性能、合理的な復号の複雑さと目が見えないエラーフローが可能です。
 \section{Introduction}
 \paragraph{}
 編みブロック符号(BBC)は二つの要素符号を相互接続し、接続状況は情報シンボルは両方の要素符号器でチェックして、お互いの要素符号のパリティーシンボルはお互いの入力になる。最近、[5]と[6]で紹介されたBCHの要素符号のBBCと階段符号は、高速光通信に対する調査を行われ、繰り返し硬判定復号を使用することで性能が良いということがわかる。BBCに関する編み畳み込み符号(BCC)は[7]で紹介され、BCJRアルゴリズムに基づく繰り返し復号が使用でき、ターボ符号と似たような符号です。しかしながらBCCは低長さ拘束長の畳み込み符号を要素符号とする。BCCの符号化方法それぞれの情報シンボルが二つの要素符号語が守られているような二次元スライドアレイで説明できる。要素符号の関係は、情報シンボルとパリティーシンボルが二次元スライドアレイに保存された位置で定義される。BBCに同質するしっかりと編んだ畳み込み符号(TBC)は情報シンボルとパリティシンボルを保存するのに高密度配列を使用することによって作られている。一方,まばらに編んだ畳み込み符号(SBC)は、低い高密度を持つので繰り返し復号化性能が改善された[7]。SBC符号の最低距離は拘束長が大きくなるほど直線的に大きくなるということは[7]で数値的に示され、SBCは漸近的に良いと言われる。
 \paragraph{}
 この論文には、AWGN通信路の上で送信するSBC符号の復号化は[7]と[8]で使用する復号器の代わりにスライディングウインドウ復号器が提案される。スライディングウインドウ復号化はLDPC畳み込み符号に関する研究がたくさん研究されており、復号化の性能、メモリー要件と待ち時間のトレードオフが簡単にできる。提案されたアルゴリズムと[7]と[8]の違いはBCJRアルゴリズムをしようする。
 \section{Sparsely Braided Convolutional Codes}
 \paragraph{}
 SBC符号は無限の二次元配列をしようして作られ、パリティーフィードバックで接続される二つのRSC符号からなる。このように、情報シンボルとパリティービットシンボルが一緒編んでいる。SBC復号は、bitwiseとblockwise二つのの種類にわかれている。bitwiseは畳み込み符号インタリーバを使用し、送信が連続されている。blockwiseはブロックインタリーバを使用し、有限長ブロックをそうしんする。この論文には、レート、$R=1/3$のblockwiseSBC符号を使用する。システム図は図1に描かれていて、要素符号は、レート$R_{cc}=2/3$のRSC符号からなる。情報系列は、大きさTのブロックごとにわかれていて、$\mathbf{u}=(\mathbf{u}_0,\mathbf{u}_1,...,\mathbf{u}_t,...)$、$\mathbf{u}_t=(u_{t,1},u_{t,2},...,u_{t,T})$。$P^{(0)},P^{(1)},P^{(2)}$は長さTのブロックインタリーバです。
 \paragraph{}
 $t=0$のとき、情報ブロック$\mathbf{u}_0$と$\widetilde{\mathbf{u}}_0=\mathbf{u}_0P^{(0)}$はそれぞれ要素符号1と要素符号2に一ビットずつ入力する。$\widetilde{\mathbf{v}}^{(2)}_{-1}
$と$\widetilde{\mathbf{v}}^{(1)}_{-1}$は大きさTのすべて0の系列で(初期条件)、それぞれ要素符号1と要素符号2に一ビットずつ入力する。それぞれの要素符号は長さTのパリティーブロック
$\hat{\mathbf{v}}^{i}_0=(\hat{v}^{(i)}_{0,1},\hat{v}^{(i)}_{0,2},...,\hat{v}^{(i)}_{0,T}),i=1,2$を出力する。$\hat{\mathbf{v}}^{1}_0,\hat{\mathbf{v}}^{2}_0$と$\mathbf{u}_0$を多重化し、通信路に送信する。

\paragraph{}
一般に、時間tのとき、パリティーブロック$\hat{\mathbf{v}}^{(1)}_t$は$\mathbf{u}_t$と$\widetilde{\mathbf{v}}^{(2)}_t=\mathbf{v}^{(2)}_{t-1}P^{(2)}$を使用して、要素符号1で計算する。
パリティーブロック$\hat{\mathbf{v}}^{(2)}_t$は$\widetilde{\mathbf{u}}_t=\mathbf{u}_tP^{(0)}$と$\widetilde{\mathbf{v}}^{(1)}_t=\mathbf{v}^{(1)}_{t-1}P^{(1)}$を使用して、要素符号2で計算する。$\hat{\mathbf{v}}^{(1)}_t,\hat{\mathbf{v}}^{2}_t$と$\mathbf{u}_t$符号系列$\mathbf{v}=(\mathbf{v}_0,\mathbf{v}_1,...,\mathbf{v}_t...) $多重化します。
\begin{equation}
\mathbf{v}_t=(v^{(0)}_{t,1},v^{(1)}_{t,1},v^{(2)}_{t,1},v^{(1)}_{t,2},v^{(0)}_{t,2},v^{(1)}_{t,2},v^{(2)}_{t,2},...,v^{(0)}_{t,T},v^{(1)}_{t,T},v^{(2)}_{t,T})
\end{equation}
この論文には、$R_{cc}=2/3$のRSC符号では末尾ビットを使わないので最初と最後の状況は同じである。
 \section{Sliding Window Decoding}
 [7]には、平行パイプライン符号化が使用された。残念ながら、必要な復号化待ち時間が大きいである。[13]-[15]で紹介されたスライディングウインドウ復号化を使用することで、より小さい符号化待ち時間と大体同じ性能が可能です。ここで、blockwiseSBC符号に関する目新しいの小さい待ち時間のスライディングウインドウ復号化方法を紹介する。
 \subsection{Window Decoding}
 AWGN通信路で得た系列は$\mathbf{r}=(\mathbf{r}_0,\mathbf{r}_1,...,\mathbf{r}_t...) $。
\begin{equation*}
\mathbf{r}_t=(r^{(0)}_{t,1},r^{(1)}_{t,1},r^{(2)}_{t,1},r^{(1)}_{t,2},r^{(0)}_{t,2},r^{(1)}_{t,2},r^{(2)}_{t,2},...,r^{(0)}_{t,T},r^{(1)}_{t,T},r^{(2)}_{t,T})
\end{equation*}
$\mathbf{l}^c=(\mathbf{l}^c_0,\mathbf{l}^c_1,...,\mathbf{l}^c_t)$は得られた通信路のLLRとする。
\begin{equation*}
\mathbf{l}^c_t=(l^{c,(0)}_{t,1},l^{c,(1)}_{t,1},l^{c,(2)}_{t,1},l^{c,(1)}_{t,2},l^{,(0)}_{t,2},l^{c,(1)}_{t,2},l^{c,(2)}_{t,2},...,l^{c,(0)}_{t,T},l^{c,(1)}_{t,T},l^{c,(2)}_{t,T})
\end{equation*}
$\mathbf{l}^c_t=(\mathbf{l}^{c,(0)}_{t},\mathbf{l}^{c,(1)}_{t},\mathbf{l}^{c,(2)}_{t})$に分解することができる。
$l^{c,j)}_{t},j = 0,1,2$
 \subsection{Window Decoding Schedules}
 
 \section{Density Evolution Analysis}
 
 
 

\end{document}