\documentclass[20 pts]{article}
\usepackage{xeCJK}
\usepackage{amsfonts}
\usepackage{amssymb}
\usepackage{amsmath}
\usepackage{bm}
\setCJKmainfont{SimSun}
\title{スライディングウインドウ復号化を使用するBraided Convolutional Codes } 
\author{Kwame Ackah Bohulu}
\date{2017/11/7}
\begin{document}
\maketitle

\newpage
\section{Density Evolution Analysis 密度進化分析(みつどうしんかぶんせき)}
Density Evolution Analysisを使用して、(バイナリ消去(しょうきょ)チャネル)
BECの上で送信するウインドウ復号化を使用する
ブロックSBC符号の漸近的な性能(zenkintekina seinou)を分析する。\\
それで、分析の結果は、
大きいブロックサイズの場合、復号化ウインドウサイズと復号化ウインドウスケジュール
の選びのに使用する。
\subsection{Erasure Probability of Component Decoders}
ウインドウ復号化を使用するブロックSBC符号の解析式を導出する(かいせきしきをどうしゅつする)
ために、それぞれのBCJR復号器の
外部出力消去確率(gaibu shutsuryoku shoukyo kakuritsu)を計算する必要ある。\\

伝達関数(dentatsu kansuu)

\subsection{Density Evolution for SBC Codes with Window Decoding}
復号化ウインドウが、時間tから 始まって、t+w+1で終わりを想定(そうてい)する。\\


垂直(すいちょく)復号化のとき、i回の垂直(すいちょく)繰り返しを行った後、情報シンボル、
入力パリティーシンボルと出力パリティーシンボルの
外部出力消去確率(gaibu shutsuryoku shoukyo kakuritsu)は式8で計算できる。\\

水平復号化のとき、左側から右側に移動するとき、D1(D2)が出力パリティーシンボルの
外部出力消去確率(gaibu shutsuryoku shoukyo kakuritsu)をa-priori確率としてD2(D1)
に入力する。\\

最後に、j回の水平繰り返しを行った後、復号化しようとするブロックの復号化消去確率は
式10で計算できる。


\subsection{Results and Discussion}
先行ったDensity Evolution Analysisで、ウインドウサイズと復号性能の関係を確認する。
そして、複雑さに対する最適なウインドウ復号スケジュールも評価する。\\

ウインドウサイズ、wに対するBECの閾値は表1にかいてあって、3より大きい場合性能は
良くならない。\\
大きいブロックサイズの場合、w=3にすれば性能はよくなるはず。\\

ウインドウ復号スケジュールに関しては、通信路の消去(しょうきょ) 確率が与えたら、
少ない繰り返し回数で
期待しているerasure確率が得る方が最適である。\\
通信路のerasure 確率、期待しているerasure確率とウインドウサイズそれぞれが0.65, 
$10^{-9}$、
 3の場合、必要な繰り返し回数が表2で書かれている。\\
 
 必要な垂直(すいちょく)繰り返し回数を1にしてもいい性能が得られるはずなので1にする。また、
 大きいブロックサイズの場合、性能たい複雑さのトレードオフのため1にする方がよい。\\
 
 表2であきらかになったのはMUとLU期待しているerasure確率をえるための繰り返し回数が一番
 少ない。また、繰り返し回数が同じである。\\
 
 なので、大きいブロックサイズの場合、MUかLUが一番良い。\\

\section{Performance of Blockwise SBC Codes with Sliding Window Decoding}
図5ではSUスライディングウインドウ復号化を使用するブロックSBC符号のビット誤り率性能が
書かれている。ブロックの大きさと性能の関係を示す。\\

あきらかになったのはブロックサイズが大きくなるほど復号化性能がよくなる。\\

BER=$10^{-5}$を得るために必要なSNR対復号化待ち時間が図6に描かれている。\\
あきらかになったのはブロックサイズが固定された場合、ウインドウサイズを大きくすると性能もよくなる
けどう、あるウインドウサイズまでよくなる。また、ある復号化待ち時間を超えたら、
大きいブロックサイズの場合、小さいウインドウサイズを
使用すると性能がよくなる。\\

繰り返し回数が固定された場合、スライディングウインドウ復号化と復号化スケジュールの
性能を図7で比べられる。\\


最後に、[7]で使用する復号化方法よりいい性能が得られたことが明らかになったが、
待ち時間が[7]
より大きいです。\\
\section{Computational Complexity}
スライディングウインドウ復号化を使用するSBC符号の計算量を調査し、計算量を減らすため
停止(ていし)ルールを提案する。\\
\subsection{Computational Complexity Analysis}
スライディングウインドウ復号器の計算は、大きさTのブロックを復号化するのに必要な操作数
で分析することができる。\\

1つのトレリス、1つの垂直(すいちょく)そして、1ビット当たり、log-MAP BCJRアルゴリズムを
使用する。\\
復号器に必要な計算は表3に書かれている。

\subsection{Stopping Rules}

\paragraph{[1]}クロスエントロピーに基づく停止ルール\\
連続の水平繰り返しをおこなった後、復号しようとする情報ブロックのAPP値の違いに基づくルール
です。\\


経験で明らかになったことは、収束(しゅうそく)したら元のあたいより$10^{-3}$下がる。\\

なので、i個の水平後にT(i)が$\eta T(0)$より小さかったら、復号化かを停止して、
ウインドウをシフトする。\\

\paragraph{[2]}LLR値の大きさに基づく停止ルール\\
復号しようとする情報ブロックの累積的(るいせきてき)LLR値の収束
(しゅうそく)に基づいている。\\


このルールはLLRの値の閾値(いきち),$\theta$とdepthfactor。MはLLR値が本当に
収束(しゅうそく)のかを確認する。\\

,$\theta$は連続のLLR値のマグニチュードの違いが十部小さいかどうかを確認する。\\

なので、M回の水平繰り返しをおこなった後値の違いが$\theta$より小さかったら、
復号化かを停止して、ウインドウをシフトする。\\

\end{document}