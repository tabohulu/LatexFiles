\documentclass[24 pts]{article}
\usepackage{xeCJK}
\usepackage{amssymb}
\usepackage{amsmath}
\usepackage{amsthm}
\usepackage{graphicx}
\graphicspath{ {images/} }
\usepackage{relsize}

\newcommand{\me}{\mathrm{e}}
\setCJKmainfont[BoldFont= Yu Mincho Demibold]{MS Mincho}
\title{コンピュータネットワーク特論レポート2	 }
\date{07-12-2016}
\author{Kwame Ackah Bohulu 1631133}
\begin{document}
\maketitle
\newpage
\paragraph{1}
Since the probability of a frame beign received undamaged is 0.8, after 10 transmissions, 2 out of the 10 frames will be in error, and those two frames will require re-transmission. After those 2 frames are re-transmitted, at least 1 out of the two frames will be in error and will therefor require another transmission. So to get the entire timing through an average of 13 times is required.
\newpage
\paragraph{2}
propagation time = $20\ ms$,
bit rate = $2\ Kbps$, No. of transmittable bits(r)
\begin{equation*}
\begin{split}
 r=20\times 10^-3 \times 2 \times10^3\\
=40 bits\\
=5 \ bytes
\end{split}
\end{equation*}
To ensure an efficiency of at least 50\% for the stop-and-wait protocol, the frame size should range from $2.5$ bytes to $5$ bytes  

\paragraph{3}
throughput(TP)=receive window/ round trip time\\
data rate =$128\times10^3$bps
frame size=$512$ bytes =$4096$ bits\\
receive window =frame size(in bits) $\times$ n, where n= window size\\
round trip time=$2\times 270$ ms + transmission time\\
transmission time =$\frac{frame \\\ size(in \\\ bits)\times n}{data \ rate}$\\
ack frame size assumed to be 0
\paragraph{}
\textbf{case n=1}\\
transmission time =$\frac{4096\times 1}{128\times 10^3}= 0.032$s\\
TP=$\frac{4096)\times 1}{0.54+0.032}=7.161$Kbps\\

\paragraph{}
\textbf{case n=7}\\
transmission time =$\frac{4096\times 7}{128\times 10^3}= 0.224$s\\
TP=$\frac{4096)\times 7}{0.54+0.224}=37.529$Kbps\\

\paragraph{}
\textbf{case n=15}\\
transmission time =$\frac{4096\times 15}{128\times 10^3}= 0.48$s\\
TP=$\frac{4096)\times 15}{0.54+0.48}=60.235$Kbps\\

\paragraph{}
\textbf{case n=127}\\
transmission time =$\frac{4096\times 127}{128\times 10^3}= 4.064$s\\
TP=$\frac{4096)\times 127}{0.54+4.064}=112.987$Kbps\\




\newpage
\paragraph{4}
propagation speed =$\frac{2}{3}\times 3\times 10^8 = 2 \times 10^8$\\
T1 data rate =1.544 Mpbs
total time required for propagation over 10 Km cable(t)=$\frac{10\times10^3}{2 \times 10^8} = 5\times 10^-5$ s\\
Amount of data that can fit into cable =$1.544\times 10^6 \times 5\times 10^-5 = 77.2$ bits




\end{document}