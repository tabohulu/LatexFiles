\documentclass[fontsize=12pt]{article}
%\usepackage{xeCJK}
\usepackage{titling}
\usepackage{amsfonts}
\usepackage{amssymb}
\usepackage{amsmath}
\usepackage{bm}
\usepackage[dvipdfmx]{graphicx}
%\setCJKmainfont{SimSun}
\title{Research Paper Summary IEEE TIT \\2013/01 - 2013/09 } 
\author{Kwame Ackah Bohulu}
\date{\today}
\begin{document}
\maketitle

\newpage
\title{Zigzag Codes: MDS Array Codes
With Optimal Rebuilding}
\author{Itzhak Tamo, Zhiying Wang, and Jehoshua Bruck, Fellow, IEEE}
\date{2013/03}
\maketitle

Abstract - Maximum distance separable (MDS) array codes are
widely used in storage systems to protect data against erasures. We
address the rebuilding ratio problem, namely, in the case of erasures,
what is the fraction of the remaining information that needs
to be accessed in order to rebuild exactly the lost information? It
is clear that when the number of erasures equals the maximum
number of erasures that an MDS code can correct, then the rebuilding
ratio is 1 (access all the remaining information). However,
the interesting and more practical case is when the number of erasures
is smaller than the erasure correcting capability of the code.
For example, consider an MDS code that can correct two erasures:
What is the smallest amount of information that one needs to access
in order to correct a single erasure? Previous work showed
that the rebuilding ratio is bounded between$\frac{1}{2}$ and$\frac{3}{4}$ 
; however,
the exact value was left as an open problem. In this paper, we solve
this open problem and prove that for the case of a single erasure
with a two-erasure correcting code, the rebuilding ratio is $\frac{1}{2}$ . In
general, we construct a new family of $r$-erasure correcting MDS
array codes that has optimal rebuilding ratio of $\frac{1}{r}$in the case of
a single erasure. Our array codes have efficient encoding and decoding
algorithms (for the cases $r=2$ and $r=3$ , they use a finite
field of size 3 and 4, respectively) and an optimal update property.
\paragraph{}
Index Terms - Distributed storage, network coding, optimal rebuilding,
RAID

\newpage
\title{On Generalized Reed-Solomon Codes Over
Commutative and Noncommutative Rings}
\author{Guillaume Quintin, Morgan Barbier, and Christophe Chabot}
\date{2013/09}
\maketitle

Abstract -In this paper, we study generalized Reed-Solomon
codes (GRS codes) over commutative and noncommutative
rings, we show that the classical Welch-Berlekamp and Guruswami-Sudan
decoding algorithms still hold in this context,
and we investigate their complexities. Under some hypothesis, the
study of noncommutative GRS codes over finite rings leads to the
fact that GRS codes over commutative rings have better parameters
than their noncommutative counterparts. Also, GRS codes
over finite fields have better parameters than their commutative
rings counterparts. But we also show that given a unique decoding
algorithm for a GRS code over a finite field, there exists a unique
decoding algorithm for a GRS code over a truncated power series
ring with a better asymptotic complexity. Moreover, we generalize
a lifting decoding scheme to obtain new unique and list decoding
algorithms designed to work when the base ring is, for example, a
Galois ring or a truncated power series ring or the ring of square
matrices over the latter ring.
\paragraph{}
Index Terms-Algebra, algorithm design and analysis, decoding,
error correction, Reed-Solomon codes.


 
\end{document}