\documentclass[fontsize=12pt]{article}
%\usepackage{xeCJK}
\usepackage{titling}
\usepackage{amsfonts}
\usepackage{amssymb}
\usepackage{amsmath}
\usepackage{bm}
\usepackage[dvipdfmx]{graphicx}
%\setCJKmainfont{SimSun}
\title{Research Paper Summary IEEE TIT \\2017/01 - 2017/02 } 
\author{Kwame Ackah Bohulu}
\date{\today}
\begin{document}
\maketitle

\newpage
\title{List Decoding of Crisscross Errors}
\author{Antonia Wachter-Zeh, Member, IEEE}
\date{2017/01}
\maketitle

In this paper, list decoding of crisscross errors in arrays over finite fields is considered. 
For this purpose, the so-called cover metric is used, where the cover of a matrix is a
 set of rows and columns which contains all non-zero elements of the matrix. 
 A Johnson-like upper bound on the maximum list size in the cover metric is derived, 
 showing that the list of codewords has polynomial size up to a certain radius. 
 Furthermore, a simple list decoding algorithm for a known optimal code construction 
 is presented, which decodes errors in the cover metric up to our upper bound. 
 These results reveal significant differences between the cover metric and the rank 
 metric and show that the cover metric is more suitable for correcting crisscross 
 errors.
 

\paragraph{}
Index Terms— Cover metric, crisscross errors, Johnson bound,
list decoding.

\newpage
\title{Generalized Integrated Interleaved Codes}
\author{Yingquan Wu, Senior Member, IEEE}
\date{2018/02}
\maketitle

— Generalized integrated interleaved codes refer to
two-level Reed–Solomon codes, such that each code of the nested
layer belongs to different subcode of the first-layer code. In this
paper, we first devise an efficient decoding algorithm by ignoring
first-layer miscorrection and by intelligently reusing preceding
results during each iteration of a decoding attempt. Neglecting
first-layer miscorrection also enables to explicitly and neatly
formulate the decoding failure probability. We next derive an
erasure correcting algorithm for redundant arrays of independent
disks systems. We further construct an algebraic systematic
encoding algorithm, which had been open. Analogously, we
propose a novel generalized integrated interleaving scheme over
binary Bose–Chaudhuri–Hocquenghem codes, reveal a lower
bound on the minimum distance, and derive a similar encoding
and decoding algorithm as those of Reed–Solomon codes.
\paragraph{}
Index Terms-Reed-Solomon codes, Bose-ChaudhuriHocquenghem
(BCH) codes, integrated interleaving,
generalized integrated interleaved codes, linear-feedbackshift-register
(LFSR) encoding, syndrome decoding, erasure
correcting, redundant arrays of independent disks (RAID).


 
\end{document}