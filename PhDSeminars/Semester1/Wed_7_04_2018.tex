\documentclass[fontsize=12pt]{article}
%\usepackage{xeCJK}
\usepackage{amsfonts}
\usepackage[makeroom]{cancel}
\usepackage{amssymb}
\usepackage{amsmath}
\usepackage{bm}
\usepackage[dvipdfmx]{graphicx}
\newcommand{\td}{\cdot\cdot\cdot}
\newcommand{\cd}{\cdot}
%\setCJKmainfont{SimSun}
\title{Blind Reconstruction of Reed-Solomon Encoder
and Interleavers Over Noisy Environment} 
\author{Kwame Ackah Bohulu}
\date{\today}
\begin{document}
\maketitle

\newpage
\paragraph{Abstract -}
Blind estimation of code and interleaver parameters
is useful in smart storage systems and ubiquitous communication
applications such as adaptive modulation and coding, reconfigurable
radio systems, non-cooperative radio systems, etc. In this
paper, we analyze Reed-Solomon (RS) encoded data stream and
propose blind estimation algorithms to identify RS code parameters.
We also provide algorithms to estimate block interleaver
parameters from RS coded and block interleaved data stream.
In addition, synchronization compensation through appropriate
bit/symbol positioning is integrated with the proposed code and
interleaver parameter estimation algorithms. Simulation results
validating the proposed algorithms are given for various test
cases involving both erroneous and non-erroneous scenarios.
Moreover, the accuracy of estimation of RS code and block
interleaver parameters are also given with detailed inferences
for different modulation schemes, codeword length, and code
dimension values. It has been inferred that the accuracy of
parameter estimation improves with decrease in code dimension
and codeword length values of RS codes. Further, the
accuracy of estimation of lower modulation order schemes is
better when compared to higher modulation order schemes as
expected. It has also been noted that the proposed code and
interleaver parameter estimation algorithms for noisy environment
consistently outperform the algorithms proposed in the
prior works.

\section{Introduction}
 
Foward error correcting (FEC) codes and interleavers are useful in both digital storage 
and communication systems for dealing with the random and burst errors respectively.
The blind reconstruction of code and interleaver parameters are important in 
non-cooperative communication systems and have certain advantages when used in 
applications such as adaptive modulation and coding(AMC), data storage systems, etc. 

Because 
the receiver has no information of the code parameters used before transmission it is
necessary for the receiver to estimate the code parameters in a non-cooperative scenario
. In AMC systems, the information about modulation and coding parameters are known 
to the receiver via control channels and the advantage of using blind estimation will be 
 the conservation of channel resources [3]-[5].

 
 \section{Mathematical Model for ARP and CPP Interleavers}
 \subsection{ARP interleavers}
 This interleaver was proposed in [2] by Berrou et al and is based on a regular permutation of period $P$ and a vector of shifts $S$
 
 \begin{equation}
 \Pi_{ARP(x)} =\Big(P.x + S_{(x \mod Q)}\Big)\mod K
 \end{equation}
 
 where $x = 0, \cdot\cdot\cdot , K - 1$ denotes the address of the data
symbol after before interleaving and $ \Pi_{ARP(x)}$ represents its corresponding
address after interleaving. $P$ is a positive integer
relatively prime to the interleaver length $K$. The disorder cycle
or disorder degree in the permutation is denoted by $Q$ and it
corresponds to the number of shifts in $S$. $K$ must be a multiple
of Q.

 \subsection{CPP interleavers}
 PP interleavers are based on permutation polynomials over integer rings $\mathbb{Z_K}$ and were proposed by Sun et al.[3][4]. The interleaver function for a CPP is shown in (\ref{two})
 
  \begin{equation}
 \Pi_{CPP(x)} =\Big(f_1\cdot x + f_2\cdot x^2 +f_3 \cdot x^3\Big) \mod K
 \label{two}
 \end{equation}
 
 where $x = 0, \cdot\cdot\cdot , K - 1$ denotes the address of the data
symbol after before interleaving and $ \Pi_{CPP(x)}$ represents its corresponding
address after interleaving. 

The necessary and sufficient conditions for generating CPP interleavers depends on the prime factorization of $K$, where the prime factorization of $K$ is considered below.

\begin{equation}
 K =2^{a_{K,1}}\cdot 3^{a_{K,2}}\cdot \prod^{w(K)}_{i=3}p_i^{a_{K,i}}
 \label{three}
 \end{equation}
 Where $w(K)$ is a positive integer greater than or equal to $2$. if $w(K)=2,\prod^{w(K)}_{i=3}p_i^{a_{K,i}}=1 $ (by definition)

 
 The conditions on the coefficients are given in Table I.
\paragraph{Note - } these conditions have to be fulfilled only for
the prime factors of K.

\section{Conditions for a CPP Interleaver to be  Expressed as an ARP Interleaver}
Resulting from the definition of the ARP interleaver, a suitable first condition is that the value of $Q$ should be a submultiple of $K$. In this section, an expression for for the value of Q for the equivalent ARP is derived which depends of three(3) variables, which are the value of $K$, the coefficient $f_3$ of the CPP and on a positive integer denoted by $l$.

Using the idea from [5] we see that a sufficient condition for an ARP equivalent form of the CPP interleaver is that

\begin{equation}
(P\cdot x) \mod K =(f_1\cdot x) \mod K, \forall \,x= 0,\cdot\cdot\cdot,K-1
\label{four}
\end{equation}

and 
\begin{equation}
 S_{(x \mod Q)}\mod K = \Big(f_2\cdot x^2 +f_3 \cdot x^3\Big) \mod K,\forall \,x= 0,\cdot\cdot\cdot,K-1
 \label{five}
\end{equation}
(\ref{four}) and (\ref{five}) are satisfied if $P=f_1$ and

\begin{equation}
\begin{split}
&\Big(f_2\cdot x^2 +f_3 \cdot x^3\Big) \mod K = \Big(f_2\cdot (x+Q)^2 +f_3 \cdot (x+Q)^3\Big) \mod K ,\forall \,x= 0,\cdot\cdot\cdot,K-1\\
&\Big(f_2\cdot x^2 +f_3 \cdot x^3\Big) \mod K = \Big(f_2\cdot x^2 +f_3 \cdot x^3 + (f_2\cdot Q^2 +f_3 \cdot Q^3)\\
&+(2\cdot f_2 \cdot Q +3\cdot f_3 \cdot Q^2)\cdot x + (3\cdot f_3 \cdot Q)\cdot x^2 \Big) \mod K ,\forall \,x= 0,\cdot\cdot\cdot,K-1\\
\label{six}
\end{split}
\end{equation}

(\ref{six}) is true if 
\begin{equation}
\begin{split}
  (f_2\cdot Q^2 &+f_3 \cdot Q^3)\\
&+(2\cdot f_2 \cdot Q +3\cdot f_3 \cdot Q^2)\cdot x + (3\cdot f_3 \cdot Q)\cdot x^2 = 0 \mod K ,\forall \,x= 0,\cdot\cdot\cdot,K-1\\
\label{seven}
\end{split}
\end{equation}

(\ref{seven}) is true if the coefficient of the $x$ term is a quadratic or linear null polynomial and from [12] we know that a quadratic null polynomial $\mod K$ only exists when $2\mid K$ and it is
\begin{equation}
Z_{QNP}(x) = (\frac{K}{2}\cdot x \frac{K}{2}\cdot x^2) \mod K
\end{equation}
also from [23] we know that there are no linear null polynomials. From the above equations (\ref{7}) is true if and only if


\begin{equation}
    \begin{cases}
      &(f_2\cdot Q^2 +f_3 \cdot Q^3) = 0 \mod K \\
      &(2\cdot f_2 \cdot Q +3\cdot f_3 \cdot Q^2) = 0 \mod K\\
      &(3\cdot f_3 \cdot Q)= 0 \mod K
    \end{cases}
    \label{nine}
  \end{equation}
  
  or when $2 \mid K$
  \begin{equation}
    \begin{cases}
      &(f_2\cdot Q^2 +f_3 \cdot Q^3) = 0 \mod K \\
      &(2\cdot f_2 \cdot Q +3\cdot f_3 \cdot Q^2) = \frac{K}{2} \mod K\\
      &(3\cdot f_3 \cdot Q)= \frac{K}{2} \mod K
    \end{cases}
    \label{ten}
  \end{equation}
   using the third equation in (\ref{nine}) we see that $$Q = \frac{l\cdot K}{3\cdot f_3}, \, l \in \mathbb{N}^+$$ 
   and we may rewrite (\ref{nine}) as 
   
   \begin{equation}
    \begin{cases}
      &Q = \frac{l\cdot K}{3\cdot f_3}, \, l \in \mathbb{N}^+\\
      &\frac{l^2\cdot K^2\cdot (3\cdot f_2 + l\cd K)}{3^3 \cd f_3^2} \in \mathbb{N}^+\\
      &\frac{l\cd K \cd (2 \cd f_2 + l \cd K)}{3\cd f_3} \in \mathbb{N}^+
    \end{cases}
    \label{eleven}
  \end{equation}
  
  also from the third equation in (\ref{ten}) we get
  $$ Q = \frac{l_o \cd K}{2 \cd 3 \cd f_3} \in \mathbb{N}_o $$
  and we may rewrite (\ref{ten}) as 
  
   \begin{equation}
    \begin{cases}
      &Q = \frac{l_o \cd K}{2 \cd 3 \cd f_3} \in \mathbb{N}_o\\
      &\frac{l_o^2\cdot K^2\cdot (2 \cd 3\cdot f_2 + l_o\cd K)}{2^3 \cd 3^3 \cd f_3^2} \in \mathbb{N}^+\\
      &\frac{l_o\cd K \cd (2 \cd f_2 + l_o \cd K)}{2^2 \cd 3\cd f_3}-\frac{1}{2} \in \mathbb{N}^+
    \end{cases}
    \label{twelve}
  \end{equation}
  
  \section{CPP Interleavers Seen as Particular Cases
of ARP Interleavers }
In this section, the conditions on powers of prime numbers from the factorization $l$ for the CPP to be espressed as ARP are presented in Theorem 1. It is shown that the powers depend on the powers from the prime factorization of $K$ and the CPP coefficients $f_2$ and $f_3$. 

First the general form of the factorization of $K$ is shown below
\begin{equation}
 K =2^{a_{K,1}}\cdot 3^{a_{K,2}}\cdot \prod^{w(K)}_{i=3}p_i^{a_{K,i}}
 \prod^{w(K)}_{w(K)-n_{4_a}+1}p_i
\end{equation}
where $n_{4_a}$ is the number of prime factors that satisfy the conditions $(p_i-1)$ is not divisible by $3$ when $p_i>3$ and $a_{K,i}=1$. It should be noted that the prime factors that satisfy this condition are written out last in the prime factorization of K. 

\paragraph{Example 1a. : }
For $K=22540$, we have a prime factorization of the form $2^2\cd 3^0\cd 7^2\cd 5^1 \cd 23^1$. We therefore have 
\begin{itemize}\item$w(K)=5, n_{4_a}=2$
\item  $p_3=7, p_4 = 5, p_5=23$
\item $a_{K,1}=2, a_{K,2}=0, a_{K,3}=2,a_{K,4}=1, a_{K,5}=1$
\end{itemize}

The general form of the  factorization of the coefficients $f_j, j=2,3$ is shown below

\begin{equation}
\begin{split}
f_j= &2^{a_{f_j},1}\cd 3^{a_{f_j},2}\cd \prod^{w(K)-n_{4_a}}_{i=3} p_i^{a_{f_j},i}\\
&\cd \prod^{w(K)}_{i=w(K)-n_{4_a}+1} p_{i,f_j}^{a_{f_j,i}} \cd \prod_{w(K)+1}^{w(f_j)} p_{i,f_j}^{a_{f_j,i}}
\end{split}
\label{fourteen}
\end{equation}
where $w(f_j)$ is an integer greater than or equal to $w(K)$

We have to mention that for a true
CPP it is possible to have the coefficient $f_2 = 0$. In this case, the
factorization of $f_2$ as in (18) is not valid and the terms
which contain the variables $f_2$ in systems (\ref{eleven}) and (\ref{twelve}) must
be removed.

\paragraph{Example 1b. : }
Let the coefficients of the CPP be $f_1=11, f_2 = 4186 =
2^1 \cd 3^0 \cd 7^1 \cd 5^0 \cd 23^1 \cd 13^1$ and $ f_3 = 322 = 2^1 \cd 3^0 \cd 7^1 \cd
5^0 \cd 23^1$. According to (\ref{fourteen}),
we have 
\begin{itemize}
\item$w( f2) = 6, w(f3) = w(K) = 5$
\item $ a_{f_2,1} = 1, a_{f_2,2} = 0, a_{f_2,3} = 1, a_{ f_2,4} = 0, a_{f_2,5} = 1, a_{f_2,6} = 1$
\item$a_{f_3,1} = 1, a_{f_3,2} = 0, a_{f_3,3} = 1, a_{f_3,4} = 0 a, a_{f_3,5} = 1$
\end{itemize}

The decomposition of $l$ from (\ref{eleven}) is 
\begin{equation}
 K =2^{a_{l,1}}\cdot 3^{a_{l,2}}\cdot \prod^{w(K)}_{i=3}p_i^{a_{l,i}}
 \prod^{w(f_3)}_{w(K)+1}p_{i,f_3}^{a_{l,i}}
\end{equation}

The decomposition of $l_o$ from (\ref{twelve}) is 
\begin{equation}
 K =3^{a_{l_o,2}}\cdot \prod^{w(K)}_{i=3}p_i^{a_{l_o,i}}
 \prod^{w(f_3)}_{w(K)+1}p_{i,f_3}^{a_{l_{o},i}}
\end{equation}

Rewriting the conditions from systems  (\ref{eleven}) and (\ref{twelve}) and
taking into account (13) - (16), we obtain the conditions for
CPP interleavers to be expressed as ARP interleavers. These
conditions are given in Theorem 1 below. (refer to paper)

\paragraph{Example 2 :} Let $K, f_1, f_2$ and $f_3$ be as in Example 1. Then,
a valid value of $l$ is $l = 42 = 2^1 \cdot 3^1 \cd 7^1 \cd 5^0 \cd 23^0$, for which $Q = 980$ results. Thus, according to (15), we have
$a_{l,1} = 1, a_{l,2} = 1, a_{l,3} = 1, a_{l,4} = 0, and a_{l,5} = 0$


\end{document}