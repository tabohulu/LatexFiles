\documentclass[fontsize=12pt]{article}
%\usepackage{xeCJK}
\usepackage{amsfonts}
\usepackage{amssymb}
\usepackage{amsmath}
\usepackage{bm}
\usepackage[dvipdfmx]{graphicx}
%\setCJKmainfont{SimSun}
\title{Reed-Solomon Codes } 
\author{Kwame Ackah Bohulu}
\date{\today}
\begin{document}
\maketitle

\newpage
\section{Introduction}
Reed-Solomon (RS) codes are t-error correcting codes which excell at correcting burst errors
 and are used in many applications including data storage and data transmission.
 Interleavers are used in conjuction with RS codes in two ways. The first is post encoding to 
 further combat the effects of burst errors or to construct so called interleaved RS codes.
 In both cases, either the block interleaver or the convolutional interleaver is used. However
 in both cases, the error correcting capability isnt significantly improved.  
 
In this research, we begin by studying how these interleavers affect the 
on the error correction capability of the RS codes in both of the above mentioned cases. 
We then deduce the 
 necessary and sufficient conditions to improve the error correcting capability of the RS 
 code for both cases via an interleaver. Based on  these conditions, we propose a new interleaver and 
 compare its error correcting capability to the block interleaver and the convolutional 
 interleaver via simulation
 
 Reed-Solomon (RS)符号はバーストエラー訂正に優れた誤り訂正符号であるため、情報伝送
とデータストレージなどに応用されている。RS符号にインタリーバを利用する場合は符号化後か
 いんたりーぶ
\end{document}