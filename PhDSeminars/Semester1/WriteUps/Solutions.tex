\documentclass[fontsize=12pt]{article}
%\usepackage{xeCJK}
\usepackage{amsfonts}
\usepackage{amssymb}
\usepackage{amsmath}
\usepackage{bm}
\usepackage[dvipdfmx]{graphicx}
%\setCJKmainfont{SimSun}
\title{HTML/CSS/Javascript
プログラミング解答 } 
\author{Kwame Ackah Bohulu}
\date{\today}


\begin{document}
\maketitle

\newpage
\section{課題1}

\begin{figure}[h!]
  \includegraphics[width=0.5\linewidth]{calculator.jpg}
  \caption{HTML,CSS,Javascript の電卓}
  \label{fig1}
\end{figure}
図\ref{fig1} の電卓のコードは以下に書かれています。
\begin{verbatim}
<html>
<head>

<script type = "text/javascript">

function displaynum(n1)
{
document.calculator.screen1.value=calculator.screen1.value+n1;
}

function solve()
{
document.calculator.screen1.value=eval(calculator.screen1.value);
}

function cl()
{
document.calculator.screen1.value="";
}
</script>
<style>
   
   .button{
   width:50;
   height:50;
   font-size:25;
   margin:2;
   background:grey;
   border:none;
   }
   .textview{
   width:225;
   margin:2;
   font-size:25;
   padding:5;
   }
   .main{
   position:absolute;
   top:25%;
   left:9%;
   }
   .bg{
   position:absolute;
   background:linear-gradient(to right,red,blue);
   height:60%;
   width:25%;
   left:5%;
   top:20%;
   }
   </style>
</head>

<body>
<div class="bg"></div>
<div class ="main">
<form name =calculator>
<input type=textview class=textview name = screen1 style = "text-align:right" disabled><br>
<input type=button class=button name = btn9 value =9 onclick ="displaynum(btn9.value)">
<input type=button class=button name = btn8 value =8 onclick="displaynum(btn8.value)">
<input type=button class=button name = btn7 value =7 onclick ="displaynum(btn7.value)">
<input type=button class=button name = btnpls value=+ onclick ="displaynum(btnpls.value)"><br>
<input type=button class=button name = btn6 value =6 onclick ="displaynum(btn6.value)">
<input type=button class=button name = btn5 value =5 onclick ="displaynum(btn5.value)">
<input type=button class=button name = btn4 value =4 onclick ="displaynum(btn4.value)">
<input type=button class=button name = btnmin value = - onclick ="displaynum(btnmin.value)"><br>
<input type=button class=button name = btn3 value =3 onclick ="displaynum(btn3.value)">
<input type=button class=button name = btn2 value =2 onclick ="displaynum(btn2.value)">
<input type=button class=button name = btn1 value =1 onclick ="displaynum(btn1.value)">
<input type=button class=button name = btnmul value =* onclick ="displaynum(btnmul.value)"><br>
<input type=button class=button name = btn0 value =0 onclick ="displaynum(btn0.value)">
<input type=button class=button name = btneq value == onclick ="solve()">
<input type=button class=button name = btndot value =. onclick ="displaynum(btndot.value)">
<input type=button class=button name = btndiv value= / onclick ="displaynum(btndiv.value)"><br>
<input type=button class=button name = btnclr value= C style="width:225" onclick ="cl()">


</form>
</body>
</html>
\end{verbatim}
\section{課題2}
\begin{verbatim}
function rotate(x,y,deg) {
rad = deg * Math.PI / 180;
yp = y*Math.cos(rad) - x*Math.sin(rad);
xp = y*Math.sin(rad) + x*Math.cos(rad);
    return [yp,xp];
}
\end{verbatim}



 
\end{document}