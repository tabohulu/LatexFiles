\documentclass[fontsize=12pt]{article}
%\usepackage{xeCJK}
\usepackage{amsfonts}
\usepackage{amssymb}
\usepackage{amsmath}
\usepackage{bm}
\usepackage[dvipdfmx]{graphicx}
%\setCJKmainfont{SimSun}
\title{On Cyclic Codes of Composite Length
and the Minimum Distance} 
\author{Kwame Ackah Bohulu}
\date{\today}
\begin{document}
\maketitle

\newpage
\section{Introduction}
The theory of error-correction codes is an important
research area, and it has many applications in modern life including the recovery of corrupted information after transmission
over an unreliable channel.
Till date, the construction of  new error-correction codes
with good parameters as well as the problem of finding the minimum distance and designing efficient decoding
and encoding algorithms is still a major challenge in coding theory.
Let $\mathbb{F}_q$ be a finite field of order $q$. Let $n_1, n_2$ be two distinct
odd primes such that $(n_1n_2, q) = 1$ and $q$ is a quadratic residue
for both $n_1$ and $n_2$.
In an interesting paper [2], Ding provided a
general construction of cyclic codes of length $n_1n_2$ and dimension
$(n_1n_2 + 1)/2$ over $F_q$ by using generalised cyclotomies
of order two in $\mathbb{Z}^*_{n_1n_2}$ and this construction is similar to that of
quadratic residue codes of prime length, which can be defined
by using cyclotomy of order two in $\mathbb{Z}^*_n$
when $n$ is a prime number.

For the case when  $n$ is a product of two distinct odd primes, there are three different generalised cyclotomies of
$\mathbb{Z}^*_n$ and these 
correspond to Ding’s first, second and third construction,
each of which yields $8$ cyclic codes of length $n$ and dimension
$(n +1)/2$. The information corresponding to the construction of these cyclic codes is tabulated in Table 1.

In this paper, theory on Ding’s three constructions
that partially explains some of the data in Table 1 is provided
(see Theorems 2, 3 and 4 in Section III): first, we prove
that under permutation equivalence, there are indeed two
codes in each construction; second, we prove an ``almost''
square-root bound  on the minimum distance(i.e. $d_{min} >
\sqrt{n_1} or
\sqrt{n_2}$ for the codes of length $n_1n_2$) which is satisfied
by all these codes; third, for Ding’s second and third
construction, we illustrate why half of the cyclic codes have
relatively small minimum distance.
Previously in [2] only
lower bounds on the minimum odd-like weight of the codes
were obtained but it is well-known that minimum odd-like
weight may be much larger than the minimum distance of
the codes. This is actually the case for the codes from Ding’s
constructions.
\section{Cyclic Codes Of Composite Length}
in this section, standard notation ,definitions, theorems and proofs are introduced here

\subsection{Notation and Definitions}
\begin{itemize}
\item $\mathbb{F}_q$ represents the finite field of order $q$, where $q$ is a prime
power. 
\item A linear $[n, k, d; q]$ code $\mathcal{C}$ is a $k$-dimensional subspace
of $\mathbb{F}_q^n$
with minimum (Hamming) distance $d = d(\mathcal{C})$.
\item A linear
$[n, k]$ code $\mathcal{C}$ over $\mathbb{F}_q$ is called a cyclic code of length $n$ if any
$(c_0, c_1, ... , c_{n-1}) \in \mathcal{C}$ implies $(c_{n-1}, c_0, c_1, . . . , c_{n-2}) \in \mathcal{C}$.
\begin{itemize}
\item By identifying any vector$ (c_0, c_1, . . . , c_{n-1}) \in \mathbb{F}_q^n$
with $$c_0x_0+ c_1x^1+ ... + c_{n-1}x^{n-1}\in \mathbb{R}_n := \mathbb{F}_q[x]/(x^n-1),$$
 $\mathcal{C}$ is a cyclic code of length $n$ over $\mathbb{F}_q$ if and
only if the corresponding subset of $\mathbb{R}_n$, (still written as $\mathcal{C}$) is an
ideal of the ring $\mathbb{R}_n$. 
\item Since every ideal of $\mathbb{R}_n$ is principal,
there is a monic polynomial $g(x) \in \mathbb{F}_q [x]$ of least degree
such that $\mathcal{C} = (g(x)) \subset \mathbb{R}_n$. 
\item $g(x)$ is unique, satisfying
$g(x)|(x^{n}−1)$ and is called the generator polynomial of $C$, and
$h(x) := (x^{n} − 1)/g(x)$ is called the parity-check polynomial
of $C$.
\end{itemize}
\item Two codes $\mathcal{C}_1$ and $\mathcal{C}_2$ are called permutation equivalent,
written as $\mathcal{C}_1 \sim \mathcal{C}_2$, if there is a permutation of coordinates
that sends $\mathcal{C}_1$ to $\mathcal{C}_2$. The permutation of coordinates is called
a permutation equivalence.
\item For any $c(x) \in \mathbb{F}_q [x]$, define Supp$(c(x))$ to be
the set of integers $i$ such that the term $x^i$ appears
in $c(x)$. Define the weight wt$(c(x))$ to be the cardinality
of the set Supp$(c(x))$. 
\end{itemize}

\subsection{Theorems}
The main theorem is the
following
\paragraph{Theorem 1:}
Let $n, r \geq 2$ be positive integers such that
$gcd(nr, q) = gcd(n, r ) = 1$. Assume that $r |(q −1)$. Let $\theta$ be a
primitive $nr$-th root of unity in some extension of $\mathbb{F}_q$ . Define
$\lambda := \theta^n$ and let $\tilde n$ be a positive integer such that $n \tilde n\equiv 1(\mod r )$. For any $0 \leq t\leq r-1$,
 let $\theta_t := \lambda^{\tilde nt}$ . We define the
map
$$ \phi : \frac{\mathbb{F}_q[x]}{(x^{nr}-1)} \rightarrow \Big ( \frac{\mathbb{F}_q[x]}{(x^{n}-1)} \Big )^r$$

by 

$$ \phi : c(x) \mapsto \frac{1}{r} \Big ( \sum_{t=0}^{r-1} c_t(x)\lambda^{-tk}\Big)_{k=0}^{r-1}$$ 
here for any $c(x) \in \mathbb{F}_q[x]/(x^{nr}-1)$ the polynomial $c_t (x) \in
 \mathbb{F}_q[x]/(x^{n}-1)$ is given by
$$c_t (x) \equiv c(x\theta_t) (\mod x^n - 1)\\  \forall 0 \leq t \leq r-1$$

Then:

\paragraph{1)} the map $\phi$ is a permutation equivalence, and $c(x) 
= 0$ if
and only if $(c_t (x))^{r-1}_{t=0}\neq 0$;
\paragraph{2)} if $c(x)\neq 0$, then
$$wt(c(x)) \geq \min \{wt(ct (x)) : c_t (x) 
\neq 0\}$$.
Moreover,
\paragraph{2.1)} if $c_0(x) = c_1(x) = ... = c_{r−1}(x)$, then $wt(c(x)) =
wt(c_0(x))$;
\paragraph{2.2)} if $c_t (x) = 0 $for some $t$, then
$$wt(c(x)) \geq 2 \min \{wt(c_t (x)) : c_t (x) 
\leq 0\}$$.

Let $\mathcal{C} = (g(x)) \subset \mathbb{F}_q[x]/(x^{nr} − 1)$ be a cyclic code with the
generator polynomial $g(x)$. Then
\paragraph{3)} $\mathcal{C}$ is permutation equivalent to $\phi(\mathcal{C})$, which is given by

$$ \phi(\mathcal{C}) = \Bigg \{ \Big ( \sum_{t=0}^{r-1} c_t(x)\lambda^{-tk}\Big)_{k=0}^{r-1} : c_t(x) \in C_t \forall t \Bigg \}, $$
where $\mathcal{C}_t = (gt (x)) \subset \mathbb{F}_q [x]/(x^n - 1)$ is a cyclic code
with the generator polynomial $g_t (x)$ given by
$$g_t (x) = gcd(g(x\theta_t ), x^n -1) \forall 0 \leq t \leq r-1 $$

\paragraph{4)} If $\mathcal{C} \neq 0$, then
$$d(\mathcal{C}) \geq min
\{d(\mathcal{C}_t ) : \mathcal{C}_t \neq 0\}$$.
Moreover,
\paragraph{4.1) }if $\mathcal{C}_0 = \mathcal{C}_1 = ... = \mathcal{C}_{r-1}$, then $d(\mathcal{C}) = d(\mathcal{C}_0)$;
\paragraph{4.2)} if $\mathcal{C}_t = 0$ for some $t$, then
$$d(\mathcal{C}) \geq 2 \min_t
\{d(\mathcal{C}_t ) : \mathcal{C}_t 
\neq 0\}.$$

\subsection{Proofs}

\paragraph{1)}
Writing 

$$  c_t(x) =\sum_{z=0}^{n-1} c_{t,z} x^z  $$
$$  c(x) = \sum_{i=0}^{nr-1}c_ix^i = \sum_{k=0}^{r-1} \sum_{z=0}^{n-1} c_{kn+z} x^{kn+z}  $$

then from $c_t (x) \equiv c(x\theta_t) (\mod x^n - 1)$, we find
$$   c_{t,z} = \sum_{k=0}^{r-1}  c_{kn+z} \theta_t^{kn+z} \forall t,z$$

since $\theta_t = \lambda^{\tilde nt}$ and $\lambda$ is the $r$-th root of unity, we can obtain

$$  c_{kn+z} = \frac{1}{r} \sum_{t=0}^{r-1} c_{t,z} \theta_t^{-kn-z}  \forall t,z $$

Therefore ,
$$  c(x) = \frac{1}{r}  \sum_{t=0}^{r-1}\sum_{z=0}^{n-1} c_{t,z} x^z \sum_{k=0}^{r-1} x^{nk} \lambda^{-t(k+\tilde nz)}  $$

For any given$z$, let $k^{`} \equiv k +\tilde nz (\mod r )$. Noting that $x^{\tilde nk}\equiv 
x^{n(k^{`}-nz)} (\mod x^{nr}-1)$ and as $k^{'}$ runs over a complete residue
system modulo $r$ , so does $k \equiv k^{`} - \tilde nz (\mod r )$., and it is clear
that $\psi : x^{nk} \mapsto x^{nk^{'}}$
induces a permutation of coordinates in
$\mathbb{F}_q [x]/(x^{nr} -1)$, thus $c(x)$ is permutation equivalent to

\begin{equation}
\begin{split}
\psi(c(x)) &
=  \frac{1}{r}  \sum_{t=0}^{r-1}\sum_{z=0}^{n-1} c_{t,z} x^z \sum_{k=0}^{r-1} x^{nk} \lambda^{-tk}\\
& = \frac{1}{r}\sum_{k=0}^{r-1} x^{nk}   \sum_{t=0}^{r-1} c_{t}(x)  \lambda^{-tk}
\end{split}
\end{equation}

Hence $\psi(c(x)) $is permutation equivalent to $\phi(c(x))$, and thus
$\phi$ is a permutation equivalence. Moreover, noting
$$c_t (x) = 0 \iff (x^n − 1)|c_t (x) \iff (x^n − \lambda^t )|c(x),$$
and
$$   x^{nr} -1 =\prod_{t=0}^{r-1}(x^n - \lambda^t ),$$

it is obvious that $c(x) \neq 0$ if and only if $(c_t (x))^{r-1}
_{t=0}\neq 0.$ This
proves 1).

\paragraph{2).} Since $c_t (x) \equiv c(x\theta_t) (\mod {x^n - 1})$ we have 

$$   wt(c(x)) = wt(c(x\theta_t)) \geq wt(c_t (x))\,\,\,\,\,\, \forall 0\leq t \leq r-1.$$

Taking $c_0(x) = ... = c_{r-1}(x)$ in (1), and using
\begin{equation*}
\sum_{t=0}^{r-1}\lambda^-{tk}= 
\begin{cases}
    r:,& \text{if } r|k\\
    0:,              & r\nmid k
\end{cases}
\end{equation*}
we find easily that $\psi(c(x)) = c_0(x)$. This proves 2) and 2.1).

For \paragraph{2.2)}, let

$$ A = \{ 0\leq t \leq r-1 : c_t(x) \neq 0 \}, \,\,\,\, \mathbf{I} = \bigcup_{t \in A} \text{Supp}(c_t(x)) $$
 from (1) we find that 
 
 $$  \text{wt}(c(x)) = \sum_{z\in \mathbf{I}}^{} \sum_{k=0}^{r-1} \text{wt} \Big (\sum_{t\in A}^{} c_{t,z}\lambda^{-tk}  \Big)  $$
 
 for each $k$ and $z$, write
 
 $$  h_{k,z} := \sum_{t \in A} c_{t,z}\lambda^{-tk}  $$
 
 and for each $z$ write 
 $ \underbar{h}_z := (h_{k,z})_{k=0}^{r-1}$ and $\underbar{c}_z:= (c_{t,z})_{t\in A}$
as column vectors. Assume that $A = \{ t_1,t_2,...,t_u \}$ Then we have

\begin{equation}
\begin{bmatrix} 
  1     & 1 &\cdot \cdot \cdot &1 \\ 
  \lambda^{-t_1}     &  \lambda^{-t_2} &\cdot \cdot \cdot & \lambda^{-t_u} \\
   \vdots & \vdots &\vdots &\vdots \\
  \lambda^{-(r-1)t_1}     &  \lambda^{-(r-1)t_2} &\cdot \cdot \cdot & \lambda^{-(r-1)t_u} \\
\end{bmatrix}
\cdot \underbar{c}_z = \underbar{h}_z
\end{equation}

and $$\text{wt}(c(x)) = \sum_{z \in \mathbf{I}} \text{wt}(\underbar{h}_z)$$

Noting that for any $z\in I$, we have $c_z \neq 0$, and the matrix
on the left side of the equation (2) is a Vandermond matrix
of size $r \times u $ where $1 \leq u < r$, we find $\text{wt}(\underbar{h}_z) \geq 2$. Thus

$$ \text{wt}(c(x)) \geq \sum_{z \in \mathbf{I}} 2 \geq 2 \min\{\text{wt}(c(x)) : c_t(x)  \}  $$
This proves 2.2).

\paragraph{3)}. For any $0 \leq t \leq r-1$ define maps, $\phi_t, \,\,\, f_t$

$$  \frac{\mathbb{F}_q[x]}{(x^{nr} - 1)} \rightarrow^{\phi_t } \frac{\mathbb{F}_q[x]}{(x^{n} - \lambda^t)} \rightarrow^{f_t } \frac{\mathbb{F}_q[x]}{(x^{n} - 1)}$$
by 
$$ \phi_t : c(x) \mapsto c(x) \,\,\,(\text{mod} \,x^n-\lambda^t), \,\,\, f_t : x \mapsto x \theta_t$$
Let $\Psi_t : = f_t \circ \phi_t and \Psi = (\Psi)_{t=0}^{r-1}$ . The isomorphism from the Chinese Remainder Theorem

$$ \Psi : \frac{\mathbb{F}_q[x]}{(x^{nr} - 1)} \rightarrow \prod_{t=0}^{r-1}\frac{\mathbb{F}_q[x]}{(x^{n} - 1)}  $$
induces an isomorphism $\Psi(\mathcal{C}) \cong \prod_{t=0}^{r-1} \Psi_t(\mathcal{C})$. Since clearly $\mathcal{C}_t = \Psi_t(\mathcal{C})$, 3) is proved.
\paragraph{4)}
Suppose $c(x) \in \mathcal{C}$ is a codeword with the minimum
distance. The corresponding ($c_t(x))_{t=0}^{r-1} \neq 0$. Since $c_t (x) \in \mathcal{C}_t$
for each $t, \text{if} c_t (x) \neq 0$, then $\text{wt}(c_t (x)) \geq d(\mathcal{C}_t )$. Hence

$$  \text{wt}(c (x)) \geq \min_t \{\text{wt}(c_t (x)) : c_t(x) \neq 0\}  \geq \min\{ d(\mathcal{C}_t) : \mathcal{C}_t \} $$

On the other hand, if $\mathcal{C}_0 = .... = \mathcal{C}_{r-1} \neq 0$ we may
take a codeword $ \bar{c}(x) \in \mathcal{C}_0$ with the minimum distance and
let $c_t (x) = \bar{c}(x) \forall t$. The corresponding codeword
$c(x) \in \mathcal{C}$ satisfies the property that $\text{wt}(c(x)) = \text{wt}(\bar{c}(x))$.
So $d(\mathcal{C}) = d(\mathcal{C}_0)$. This proves 4) and 4.1). The proof of 4.2) is
similar.
\end{document}