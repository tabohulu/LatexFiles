\documentclass[fontsize=12pt]{article}
%\usepackage{xeCJK}
\usepackage{amsfonts}
\usepackage{amssymb}
\usepackage{amsmath}
\usepackage{bm}
\usepackage[dvipdfmx]{graphicx}
%\setCJKmainfont{SimSun}
\title{Performance of Reed–Solomon codes using the
Guruswami–Sudan algorithm with improved
interpolation efficiency} 
\author{Kwame Ackah Bohulu}
\date{\today}
\begin{document}
\maketitle

\newpage
\section{Groups}
 A group is defined as a set $\mathbf{G}$ with an operation $\Delta$ which 
is associative (i.e $(a\Delta b) \Delta c = a\Delta (b \Delta c ))$ has a neutral elements and 
each element has an inverse. Below is a formal definition
 \paragraph{Definition: } A set $\mathbf{G}$ is called a group if it satisfies the ff axioms
 
 \begin{itemize}
 \item operation $\Delta$ is associative  (i.e $(a\Delta b) \Delta c = 
 a\Delta (b \Delta c ))$
 \item There exists an element $e$ in $G$ such that $a \Delta e = e \Delta a =a$ for 
 every $a$ in $\mathbf{G}$
 \item For every element $a$ in $G$, there is an element $a^{-1}$ in $G$ such that 
 $a \Delta a^{-1} = a^{-1} \Delta a  = e$
 

 The symbol used to represent a group is $< \mathbf{G}, \Delta >$ 
 
 Groups are important because they serve as a foundation from which more complex
 algebraic structures can be created. In many scientific applications, finite groups are 
 highly important because in most real world applications a finite number of objects is
 dealt with
 
 If the commutative law ($a \Delta b = b \Delta a$) holds in a group, it is known as 
 an Abelian group.
 
 Examples of groups include the additive group of integers($<\mathbb{Z}, +>$), 
 additive group of rational
 numbers ($<\mathbb{Q}, +>$)and the additive group of real numbers
 ($<\mathbb{R}, +>$)
 \end{itemize}


\section{Elementary properties of Groups}
Below are a few elementary properties of Groups
\begin{itemize}
\item Every group has exactly one identity element and one inverse element
\item For additive groups, the identity element is $0$ and the inverse of $a$ is
$-a$
\item For multiplicative groups, the identity element is $1$ and the inverse of $a$ is 
$a^{-1}$
\end{itemize}
 
 The most basic rule of group-related calculations is the cancellation rule given by
 the theorem below
 
 \paragraph{Theorem 1: } if $\mathbf{G}$ is a group and $a,b,c$ are elements of 
 $\mathbf{G}$,then
 \begin{itemize}
\item $ab=ac$ implies $b=c$ and
\item $ba=ca$ implies $b=c$
\end{itemize}
To prove Theorem 1, multiply the respective equations $a^{-1}$

\paragraph{Theorem 2: }if $\mathbf{G}$ is a group and $a,b$ are elements of 
 $\mathbf{G}$,then
 $$ ab =e \text{ implies } a=b^{-1}$$ and
 $$ ba =e \text{ implies } b=a^{-1}$$
 To proove Theorem 2 use the fact that $aa^{-1}=bb^{-1} = e$
 
 The next theorem gives useful information related to how to calculate inverses
 
 \paragraph{Theorem 3: }if $\mathbf{G}$ is a group and $a,b$ are elements of 
 $\mathbf{G}$,then
 \begin{itemize}
 \item $(ab)^{-1} = b^{-1}a^{-1}$
 \item $(a^{-1})^{-1} =a$
 \end{itemize}
 
 If $\mathbf{G}$ is a finite group, the number of elements in $\mathbf{G}$ is called
 the order of $\mathbf{G}$ and is written as $|G|$
 
 \subsection{Subgroups}
 \paragraph{Definition: }Let $G$ be a group and $S$ a nonempty subset of $G$. 
 If $S$ is closed with respect 
 to opetation $\Delta$ and closed with respect to inverses $S$ is a subgroup of $G$.
 
 An example is the set of all even integers which is a subgroup of the additive group
 of integers $\mathbb{Z}$
 From henceforth, we will use the set symbol to represent the group and assume that
 the operation that is being used in the group is multiplication. 
 It is worth noting that if $S$ is a subgroup of $G$, the operations on $S$ and $G$
 are the same. Additionally, if $G$ is a group and $S$ is its subgroup, $S$ is also a 
 group.
 
 Subgroups are useful given the fact that they can be used to show that certain things
 are groups. The smallest and largest possible subgroups of a group $G$ are the
 one-element subgroup that contains the identity element $e$ and the group $G$
 itself respectively. These groups are known as trivial subgroups, whiles every other 
 subgroup is known as a proper subgroup.
 
 Now, assuming group $G$ has elements $a,b,c$ it is possible to define a subgroup
 $S$ made up of all product combinations of $a,b,c$ and their inverses $a^{-1},
 b^{-1},c^{-1}$. In such a situation, we refer to $S$ as the subgroup of $G$ 
 generated by $a,b,c$. It is also possible to generate a finite subgroup of $n$ 
 elements $S$ from a single element $a$ in $G$ using the previous methodology. A
 subgroup $S$ generated in such a way is known as a \textit{cyclic subgroup}. 
 Cyclic groups are denoted by $<a>$ and $a$ is called the \textit{generator} of $S$.
 It is worth noting that $<a>$ contains all possible products of $a \text{ and } 
 a^{-1}$.
 By extension, a cyclic group G is a generated by a single element $a$ i.e. $G=<a>$
 A set of equations involving only the generators and their inverses is known as the
 defining equation of $G$ if these equations completely determine the multiplication
 table of $G$
 
\end{document}