\documentclass[fontsize=12pt]{article}
%\usepackage{xeCJK}
\usepackage{amsfonts}
\usepackage{amssymb}
\usepackage{amsmath}
\usepackage{amsthm}
\usepackage{mathrsfs}
\usepackage{relsize}
\usepackage{bm}
\usepackage[dvipdfmx]{graphicx}
\theoremstyle{definition}
\newtheorem{definition}{Definition}
\newtheorem{theorem}{Theorem}[section]
%\setCJKmainfont{SimSun}
\title{Formula to Calculate Weight for Low-Weight Weight 3 Inputs and Proof} 
\author{Kwame Ackah Bohulu}
\date{\today}
\begin{document}
\maketitle

\newpage


\section{Equation and Proof}
We begin by defining the following terms.
\begin{equation}
\begin{aligned}
&\alpha_0= (0 1 1),~\alpha_1= (1 0 1 ),~\alpha_2= (1 1 0),\\
&\alpha_0^{'}= (1 1 1),~\alpha_1^{'}= (0 1 1),~\alpha_2^{'}= (0 0 1),\\
& u= c_1 \mod 3,~v= c_2 \mod 3,\\
&d_i =\left \lfloor{\frac{c_i -c_{i-1} -2}{3}}\right \rfloor,~i=\{1,2\}
\end{aligned}
\end{equation}
\begin{theorem}

Given a low-weight weight 3 input $B(x)=x^{c_0}+x^{c_1}+x^{c_2}$, the parity sequence weight for the convolutional codeword can be calculated as 

\begin{equation}
\begin{aligned}
&w_H(\alpha^{'}_0)+w_H(\alpha_0+\alpha^{'}_u)+w_H(\alpha_0+\alpha^{}_u
+\alpha^{'}_v)\\
&+d_1\left(w_H(\alpha^{}_0)\right)+d_2\left(w_H(\alpha^{}_0 + \alpha^{}_u)\right)
\end{aligned}
\end{equation}

\end{theorem}
where $w_H(\cdot)$ is the Hamming weight of the sequence. All binary sequence additions are done in GF(2)
\begin{proof}
Given $B(x)$ it is possible to write it in a tabular form as shown below.We assume that $B(x)$ is written in its simplest form, which makes $c_0=0$ and it is always located at row zero and column zero. We can see that $u,v$ is just used to determine which column $c_1,c_2$ occur in. Using the knowledge the impulse response, it is possible to find the parity weight of the codeword. 

\begin{center}
\begin{tabular}{| c | c | c |}
\hline
$c_0$  & & \\
\hline
 & & \\
\hline
 & & \\
\hline
 &$c_1$ & \\
\hline
 & & \\
\hline
 & & \\
\hline
& & $c_2$\\
\hline
\end{tabular}
\quad
\begin{tabular}{| c | c | c |}
\hline
$c_0$  & & \\
\hline
 & & \\
\hline
 & & \\
\hline
 & & $c_1$ \\
\hline
 & & \\
\hline
 & & \\
\hline
& $c_2$ & \\
\hline
\end{tabular}
\end{center}
The table below shows a tabular representation of the impulse response with respect to the position of $c_0,c_1~\text{and} ~c_2$ for the table on the left.

\begin{center}
\begin{tabular}{| c | c | c |}
\hline
$1$ & 1 &1 \\
\hline
0 &1 &1 \\
\hline
 0 &1 &1 \\
\hline
 0 &1 &1 \\
\hline
 0 &1 &1 \\
\hline
 0 &1 &1 \\
\hline
0 &1 &1 \\
\hline
\end{tabular}
\quad
\begin{tabular}{| c | c | c |}
\hline
$0$ & 0 &0 \\
\hline
$0$ & 0 &0 \\
\hline
 $0$ & 0 &0 \\
\hline
 0 &1 &1 \\
\hline
 1 &0 &1 \\
\hline
 1 &0 &1 \\
\hline
1 &0 &1 \\
\hline
\end{tabular}
\quad
\begin{tabular}{| c | c | c |}
\hline
$0$ & 0 &0 \\
\hline
$0$ & 0 &0 \\
\hline
 $0$ & 0 &0 \\
\hline
 $0$ & 0 &0 \\
\hline
  $0$ & 0 &0 \\
\hline
 $0$ & 0 &0 \\
\hline
0 &0 &1 \\
\hline
\end{tabular}
\end{center}

To calculate the parity weight, we just need to find the Hamming weight of each row , sum the Hamming weight of each row and then find the cummulative sum. Using the previously defined term, we may rewrite the below in the form below.

\begin{center}
\begin{tabular}{| c |}
\hline
$\alpha_0^{'}$ \\
\hline
$\alpha_0$ \\
\hline
$\alpha_0$ \\
\hline
$\alpha_0$ \\
\hline
 $\alpha_0$ \\
\hline
 $\alpha_0$ \\
\hline
$\alpha_0$ \\
\hline
\end{tabular}
\quad
\begin{tabular}{| c |}
\hline
$\mathbf{0}$  \\
\hline
$\mathbf{0}$ \\
\hline
 $\mathbf{0}$ \\
\hline
 $\alpha_1^{'}$ \\
\hline
$\alpha_1$ \\
\hline
$\alpha_1$  \\
\hline
$\alpha_1$  \\
\hline
\end{tabular}
\quad
\begin{tabular}{| c |}
\hline
$\mathbf{0}$  \\
\hline
$\mathbf{0}$ \\
\hline
 $\mathbf{0}$ \\
\hline
$\mathbf{0}$ \\
\hline
$\mathbf{0}$ \\
\hline
$\mathbf{0}$ \\
\hline
$\alpha_2^{'}$  \\
\hline
\end{tabular}
\end{center}

where $\mathbf{0}=(0 0 0)$. We note that depending on the value of $u,v$, the above table will be slightly different.

Finding the Hamming weight of for each corrsponding row and adding them up results in the equation below 
\begin{equation}
\begin{aligned}
&w_H(\alpha^{'}_0)  +w_H(\alpha_0+\alpha^{'}_1)+w_H(\alpha_0+\alpha^{}_1
+\alpha^{'}_2)\\
&+d_1\left(w_H(\alpha^{}_0)\right)+d_2\left(w_H(\alpha^{}_0 + \alpha^{}_1)\right)
\end{aligned}
\end{equation}

where $d_i,~i=\{1,2\}$ is the number of non-overlapping rows between $c_0,c_1$ and $c_1,c_2$ respectively and is given by

$$d_i =\left \lfloor{\frac{c_i -c_{i-1} -2}{3}}\right \rfloor,~i=\{1,2\}$$

Taking into account the possibility of $c_1,c_2$ appearing in different columns we rewrite the above equation as 

\begin{equation}
\begin{aligned}
&w_H(\alpha^{'}_0)+w_H(\alpha_0+\alpha^{'}_u)+w_H(\alpha_0+\alpha^{}_u
+\alpha^{'}_v)\\
&+d_1\left(w_H(\alpha^{}_0)\right)+d_2\left(w_H(\alpha^{}_0 + \alpha^{}_u)\right)
\end{aligned}
\end{equation}


\end{proof}

\end{document}