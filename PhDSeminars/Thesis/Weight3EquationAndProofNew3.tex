\documentclass[11pt, oneside, dvipdfmx]{book}
\newcommand{\folder}{/usr/local/share/texmf}
%\newcommand{\folder}{/home/hanchenggao/Documents/texmf}
%Page setting
\setlength{\textwidth}{460pt}
\setlength{\topmargin}{-1truemm}
\setlength{\textheight}{650pt}
\setlength{\oddsidemargin}{-0.2truemm}
\setlength{\evensidemargin}{-0.2truemm}

%%%%%%%%%%%%
% 	 UsePackege	%
%%%%%%%%%%%%
\usepackage{amssymb}
\usepackage{amsmath}
\usepackage{amsthm}
\usepackage{ascmac}
\usepackage{lscape}
\usepackage{pifont}
\usepackage{cite}
\usepackage{ifthen}
\usepackage{framed}
\usepackage{mathrsfs}
\usepackage{booktabs}
\usepackage{algorithm}
\usepackage{algorithmic}
\usepackage{comment}
\usepackage{longtable}
% Font
%%%%%%%%%%%%%%
% 	Bold Number	%
%%%%%%%%%%%%%%
\newcommand{\bzero}{{\bf 0}}
\newcommand{\bone}{{\bf 1}}
%%%%%%%%%%%%%%
% 	Bold English	%
%%%%%%%%%%%%%%
% a
\newcommand{\ba}{{\mbox{\boldmath$a$}}}
\newcommand{\bA}{{\mbox{\boldmath$A$}}}
\newcommand{\sba}{{\mbox{\scriptsize\boldmath $a$}}}
% b
\newcommand{\bb}{{\mbox{\boldmath$b$}}}
\newcommand{\bB}{{\mbox{\boldmath$B$}}}
\newcommand{\sbb}{{\mbox{\scriptsize\boldmath $b$}}}
% c
\newcommand{\bc}{{\mbox{\boldmath$c$}}}
\newcommand{\bC}{{\mbox{\boldmath$C$}}}
\newcommand{\sbC}{{\mbox{\scriptsize\boldmath $C$}}}
\newcommand{\sbc}{{\mbox{\scriptsize\boldmath $c$}}}
% d
\newcommand{\bd}{{\mbox{\boldmath$d$}}}
\newcommand{\bD}{{\mbox{\boldmath$D$}}}
\newcommand{\sbd}{{\mbox{\scriptsize\boldmath $d$}}}
% e
\newcommand{\be}{{\mbox{\boldmath$e$}}}
\newcommand{\bE}{{\mbox{\boldmath$E$}}}
\newcommand{\sbe}{{\mbox{\scriptsize\boldmath $e$}}}
% f
\newcommand{\bbf}{{\mbox{\boldmath$f$}}}
\newcommand{\bF}{{\mbox{\boldmath$F$}}}
% g
\newcommand{\bg}{{\mbox{\boldmath$g$}}}
\newcommand{\bG}{{\mbox{\boldmath$G$}}}
% h
\newcommand{\bh}{{\mbox{\boldmath$h$}}}
\newcommand{\bH}{{\mbox{\boldmath$H$}}}
\newcommand{\sbH}{{\mbox{\scriptsize\boldmath$H$}}}
\newcommand{\sbh}{{\mbox{\scriptsize\boldmath$h$}}}
% i
\newcommand{\bi}{{\mbox{\boldmath$i$}}}
\newcommand{\sbi}{{\mbox{\scriptsize \boldmath$i$}}}
\newcommand{\bI}{{\mbox{\boldmath$I$}}}
% i
\newcommand{\bj}{{\mbox{\boldmath$j$}}}
\newcommand{\sbj}{{\mbox{\scriptsize \boldmath$j$}}}
\newcommand{\bJ}{{\mbox{\boldmath$J$}}}
% k
\newcommand{\bk}{{\mbox{\boldmath$k$}}}
\newcommand{\bK}{{\mbox{\boldmath$K$}}}
% l
%\newcommand{\ell}{{\mbox{\boldmath $l$}}}
\newcommand{\bl}{{\mbox{\boldmath$l$}}}
\newcommand{\bL}{{\mbox{\boldmath$L$}}}
% m
\newcommand{\bm}{{\mbox{\boldmath$m$}}}
\newcommand{\sbm}{{\mbox{\scriptsize \boldmath$m$}}}
\newcommand{\bM}{{\mbox{\boldmath$M$}}}
% n
\newcommand{\bn}{{\mbox{\boldmath$n$}}}
\newcommand{\bN}{{\mbox{\boldmath$N$}}}
\newcommand{\sbn}{{\mbox{\scriptsize \boldmath$n$}}}
\newcommand{\sbN}{{\mbox{\scriptsize\boldmath $N$}}}
% o
\newcommand{\bo}{{\mbox{\boldmath$o$}}}
\newcommand{\bO}{{\mbox{\boldmath$O$}}}
% p
\newcommand{\bp}{{\mbox{\boldmath$p$}}}
\newcommand{\bP}{{\mbox{\boldmath$P$}}}
\newcommand{\sbp}{{\mbox{\scriptsize \boldmath$p$}}}
% q
\newcommand{\bq}{{\mbox{\boldmath$q$}}}
\newcommand{\bQ}{{\mbox{\boldmath$Q$}}}
% r
\newcommand{\br}{{\mbox{\boldmath$r$}}}
\newcommand{\bR}{{\mbox{\boldmath$R$}}}
\newcommand{\sbr}{{\mbox{\scriptsize\boldmath$r$}}}
% s
\newcommand{\bs}{{\mbox{\boldmath$s$}}}
\newcommand{\sbs}{{\mbox{\scriptsize \boldmath$s$}}}
\newcommand{\bS}{{\mbox{\boldmath$S$}}}
% t
\newcommand{\bt}{{\mbox{\boldmath$t$}}}
\newcommand{\bT}{{\mbox{\boldmath$T$}}}
% u
\newcommand{\bu}{{\mbox{\boldmath$u$}}}
\newcommand{\bU}{{\mbox{\boldmath$U$}}}
\newcommand{\sbu}{{\mbox{\scriptsize\boldmath $u$}}}
% v
\newcommand{\bv}{{\mbox{\boldmath$v$}}}
\newcommand{\bV}{{\mbox{\boldmath$V$}}}
% w
\newcommand{\bw}{{\mbox{\boldmath$w$}}}
\newcommand{\bW}{{\mbox{\boldmath$W$}}}
\newcommand{\sbW}{{\mbox{\scriptsize\boldmath $W$}}}
\newcommand{\sbw}{{\mbox{\scriptsize\boldmath $w$}}}
% x
\newcommand{\bx}{{\mbox{\boldmath$x$}}}
\newcommand{\bX}{{\mbox{\boldmath$X$}}}
\newcommand{\sbx}{{\mbox{\scriptsize\boldmath $x$}}}
\newcommand{\sbX}{{\mbox{\scriptsize\boldmath $X$}}}
%y
\newcommand{\by}{{\mbox{\boldmath$y$}}}
\newcommand{\bY}{{\mbox{\boldmath$Y$}}}
\newcommand{\sby}{{\mbox{\scriptsize\boldmath $y$}}}
\newcommand{\sbY}{{\mbox{\scriptsize\boldmath $Y$}}}
%z
\newcommand{\bz}{{\mbox{\boldmath$z$}}}
\newcommand{\bZ}{{\mbox{\boldmath$Z$}}}
\newcommand{\sbz}{{\mbox{\scriptsize\boldmath$z$}}}
\newcommand{\sbZ}{{\mbox{\scriptsize\boldmath$Z$}}}


%%%%%%%%%%%
% 	Bold Grik	%
%%%%%%%%%%%
% alpha
\newcommand{\balpha}{{\mbox{\boldmath$\alpha$}}}
\newcommand{\sbalpha}{{\mbox{\scriptsize\boldmath$\alpha$}}}
% beta
\newcommand{\bbeta}{{\mbox{\boldmath$\beta$}}}
\newcommand{\sbbeta}{{\mbox{\scriptsize\boldmath$\beta$}}}
% gamma
\newcommand{\bgamma}{{\mbox{\boldmath$\gamma$}}}
\newcommand{\bGamma}{{\mbox{\boldmath$\Gamma$}}}
\newcommand{\sbgamma}{{\mbox{\scriptsize\boldmath$\gamma$}}}
% delta
\newcommand{\bdelta}{{\mbox{\boldmath$\delta$}}}
\newcommand{\bDelta}{{\mbox{\boldmath$\Delta$}}}
\newcommand{\sbdelta}{{\mbox{\scriptsize\boldmath$\delta$}}}
% epsilon
\newcommand{\bepsilon}{{\mbox{\boldmath$\epsilon$}}}
\newcommand{\bvarepsilon}{{\mbox{\boldmath$\varepsilon$}}}
\newcommand{\sbvarepsilon}{{\mbox{\scriptsize\boldmath$\varepsilon$}}}
% zeta
\newcommand{\bzeta}{{\mbox{\boldmath$\zeta$}}}
% eta
\newcommand{\etab}{{\mbox{\boldmath$\eta$}}}
% theta
\newcommand{\btheta}{{\mbox{\boldmath$\theta$}}}
\newcommand{\bTheta}{{\mbox{\boldmath$\Theta$}}}
\newcommand{\bvartheta}{{\mbox{\boldmath$\vartheta$}}}
% iota
\newcommand{\biota}{{\mbox{\boldmath$\iota$}}}
\newcommand{\sbiota}{{\mbox{\scriptsize\boldmath$\iota$}}}
% kappa
\newcommand{\bkappa}{{\mbox{\boldmath$\kappa$}}}
% lambda
\newcommand{\blambda}{{\mbox{\boldmath$\lambda$}}}
\newcommand{\bLambda}{{\mbox{\boldmath$\Lambda$}}}
% mu
\newcommand{\bmu}{{\mbox{\boldmath$\mu$}}}
\newcommand{\sbmu}{{\mbox{\scriptsize\boldmath$\mu$}}}
% nu
\newcommand{\bnu}{{\mbox{\boldmath$\nu$}}}
\newcommand{\sbnu}{{\mbox{\scriptsize\boldmath$\nu$}}}
% xi
\newcommand{\bxi}{{\mbox{\boldmath$\xi$}}}
\newcommand{\bXi}{{\mbox{\boldmath$\Xi$}}}
% pi
\newcommand{\bpi}{{\mbox{\boldmath$\pi$}}}
\newcommand{\bPi}{{\mbox{\boldmath$\Pi$}}}
\newcommand{\bvarpi}{{\mbox{\boldmath$\varpi$}}}
% rho
\newcommand{\brho}{{\mbox{\boldmath$\rho$}}}
\newcommand{\bvarrho}{{\mbox{\boldmath$\varrho$}}}
% sigma
\newcommand{\bsigma}{{\mbox{\boldmath$\sigma$}}}
\newcommand{\bSigma}{{\mbox{\boldmath$\Sigma$}}}
\newcommand{\bvarsigma}{{\mbox{\boldmath$\varsigma$}}}
% tau
\newcommand{\btau}{{\mbox{\boldmath$\tau$}}}
% upsilon
\newcommand{\bupsilon}{{\mbox{\boldmath$\upsilon$}}}
\newcommand{\bUpsilon}{{\mbox{\boldmath$\Upsilon$}}}
% phi
\newcommand{\bphi}{{\mbox{\boldmath$\phi$}}}
\newcommand{\bPhi}{\mathbf \Phi}
\newcommand{\bvarphi}{{\mbox{\boldmath$\varphi$}}}
% chi
\newcommand{\bchi}{{\mbox{\boldmath$\chi$}}}
% psi
\newcommand{\bpsi}{{\mbox{\boldmath$\psi$}}}
\newcommand{\bPsi}{\mathbf \Psi}
% omega
\newcommand{\bomega}{{\mbox{\boldmath$\omega$}}}
\newcommand{\bOmega}{{\mbox{\boldmath$\Omega$}}}
% nabla
\newcommand{\bnabla}{{\mbox{\boldmath$\nabla$}}}

% calligraphic letters
\newcommand{\cA}{{\cal A}}
\newcommand{\cB}{{\cal B}}
\newcommand{\cC}{{\cal C}}
\newcommand{\cD}{{\cal D}}
\newcommand{\cE}{{\cal E}}
\newcommand{\cF}{{\cal F}}
\newcommand{\cG}{{\cal G}}
\newcommand{\cH}{{\cal H}}
\newcommand{\cI}{{\cal I}}
\newcommand{\cJ}{{\cal J}}
\newcommand{\cK}{{\cal K}}
\newcommand{\cL}{{\cal L}}
\newcommand{\cM}{{\cal M}}
\newcommand{\cN}{{\cal N}}
\newcommand{\cO}{{\cal O}}
\newcommand{\cP}{{\cal P}}
\newcommand{\cQ}{{\cal Q}}
\newcommand{\cR}{{\cal R}}
\newcommand{\cS}{{\cal S}}
\newcommand{\cT}{{\cal T}}
\newcommand{\cU}{{\cal U}}
\newcommand{\cV}{{\cal V}}
\newcommand{\cW}{{\cal W}}
\newcommand{\cX}{{\cal X}}
\newcommand{\cY}{{\cal Y}}
\newcommand{\cZ}{{\cal Z}}
% calligraphic letters
\newcommand{\mA}{{\mathscr A}}
\newcommand{\mB}{{\mathscr B}}
\newcommand{\mC}{{\mathscr C}}
\newcommand{\mD}{{\mathscr D}}
\newcommand{\mE}{{\mathscr E}}
\newcommand{\mF}{{\mathscr F}}
\newcommand{\mG}{{\mathscr G}}
\newcommand{\mH}{{\mathscr H}}
\newcommand{\mI}{{\mathscr I}}
\newcommand{\mJ}{{\mathscr J}}
\newcommand{\mK}{{\mathscr K}}
\newcommand{\mL}{{\mathscr L}}
\newcommand{\mM}{{\mathscr M}}
\newcommand{\mN}{{\mathscr N}}
\newcommand{\mO}{{\mathscr O}}
\newcommand{\mP}{{\mathscr P}}
\newcommand{\mQ}{{\mathscr Q}}
\newcommand{\mR}{{\mathscr R}}
\newcommand{\mS}{{\mathscr S}}
\newcommand{\mT}{{\mathscr T}}
\newcommand{\mU}{{\mathscr U}}
\newcommand{\mV}{{\mathscr V}}
\newcommand{\mW}{{\mathscr W}}
\newcommand{\mX}{{\mathscr X}}
\newcommand{\mY}{{\mathscr Y}}
\newcommand{\mZ}{{\mathscr Z}}
% calligraphic letters
\newcommand{\bbA}{{\mathbb A}}
\newcommand{\bbB}{{\mathbb B}}
\newcommand{\bbC}{\mathbb{C}}
\newcommand{\bbD}{\mathbb{D}}
\newcommand{\bbE}{{\mathbb E}}
\newcommand{\bbF}{{\mathbb F}}
\newcommand{\bbG}{{\mathbb G}}
\newcommand{\bbH}{{\mathbb H}}
\newcommand{\bbI}{{\mathbb I}}
\newcommand{\bbJ}{{\mathbb J}}
\newcommand{\bbK}{{\mathbb K}}
\newcommand{\bbL}{\mathbb{L}}
\newcommand{\bbM}{{\mathbb M}}
\newcommand{\bbN}{\mathbb{N}}
\newcommand{\bbO}{{\mathbb O}}
\newcommand{\bbP}{\mathbb{P}}
\newcommand{\bbQ}{{\mathbb Q}}
\newcommand{\bbR}{\mathbb{R}}
\newcommand{\bbS}{\mathbb{S}}
\newcommand{\bbT}{{\mathbb T}}
\newcommand{\bbU}{{\mathbb U}}
\newcommand{\bbV}{\mathbb{V}}
\newcommand{\bbW}{{\mathbb W}}
\newcommand{\bbX}{\mathbb{X}}
\newcommand{\bbY}{\mathbb{Y}}
\newcommand{\bbZ}{{\mathbb Z}}
%%%%%%%%%%%%%%
% 	Tilde English	%
%%%%%%%%%%%%%%
% a
\newcommand{\tila}{\tilde{a}}
\newcommand{\tilA}{\tilde{A}}
% b
\newcommand{\tilb}{\tilde{b}}
\newcommand{\tilB}{\tilde{B}}
% c
\newcommand{\tilc}{\tilde{c}}
\newcommand{\tilC}{\tilde{C}}
% d
\newcommand{\tild}{\tilde{d}}
\newcommand{\tilD}{\tilde{D}}
% e
\newcommand{\tile}{\tilde{e}}
\newcommand{\tilE}{\tilde{E}}
% f
\newcommand{\tilf}{\tilde{f}}
\newcommand{\tilF}{\tilde{F}}
% g
\newcommand{\tilg}{\tilde{g}}
\newcommand{\tilG}{\tilde{G}}
% h
\newcommand{\tilh}{\tilde{h}}
\newcommand{\tilbh}{\tilde\bh}
\newcommand{\tilH}{\tilde{H}}
% i
\newcommand{\tili}{\tilde{i}}
\newcommand{\tilI}{\tilde{I}}
% i
\newcommand{\tilj}{\tilde{j}}
\newcommand{\tilJ}{\tilde{J}}
% k
\newcommand{\tilk}{\tilde{k}}
\newcommand{\tilK}{\tilde{K}}
% l
\newcommand{\till}{\tilde{l}}
\newcommand{\tilL}{\tilde{L}}
% m
\newcommand{\tilm}{\tilde{m}}
\newcommand{\tilM}{\tilde{M}}
% n
\newcommand{\tiln}{\tilde{n}}
\newcommand{\tilN}{\tilde{N}}
% o
\newcommand{\tilo}{\tilde{o}}
\newcommand{\tilO}{\tilde{O}}
% p
\newcommand{\tilp}{\tilde{p}}
\newcommand{\tilP}{\tilde{P}}
% q
\newcommand{\tilq}{\tilde{q}}
\newcommand{\tilQ}{\tilde{Q}}
\newcommand{\tilbq}{\tilde{\bq}}
% r
\newcommand{\tilr}{\tilde{r}}
\newcommand{\tilR}{\tilde{R}}
\newcommand{\tilcR}{\tilde{\cR}}
\newcommand{\tilbr}{\tilde{\br}}
% s
\newcommand{\tils}{\tilde{s}}
\newcommand{\tilS}{\tilde{S}}
% t
\newcommand{\tilt}{\tilde{t}}
\newcommand{\tilT}{\tilde{T}}
% u
\newcommand{\tilu}{\tilde{u}}
\newcommand{\tilU}{\tilde{U}}
% v
\newcommand{\tilv}{\tilde{v}}
\newcommand{\tilV}{\tilde{V}}
% w
\newcommand{\tilw}{\tilde{w}}
\newcommand{\tilW}{\tilde{W}}
% x
\newcommand{\tilx}{\tilde{x}}
\newcommand{\tilbx}{\tilde{\bx}}
\newcommand{\tilX}{\tilde{X}}
\newcommand{\tilbX}{\tilde{\bX}}
%y
\newcommand{\tily}{\tilde{y}}
\newcommand{\tilY}{\tilde{Y}}
\newcommand{\tilby}{\tilde{\by}}
\newcommand{\tilbY}{\tilde{\bY}}
%z
\newcommand{\tilz}{\tilde{z}}
\newcommand{\tilZ}{\tilde{Z}}
\newcommand{\tilbZ}{\tilde{\bZ}}


%%%%%%%%%%%%
%% 	Bold Grik	%
%%%%%%%%%%%%
%% alpha
\newcommand{\tilalpha}{\tilde{\alpha}}
%\newcommand{\sbalpha}{{\mbox{\scriptsize\boldmath$\alpha$}}}
%% beta
\newcommand{\tilbeta}{{\tilde{\beta}}}
%\newcommand{\sbbeta}{{\mbox{\scriptsize\boldmath$\beta$}}}
%% gamma
\newcommand{\tilgamma}{{\tilde{\gamma}}}
%\newcommand{\bGamma}{{\mbox{\boldmath$\Gamma$}}}
%\newcommand{\sbgamma}{{\mbox{\scriptsize\boldmath$\gamma$}}}
%% delta
%\newcommand{\bdelta}{{\mbox{\boldmath$\delta$}}}
%\newcommand{\bDelta}{{\mbox{\boldmath$\Delta$}}}
%\newcommand{\sbdelta}{{\mbox{\scriptsize\boldmath$\delta$}}}
%% epsilon
%\newcommand{\bepsilon}{{\mbox{\boldmath$\epsilon$}}}
%\newcommand{\bvarepsilon}{{\mbox{\boldmath$\varepsilon$}}}
%\newcommand{\sbvarepsilon}{{\mbox{\scriptsize\boldmath$\varepsilon$}}}
%% zeta
%\newcommand{\bzeta}{{\mbox{\boldmath$\zeta$}}}
%% eta
%\newcommand{\etab}{{\mbox{\boldmath$\eta$}}}
%% theta
%\newcommand{\btheta}{{\mbox{\boldmath$\theta$}}}
%\newcommand{\bTheta}{{\mbox{\boldmath$\Theta$}}}
%\newcommand{\bvartheta}{{\mbox{\boldmath$\vartheta$}}}
%% iota
%\newcommand{\biota}{{\mbox{\boldmath$\iota$}}}
%\newcommand{\sbiota}{{\mbox{\scriptsize\boldmath$\iota$}}}
%% kappa
%\newcommand{\bkappa}{{\mbox{\boldmath$\kappa$}}}
%% lambda
%\newcommand{\blambda}{{\mbox{\boldmath$\lambda$}}}
%\newcommand{\bLambda}{{\mbox{\boldmath$\Lambda$}}}
%% mu
%\newcommand{\bmu}{{\mbox{\boldmath$\mu$}}}
%\newcommand{\sbmu}{{\mbox{\scriptsize\boldmath$\mu$}}}
%% nu
%\newcommand{\bnu}{{\mbox{\boldmath$\nu$}}}
%\newcommand{\sbnu}{{\mbox{\scriptsize\boldmath$\nu$}}}
%% xi
%\newcommand{\bxi}{{\mbox{\boldmath$\xi$}}}
%\newcommand{\bXi}{{\mbox{\boldmath$\Xi$}}}
%% pi
%\newcommand{\bpi}{{\mbox{\boldmath$\pi$}}}
%\newcommand{\bPi}{{\mbox{\boldmath$\Pi$}}}
%\newcommand{\bvarpi}{{\mbox{\boldmath$\varpi$}}}
%% rho
%\newcommand{\brho}{{\mbox{\boldmath$\rho$}}}
%\newcommand{\bvarrho}{{\mbox{\boldmath$\varrho$}}}
%% sigma
%\newcommand{\bsigma}{{\mbox{\boldmath$\sigma$}}}
%\newcommand{\bSigma}{{\mbox{\boldmath$\Sigma$}}}
%\newcommand{\bvarsigma}{{\mbox{\boldmath$\varsigma$}}}
% tau
\newcommand{\tiltau}{{\mbox{$\tilde{\tau}$}}}
%% upsilon
%\newcommand{\bupsilon}{{\mbox{\boldmath$\upsilon$}}}
%\newcommand{\bUpsilon}{{\mbox{\boldmath$\Upsilon$}}}
%% phi
%\newcommand{\bphi}{{\mbox{\boldmath$\phi$}}}
%\newcommand{\bPhi}{\mathbf \Phi}
%\newcommand{\bvarphi}{{\mbox{\boldmath$\varphi$}}}
%% chi
%\newcommand{\bchi}{{\mbox{\boldmath$\chi$}}}
%% psi
%\newcommand{\bpsi}{{\mbox{\boldmath$\psi$}}}
%\newcommand{\bPsi}{\mathbf \Psi}
%% omega
%\newcommand{\bomega}{{\mbox{\boldmath$\omega$}}}
%\newcommand{\bOmega}{{\mbox{\boldmath$\Omega$}}}

%%%%%%%%%%%%%%
% 	Tilde English	%
%%%%%%%%%%%%%%
% a
\newcommand{\bara}{\bar{a}}
\newcommand{\barA}{\bar{A}}
% b
\newcommand{\barb}{\bar{b}}
\newcommand{\barB}{\bar{B}}
% c
\newcommand{\barc}{\bar{c}}
\newcommand{\barC}{\bar{C}}
% d
\newcommand{\bard}{\bar{d}}
\newcommand{\barD}{\bar{D}}
% e
\newcommand{\bare}{\bar{e}}
\newcommand{\barE}{\bar{E}}
% f
\newcommand{\barbf}{\bar{f}}
\newcommand{\barF}{\bar{F}}
% g
\newcommand{\barg}{\bar{g}}
\newcommand{\barG}{\bar{G}}
% h
\newcommand{\barh}{\bar{h}}
\newcommand{\barbh}{\bar\bh}
\newcommand{\barH}{\bar{H}}
% i
\newcommand{\bari}{\bar{i}}
\newcommand{\barI}{\bar{I}}
% i
\newcommand{\barj}{\bar{j}}
\newcommand{\barJ}{\bar{J}}
% k
\newcommand{\bark}{\bar{k}}
\newcommand{\barK}{\bar{K}}
% l
\newcommand{\barl}{\bar{l}}
\newcommand{\barL}{\bar{L}}
% m
\newcommand{\barm}{\bar{m}}
\newcommand{\barM}{\bar{M}}
% n
\newcommand{\barn}{\bar{n}}
\newcommand{\barN}{\bar{N}}
% o
\newcommand{\baro}{\bar{o}}
\newcommand{\barO}{\bar{O}}
% p
\newcommand{\barp}{\bar{p}}
\newcommand{\barP}{\bar{P}}
% q
\newcommand{\barq}{\bar{q}}
\newcommand{\barQ}{\bar{Q}}
% r
\newcommand{\barr}{\bar{r}}
\newcommand{\barR}{\bar{R}}
% s
\newcommand{\bars}{\bar{s}}
\newcommand{\barS}{\bar{S}}
% t
\newcommand{\bart}{\bar{t}}
\newcommand{\barT}{\bar{T}}
% u
\newcommand{\baru}{\bar{u}}
\newcommand{\barU}{\bar{U}}
% v
\newcommand{\barv}{\bar{v}}
\newcommand{\barV}{\bar{V}}
% w
\newcommand{\barw}{\bar{w}}
\newcommand{\barW}{\bar{W}}
\newcommand{\barbw}{\bar{\bw}}
% x
\newcommand{\barx}{\bar{x}}
\newcommand{\barX}{\bar{X}}
%y
\newcommand{\bary}{\bar{y}}
\newcommand{\barY}{\bar{Y}}
%z
\newcommand{\barz}{\bar{z}}
\newcommand{\barZ}{\bar{Z}}
%%%%%%%%%%%%%%
% 	Hat English	%
%%%%%%%%%%%%%%
% a
\newcommand{\hata}{\hat{a}}
\newcommand{\hatA}{\hat{A}}
% b
\newcommand{\hatb}{\hat{b}}
\newcommand{\hatB}{\hat{B}}
% c
\newcommand{\hatc}{\hat{c}}
\newcommand{\hatC}{\hat{C}}
\newcommand{\hatbc}{\hat{\bc}}
% d
\newcommand{\hatd}{\hat{d}}
\newcommand{\hatD}{\hat{D}}
% e
\newcommand{\hate}{\hat{e}}
\newcommand{\hatE}{\hat{E}}
% f
\newcommand{\hatbf}{\hat{f}}
\newcommand{\hatF}{\hat{F}}
% g
\newcommand{\hatg}{\hat{g}}
\newcommand{\hatG}{\hat{G}}
% h
\newcommand{\hath}{\hat{h}}
\newcommand{\hatbh}{\hat\bh}
\newcommand{\hatH}{\hat{H}}
% i
\newcommand{\hati}{\hat{i}}
\newcommand{\hatI}{\hat{I}}
% i
\newcommand{\hatj}{\hat{j}}
\newcommand{\hatJ}{\hat{J}}
% k
\newcommand{\hatk}{\hat{k}}
\newcommand{\hatK}{\hat{K}}
% l
\newcommand{\hatl}{\hat{l}}
\newcommand{\hatL}{\hat{L}}
% m
\newcommand{\hatm}{\hat{m}}
\newcommand{\hatM}{\hat{M}}
% n
\newcommand{\hatn}{\hat{n}}
\newcommand{\hatN}{\hat{N}}
% o
\newcommand{\hato}{\hat{o}}
\newcommand{\hatO}{\hat{O}}
% p
\newcommand{\hatp}{\hat{p}}
\newcommand{\hatP}{\hat{P}}
% q
\newcommand{\hatq}{\hat{q}}
\newcommand{\hatQ}{\hat{Q}}
% r
\newcommand{\hatr}{\hat{r}}
\newcommand{\hatR}{\hat{R}}
\newcommand{\hatbr}{\hat{\br}}
% s
\newcommand{\hats}{\hat{s}}
\newcommand{\hatS}{\hat{S}}
% t
\newcommand{\hatt}{\hat{t}}
\newcommand{\hatT}{\hat{T}}
% u
\newcommand{\hatu}{\hat{u}}
\newcommand{\hatU}{\hat{U}}
% v
\newcommand{\hatv}{\hat{v}}
\newcommand{\hatV}{\hat{V}}
% w
\newcommand{\hatw}{\hat{w}}
\newcommand{\hatW}{\hat{W}}
% x
\newcommand{\hatx}{\hat{x}}
\newcommand{\hatX}{\hat{X}}
\newcommand{\hatbX}{\hat{\bX}}
%y
\newcommand{\haty}{\hat{y}}
\newcommand{\hatY}{\hat{Y}}
%z
\newcommand{\hatz}{\hat{z}}
\newcommand{\hatZ}{\hat{Z}}
%%%%%%%%%%%%%%
% 	Tilde English	%
%%%%%%%%%%%%%%
% A
\newcommand{\mathall}{{\rm all}}
\newcommand{\mathand}{{\rm and}}
% B
\newcommand{\BER}{{\rm BER}}
% C
\newcommand{\Cov}{{\rm Cov}}
\newcommand{\conv}{{\rm conv}}
% D
\newcommand{\diver}{\mathrm{div~}}
% E
\newcommand{\eg}{\textit{e.g.}}
\newcommand{\etc}{\textit{etc.}}
\newcommand{\etal}{\textit{et al.}}
\newcommand{\iid}{\textit{i.i.d.}}
\newcommand{\ie}{\textit{i.e.}}
% F
\newcommand{\mathfor}{{\rm for}}
% G
\newcommand{\GF}{{\rm GF}}
\newcommand{\grad}{\mathrm{grad~}}
% I

\newcommand{\mathif}{{\rm if}}
% O
\newcommand{\opt}{{\rm opt}}
\newcommand{\mathor}{{\rm or}}
\newcommand{\mathotherwise}{{\rm otherwise}}
% R
\newcommand{\rot}{\mathrm{rot~}}
% S
\newcommand{\SER}{{\rm SER}}
\newcommand{\st}{~\mathrm{s.t.}~}
% T
%\newcommand{\th}{\mathrm{th}}
% V
\newcommand{\Var}{{\rm Var}}





% Environment
%%%%%%%%%%%%
% 	 Operations 		%
%%%%%%%%%%%%
\DeclareMathOperator{\diag}{diag}
\DeclareMathOperator*{\argmin}{arg~min}
\DeclareMathOperator*{\argmax}{arg~max}
\DeclareMathOperator*{\minmax}{min~max}
\DeclareMathOperator*{\maxmin}{max~min}
\DeclareMathOperator{\lcm}{lcm}
\DeclareMathOperator{\mtxop}{mtx}
\DeclareMathOperator{\vecop}{vec}
\DeclareMathOperator{\sinc}{sinc}
\DeclareMathOperator{\sgn}{sgn}
\DeclareMathOperator{\rank}{rank}
\DeclareMathOperator{\tr}{tr}
\DeclareMathOperator{\sig}{sig}
% New Environment
\newenvironment{MyDefinition}[1]
    {\begin{itembox}[l]{\bf #1}
    \begin{definition}}%
    {\end{definition}
    \end{itembox}}
    
\newenvironment{MyTheorem}
    {\begin{shadebox}
    \bigskip
    \begin{theorem}}%
    {\end{theorem}
    \smallskip
    \end{shadebox}}
 
\newenvironment{MyProof}
    {\begin{boxnote}
    	%\begin{quote}
    %\begin{proof}
    }
    {%\end{proof}
    %\end{quote}
\end{boxnote}}
 
 \newenvironment{MyExample}
 {\begin{shaded}
 		\begin{quote}
 		\begin{example}}%
 		{\end{example}
   \end{quote}
\end{shaded}}
  
\newenvironment{MyLemma}
    {\begin{doublebox}
    \begin{lemma}}%
    {\end{lemma}
    \end{doublebox}}
    

    
% References
%\newcommand{\IEEE_L_COM}{{IEEE} Commun. Lett.~}

\newcommand{\GCOM}{{IEEE} Global Telecommun. Conf.~}
\newcommand{\ICC}{{IEEE} Int. Conf. Commun.~}
\newcommand{\ISIT}{{IEEE} Int. Symp. Infor. Theory~}
\newcommand{\IWSDA}{{IEEE} Int. Workshop Signal Design and Its Applications in Commun.~}
\newcommand{\MILCOM}{{IEEE} Military Commun. Conf.~}
\newcommand{\PIMRC}{{IEEE} Int. Symp. Personal, Indoor and Mobile Radio Commun.~}
\newcommand{\VTC}{{IEEE} Veh. Tech. Conf.~}
\newcommand{\WCNC}{{IEEE} Wireless Commun. Networking Conf.~}
\bibliographystyle{IEEEtran}
% New Theorems
\theoremstyle{definition}
% Theorem style includes plain, definition, and remark
\newtheorem{theorem}{Theorem}
%\newtheorem{algorithm}{Algorithm}
\newtheorem{definition}{Definition}
\newtheorem{example}{Example}
\newtheorem{conjecture}{Conjecture}
\newtheorem{criterion}{Criterion}
\newtheorem{lemma}{Lemma}
\newtheorem{proposition}{Proposition}
\newtheorem{corollary}{Corollary}
\newtheorem{assumption}{Assumption}
\newtheorem{remark}{Remark}
\newtheorem{problem}{Problem}[section]
\usepackage[dvipdfmx]{color}
\usepackage[dvipdfmx]{graphicx}
\usepackage[sectionbib]{chapterbib}
%\renewcommand{\bibname}{参考情報}
\definecolor{shadecolor}{rgb}{0.9,0.9,0.9}
\makeatletter
\def\th@plain{\upshape}
\makeatother

\renewcommand{\theequation}{\arabic{chapter}-\arabic{equation}}
\renewcommand{\thefigure}{\arabic{chapter}-\arabic{figure}}
\usepackage {graphicx}
\usepackage {graphics}
\usepackage {graphics}
\title{Formula to Calculate Weight for Low-Weight Weight 3 Inputs and Proof} 
\author{Kwame Ackah Bohulu}
\date{\today}
\begin{document}
\maketitle

\newpage


\section{Equation and Proof}
%\begin{theorem}
%Let $Q(x) =x^{a\tau+t}(1+x^{\beta \tau +1}+x^{\gamma \tau +2})$ be the polynomial representation of a weight $3$ RTZ input.
%The Hamming weight, $w_H$ of a turbo codeword generated by a weight-$3$ RTZ input is given by 
%\begin{equation}
%7+2(\max\{l_1,l_2\}+\max\{l^{\prime}_1,l^{\prime}_2\})
%\end{equation}
%\end{theorem}
\begin{proof}
Since the impulse response is
\[
(1~1~1~0~1~1~0~1~1~0~\cdots)
\]
Let \\$\bphi_1=(0~0~1),~\bphi'_1=(0~1~0),~\bphi''_1=(1~0~0)$, \\
$\bphi_2=(0~1~1),~\bphi'_2=(1~1~0),~\bphi''_2=(1~0~1)$, \\
$\bphi_3=(1~1~1)$. 

Then, the weight-3 RTZ occurs since $\bphi_2+\bphi'_2+\bphi''_2=\bzero_3$. 

Now, we consider the weight of the vector derived by the sumation of the followings vectors.
\begin{eqnarray*}
(\bzero_{3i}~\bphi_1~\bphi'_2~\cdots)\cr
(\bzero_{3j}~\bphi_2~\bphi''_2~\cdots)\cr
(\bzero_{3k}~\bphi_3~\bphi_2~\cdots)
\end{eqnarray*}
To simplify calculation, we have included an addition table for all the vectors which is shown in Table \ref{tb1}

\begin{table}[h!]
\centering
\begin{tabular}{c || c  | c  | c  | c  | c  | c  | c } 
 $$ & $\bphi_1$ & $\bphi'_1$ & $\bphi''_1$ & $\bphi_2$ & $\bphi'_2$ & $\bphi''_2$ & $\bphi_3$ \\
   \hline\hline
   %row1
$\bphi_1$ & $\bzero_3$ & $-$ & $-$ & $-$ & $-$ & $-$ & $-$ \\
   \hline
      %row2
$\bphi'_1$ & $\bphi_2$ & $\bzero_3$ & $-$ & $-$ & $-$ & $-$ & $-$ \\
   \hline
      %row3
$\bphi''_1$ & $\bphi''_2$ & $\bphi'_2$ & $\bzero_3$ & $-$ & $-$ & $-$ & $-$ \\
   \hline
      %row4
$\bphi_2$ & $\bphi'_1$ & $\bphi_1$ & $\bphi_3$ & $\bzero_3$ & $-$ & $-$ & $-$ \\
   \hline
      %row5
$\bphi'_2$ & $\bphi_3$ & $\bphi''_1$ & $\bphi'_1$ & $\bphi''_2$ & $\bzero_3$ & $-$ & $-$ \\
   \hline
      %row6
$\bphi''_2$ & $\bphi''_1$ & $\bphi_3$ & $\bphi_1$ & $\bphi'_2$ & $\bphi_2$ & $\bzero_3$ & $-$ \\
   \hline
      %row7
$\bphi_3$ & $\bphi'_2$ & $\bphi''_2$ & $\bphi_2$ & $\bphi''_1$ & $\bphi_1$ & $\bphi'_1$ & $\bzero_3$ \\
   \hline
  \end{tabular}
\caption{Truth Table}
\label{tb1}
\end{table}
Furthermore, we consider 4 general cases for all possible values of $i,j,k$ These cases are $(=~=),~(=~<),~(<~=)$ and $(<~<)$
\paragraph{Case 0: $i=j=k$ \newline}

 For this case, the vectors to sum will be 
 \begin{align*}
(\bzero_{3i}~\bphi_1~\bphi'_2~\cdots)\\
(\bzero_{3j}~\bphi_2~\bphi''_2~\cdots)\\
(\bzero_{3k}~\bphi_3~\bphi_2~\cdots)\\
\cline{1-2}
(\bzero_{3i}~\bphi''_2~\bzero_{3}~\cdots)
\end{align*}
 
and  the derived vector will be $(\bzero_{3i}~\bphi''_2~\bzero_{3}~\cdots)$ with a weight of $w_p=2$
 
 %========case =  < ===========
 
 \paragraph{Case 1a: $i=j<k$\newline}
 vector to sum:
 \begin{align*}
 (\bzero_{3}~\cdots~\bzero_{3}~\bphi_1~\bphi'_2~\cdots~\bphi_2'~\bphi_2'~\bphi_2'~\cdots)\\
 (\bzero_{3}~\cdots~\bzero_{3}~\bphi_2~\bphi''_2~\cdots~\bphi''_2~\bphi''_2\bphi''_2~\cdots)\\
+(\bzero_{3}~~\cdots~\cdots~\cdots~\cdots~\bzero_{3}~\bphi_3~\bphi_2~\cdots)\\
\cline{1-2}
(\bzero_{3}~\cdots~\bzero_{3}~\bphi'_1~\bphi_2~\cdots~\bphi_2~\bphi''_1~\bzero_3~\cdots)
\end{align*}
derived vector : $(\bzero_{3j}~\bphi'_1~(\bphi_2)_{k-j-1}~\bphi''_1~\bzero_3~\cdots)$
\newline
Parity weight: \begin{equation}
\begin{split}
w_p=2(k-j)
\end{split}
\end{equation}

\paragraph{Case 1b: $i=k<j$ \newline}
 vectors to sum:
 \begin{align*}
(\bzero_{3}~\cdots~\bzero_{3}~\bphi_1~\bphi'_2~\bphi'_2~\bphi'_2~\bphi'_2~\cdots)\\
(\bzero_{3}\cdots\cdots~\cdots~\cdots~\bzero_{3}~\bphi_2~\bphi''_2~\cdots)\\
+(\bzero_{3}~\cdots~\bzero_{3}~\bphi_3~\bphi_2~\bphi_2~\bphi_2~\bphi_2~\cdots)\\
\cline{1-2}
(\bzero_{3}~\cdots~\bzero_{3}~\bphi'_2~\bphi''_2~\bphi''_2~\bphi'_2~\bzero_3~\cdots)
\end{align*}
derived vector : $(\bzero_{3i}~\bphi'_2~(\bphi''_2)_{j-k-1}~\bphi'_2~\bzero_3~\cdots)$
\newline
Parity weight: \begin{equation}
\begin{split}
w_p=2(j-i)+2
\end{split}
\end{equation}

\paragraph{Case 1c: $j=k<i$\newline}
 vectors to sum:
\begin{align*}
(\bzero_{3}~~\cdots~\cdots~\cdots~\cdots~\bzero_{3}~\bphi_1~\bphi'_2~\cdots)\\
(\bzero_{3}~\cdots~\bzero_{3}~\bphi_2~\bphi''_2~\cdots~\bphi''_2~\bphi''_2\bphi''_2~\cdots)\\
+(\bzero_{3}~\cdots~\bzero_{3}~\bphi_3~\bphi_2~\cdots~\bphi_2~\bphi_2~\bphi_2~\cdots)\\
\cline{1-2}
(\bzero_{3}~\cdots~\bzero_{3}~\bphi''_1~\bphi'_2~\cdots~\bphi'_2~\bphi_3~\bzero_3~\cdots)
\end{align*}
derived vector : $(\bzero_{3j}~\bphi''_1~(\bphi_2)_{i-j-1}~\bphi_3~\bzero_3~\cdots)$\newline
Parity weight: \begin{equation}
\begin{split}
w_p=2(i-j)+2
\end{split}
\end{equation}
\newpage
%========case < = ==========
\paragraph{Case 2a: $i<j=k$\newline}
 vectors to sum:
\begin{align*}
(\bzero_3 ~\cdots~\bphi_1~\bphi'_2~\cdots~\bphi'_2~\bphi'_2~\bphi'_2~\cdots)\\
(\bzero_3~\cdots~\cdots~\cdots~\cdots~\bzero_3~\phi_2~\phi''_2~\cdots)\\
+(\bzero_3~\cdots~\cdots~\cdots~\cdots~\bzero_3~\phi_3~\phi_2~\cdots)\\
\cline{1-2}
(\bzero_{3}~\cdots~\bzero_{3}\bphi_1~\bphi'_2~\cdots~\bphi'_2~\bphi'_1~\bzero_3~\cdots)
\end{align*}
derived vector : $(\bzero_{3i}~\bphi_1~(\bphi'_2)_{k-i-1}~\bphi'_1~\bzero_3~\cdots)$
\newline
Parity weight: \begin{equation}
\begin{split}
w_p=2(k-i)
\end{split}
\end{equation}

\paragraph{Case 2b: $j<k=i$ \newline}
 vectors to sum:
\begin{align*}
(\bzero_3~\cdots~\cdots~\cdots~\cdots~\bzero_3~\phi_1~\phi'_2~\cdots)\\
(\bzero_3~\cdots~\bzero_3~\phi_2~\phi''_2~\cdots~\phi''_2~\phi''_2~\phi''_2~\cdots)\\
+(\bzero_3~\cdots~\cdots~\cdots~\cdots~\bzero_3~\phi_3~\phi_2~\cdots)\\
\cline{1-2}
(\bzero_{3}~\cdots\bzero_3~\bphi_2~\bphi''_2~\cdots~\bphi''_2~\bphi_2~\bzero_3~\cdots)
\end{align*}
derived vector : $(\bzero_{3j}~\bphi_2~\bphi''_2)_{i-j-1}~\bphi_2~\bzero_3~\cdots)$\newline
Parity weight: \begin{equation}
\begin{split}
w_p=2(i-j)+2
\end{split}
\end{equation}

\paragraph{Case 2c: $k<i=j$ \newline}
 vectors to sum:
\begin{align*}
(\bzero_3~\cdots~\cdots~\cdots~\cdots~\bzero_3~\phi_1~\phi'_2~\cdots)\\
(\bzero_3~\cdots~\cdots~\cdots~\cdots~\bzero_3~\phi_2~\phi''_2~\cdots)\\
+(\bzero_3~\cdots~\bzero_3~\phi_3~\phi_2~\cdots~\phi_2~\phi_2~\phi_2~\cdots)\\
\cline{1-2}
(\bzero_{3}~\cdots\bzero_3~\bphi_3~\bphi_2~\cdots~\bphi_2~\bphi_1~\bzero_3~\cdots)
\end{align*}


derived vector : $(\bzero_{3k}~\bphi_3~(\bphi_2)_{j-k-1}~\bphi_1~\bzero_3~\cdots)$
\newline
Parity weight: \begin{equation}
\begin{split}
w_p=2(j-k)+2
\end{split}
\end{equation}
\newpage
 %=====case < <==========
 \paragraph{Case 3a: $i<j<k$\newline}
  vectors to sum:
\begin{eqnarray*}
(\bzero_3~\cdots~\bzero_3~\phi_1~\phi'_2~\cdots~\phi'_2~\phi'_2~\phi'_2~\cdots~\phi'_2~\phi'_2~\phi'_2~\cdots)\cr
(\bzero_3~\cdots~\bzero_3~\bzero_3~\bzero_3~\cdots~\bzero_3~\phi_2~\phi''_2~\cdots~\phi''_2~\phi''_2~\phi''_2~\cdots)\cr
+(\bzero_3~\cdots~\bzero_3~\bzero_3~\bzero_3~\cdots~\cdots~\cdots~\cdots~\bzero_3~\phi_3~\phi_2~\cdots)\cr
\cline{1-2}
(\bzero_{3}~\cdots~\bzero_3~\bphi_1~\bphi'_2\cdots\bphi'_2~\bphi''_2~\bphi_2~\cdots~\bphi_2~\bphi''_1~\bzero_3~\cdots)
\end{eqnarray*}


derived vector : $(\bzero_{3i}~\bphi_1~(\bphi'_2)_{j-i-1}~\bphi''_2~(\bphi_2)_{k-j-1}~\bphi''_1~\bzero_3~\cdots)$
\newline
Parity weight: \begin{equation}
\begin{split}
w_p&=2(j-i)+1+2(k-j-1)+1\\
&=2(k-i)
\end{split}
\end{equation}

\paragraph{Case 3b: $i<k<j$ \newline}
  vectors to sum:
\begin{eqnarray*}
(\bzero_3~\cdots~\bzero_3~\phi_1~\phi'_2~\cdots~\phi'_2~\phi'_2~\phi'_2~\cdots~\phi'_2~\phi'_2~\phi'_2\cdots)\cr
(\bzero_3~\cdots~\bzero_3~\bzero_3~\bzero_3~\cdots~\cdots~\cdots~\cdots~\bzero_3~\phi_2~\phi''_2\cdots)\cr
+(\bzero_3~\cdots~\bzero_3~\bzero_3~\bzero_3~\cdots~\bzero_3~\phi_3~\phi_2~\cdots~\phi_2~\phi_2~\phi_2\cdots)\cr
\cline{1-2}
(\bzero_{3}~\cdots~\bzero_3~\bphi_1~\bphi'_2\cdots~\bphi'_2~\bphi_1~\bphi''_2~\cdots~\bphi''_2~\bphi'_2~\bzero_3\cdots)
\end{eqnarray*}

derived vector : $(\bzero_{3i}~\bphi_1~(\bphi'_2)_{k-i-1}~\bphi_1~(\bphi''_2)_{j-i-1}~\bphi'_2~\bzero_3~\cdots)$
\newline
Parity weight: \begin{equation}
\begin{split}
w_p&=2(k-i)+2(j-k)\\
&=2(j-i)
\end{split}
\end{equation}


\paragraph{Case 3c: $j<k<i$ \newline}
 vectors to sum:
\begin{eqnarray*}
(\bzero_3~\cdots~\bzero_3~\bzero_3~\bzero_3~\cdots~\cdots~\cdots~\cdots~\bzero_3~\phi_1~\phi'_2\cdots)\cr
(\bzero_3~\cdots~\bzero_3~\phi_2~\phi''_2~\cdots~\phi''_2~\phi''_2~\phi''_2~\cdots~\phi''_2~\phi''_2~\phi''_2\cdots)\cr
+(\bzero_3~\cdots\bzero_3~\bzero_3~\bzero_3~\cdots~\bzero_3~\phi_3~\phi_2~\cdots~\phi_2~\phi_2~\phi_2\cdots)\cr
\cline{1-2}
(\bzero_{3}~\cdots~\bzero_3~\bphi_2~\bphi''_2\cdots~\bphi''_2~\bphi'_1~\bphi'_2\cdots~\bphi'_2~\bphi_3~\bzero_3\cdots)
\end{eqnarray*}
derived vector : $(\bzero_{3k}~\bphi_2~(\bphi''_2)_{k-j-1}~\bphi'_1~(\bphi'_2)_{i-k-1}~\bphi_3~\bzero_3~\cdots)$\newline
Parity weight: \begin{equation}
\begin{split}
w_p &=2(k-j)+1 +2(i-k)+1 \\
&=2(i-j)+2
\end{split}
\end{equation}
\newpage
\paragraph{Case 3d: $j<i<k$\newline}
 vectors to sum:
\begin{eqnarray*}
(\bzero_3~\cdots\bzero_3~\bzero_3~\bzero_3~\cdots~\bzero_3~\phi_1~\phi'_2~\cdots~\phi'_2~\phi'_2~\phi'_2\cdots)\cr
(\bzero_3~\cdots~\bzero_3~\phi_2~\phi''_2~\cdots~\phi''_2~\phi''_2~\phi''_2~\cdots~\phi''_2~\phi''_2~\phi''_2\cdots)\cr
+(\bzero_3~\cdots~\bzero_3~\bzero_3~\bzero_3~~\cdots~\cdots~\cdots~\cdots~\bzero_3~\phi_3~\phi_2\cdots)\cr
\cline{1-2}
(\bzero_{3}~\cdots~\bzero_3~\bphi_2~\bphi''_2\cdots~\bphi''_2~\bphi''_1~\bphi_2\cdots~\bphi_2~\bphi''_1~\bzero_3\cdots)
\end{eqnarray*}
derived vector : $(\bzero_{3j}~\bphi_2~(\bphi''_2)_{i-j-1}~\bphi''_1~(\bphi_2)_{k-i-1}~\bphi''_1~\bzero_3~\cdots)$
\newline
Parity weight: \begin{equation}
\begin{split}
w_p&=2(i-j)+1+2(k-i-1)+1\\
&=2(k-j)
\end{split}
\end{equation}


\paragraph{Case 3e: $k<i<j$ \newline}
vectors to sum:
\begin{eqnarray*}
(\bzero_3~\cdots\bzero_3~\bzero_3~\bzero_3~\cdots~\bzero_3~\phi_1~\phi'_2~\cdots~\phi'_2~\phi'_2~\phi'_2\cdots)\cr
(\bzero_3~\cdots~\bzero_3~\bzero_3~\bzero_3~\cdots~\cdots~\cdots~\cdots~\bzero_3~\phi_2~\phi''_2\cdots)\cr
+(\bzero_3~\cdots~\bzero_3~\phi_3~\phi_2~\cdots~\phi_2~\phi_2~\phi_2~\cdots~\phi_2~\phi_2~\phi_2\cdots)\cr
\cline{1-2}
(\bzero_{3}~\cdots~\bzero_3~\bphi_3~\bphi_2\cdots~\bphi_2~\bphi'_1~\bphi''_2\cdots~\bphi''_2~\bphi'_2~\bzero_3\cdots)
\end{eqnarray*}


derived vector : $(\bzero_{3k}~\bphi_3~(\bphi_2)_{i-k-1}~\bphi'_1~(\bphi''_2)_{j-i-1}~\bphi'_2~\bzero_3~\cdots)$
\newline
Parity weight: \begin{equation}
\begin{split}
w_p&=2(i-k)+2+2(j-i)\\
&=2(j-k)+2
\end{split}
\end{equation}


\paragraph{Case 3f: $k<j<i$\newline}
\begin{eqnarray*}
(\bzero_3~\cdots~\bzero_3~\bzero_3~\bzero_3~\cdots~\cdots~\cdots~\cdots~\bzero_3~\phi_1~\phi'_2\cdots)\cr
(\bzero_3~\cdots\bzero_3~\bzero_3~\bzero_3~\cdots~\bzero_3~\phi_2~\phi''_2~\cdots~\phi''_2~\phi''_2~\phi''_2\cdots)\cr
+(\bzero_3~\cdots~\bzero_3~\phi_3~\phi_2~\cdots~\phi_2~\phi_2~\phi_2~\cdots~\phi_2~\phi_2~\phi_2\cdots)\cr
\cline{1-2}
(\bzero_{3}~\cdots~\bzero_3~\bphi_3~\bphi_2\cdots~\bphi_2~\bzero_3~\bphi'_2\cdots~\bphi'_2~\bphi_3~\bzero_3\cdots)
\end{eqnarray*}
derived vector : $(\bzero_{3k}~\bphi_3~(\bphi_2)_{j-k-1}~\bzero_3~(\bphi'_2)_{i-j-1}~\bphi_3~\bzero_3~\cdots)$\newline
Parity weight: \begin{equation}
\begin{split}
w_p &=2(j-k)+1 +2(i-j)+1 \\
&=2(i-k)+2
\end{split}
\end{equation}

%=================================


%\paragraph{extras}




%\paragraph{Case 1d: $i=j,j>k$ \newline}
%derived vector : $(\bzero_{3k}~\bh_3~(\bphi_3)_{i-k-1}~\bh_1~\bzero_3~\cdots)$\newline
%Parity weight: \begin{equation}
%\begin{split}
%w_p=2(i-k)+2
%\end{split}
%\end{equation}



%\paragraph{Case 2e: $j=k,k>i$ \newline}
%derived vector : $(\bzero_{3i}~\bh_1~(\bphi_1)_{j-i-1}~\bh''_1~\bzero_3~\cdots)$
%\newline
%Parity weight: \begin{equation}
%\begin{split}
%w_p=2(j-i)
%\end{split}
%\end{equation}
%\newpage 


%\paragraph{Case 3b: $k>i>j$\newline}
%derived vector : $(\bzero_{3j}~\bh_2~(\bphi_2)_{i-j-1}~\bh'_1~(\bphi_3)_{k-i-1}~\bh'_1~\bzero_3~\cdots)$
%\newline
%Parity weight: \begin{equation}
%\begin{split}
%w_p&=2(i-j)+1+2(k-i-1)+1\\
%&=2(k-j)
%\end{split}
%\end{equation}



%\paragraph{Case 3d: $k=i,i>j$\newline}
%derived vector : $(\bzero_{3j}~\bh_2~(\bphi_2)_{k-j-1}~\bphi_3~\bzero_3~\cdots)$
%\newline
%Parity weight: \begin{equation}
%\begin{split}
%w_p=2(k-j)+2
%\end{split}
%\end{equation}

From all the above cases we can conclude that the parity weight for a weight-$3$ RTZ sequence may be calculated as
\begin{equation}
w_p=
\begin{cases}
2l,& i<k \\
2l+2 & i \geq k\\
\end{cases}
\end{equation}
where $l=\max \{ i,j,k \} - \min \{ i,j,k \}$ is known as the layer distance.

We consider two weight-3 RTZ inputs $P(x)=1+x^2+x^4$ and $P'(x)=x^2+x^4+x^6$, which is a shifted version of $P(x)$. 
For $P(x),~k=0,~j=1,~i=0,~l=1$ and since $i=k$, we use the equation $w_p=2l+2=2(1)+2=4$
For $P(x),~k=2~j=1,~i=0,~l=2$ and since $i<k$, we use the equation $w_p=2l=2(2)=4$

This means that shifted versions of a weight-$3$ RTZ input have the same weight and without loss of generality, we may assume that all weight-$3$ begin at index 0 which means that we may ignore the case where $i<k$. We therefore have

\begin{equation}w_p=2l+2\end{equation}
Assuming that after interleaving, another weight-$3$ RTZ input is produced. Let $i',j',k',l'$ and $w'_p$ be similarly defined. Then the Hamming weight $w_H$of the turbo codeword produced can be calculated as
\begin{equation}
w_H=
7+2(l+l') 
\end{equation}

\end{proof}
\newpage





\end{document}
