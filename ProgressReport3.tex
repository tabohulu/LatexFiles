\documentclass[20 pts]{article}
\usepackage{xeCJK}
\usepackage{amsfonts}
\usepackage{amssymb}
\usepackage{amsmath}
\usepackage{bm}
\setCJKmainfont{SimSun}
\title{Trellis Termination} 
\author{Kwame Ackah Bohulu}
\date{2017/11/16}
\begin{document}
\maketitle

\newpage
\section{Turbo Codes using linear interleavers}
mapping function for linear interleaver

$$\Pi_L=Di \mod N,\,\,\, 0\leq i\leq N-1, \,\,\,\, gcd(N,D)=1$$

focus on weight-2 inputs. let the distance between the ``1'' bits be $t$. For a block
size $N$ range of $t$ is $$1 \leq t \leq N-1$$ . After interleaving the position of
the ``1'' bits are changed. Let the distance between the ``1'' bits of the interleaved 
sequence be $s$.

Using the linear interleaver, we assume that an input with the ``1'' bits seperated
by distance$t$ will always be interleaved to sequence with the ``1'' bits seperated by 
distance $s$. However this holds true only when the first bit position is $1$. When the
first bit position is greater than one a distance of $t$ is interleaved to a distance of $s$
or $-s \mod N$ depending on the position of the ``1'' bits.
\end{document}