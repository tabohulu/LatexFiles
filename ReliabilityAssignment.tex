\documentclass{beamer}  % presen 12pt
\mode<presentation>
%\usepackage{hyperref}
\usetheme{Madrid}
\usepackage{beamerthemesplit}
\usepackage[export]{adjustbox}
\usepackage{color}
\usepackage{tikz}
%

\input{../../Lab/stylefiles/rules/MyPackages}
% Font
\input{../../Lab/stylefiles/rules/MyBold}
\input{../../Lab/stylefiles/rules/MyCal}
\input{../../Lab/stylefiles/rules/MyScr}
\input{../../Lab/stylefiles/rules/MyBB}
\input{../../Lab/stylefiles/rules/MyTilde}
\input{../../Lab/stylefiles/rules/MyBar}
\input{../../Lab/stylefiles/rules/MyHat}
\input{../../Lab/stylefiles/rules/MyRoman}
% Environment
\input{../../Lab/stylefiles/rules/MyOperations}
\input{../../Lab/stylefiles/rules/MyNewEnvironment}
% References
\input{../../Lab/stylefiles/rules/MyJournals}
\bibliographystyle{IEEEtran}
\usepackage{graphicx}
\usepackage{lscape}
\usepackage{beamerthemesplit}
\setbeamertemplate{navigation symbols}{}
\setbeamertemplate{footline}[frame number]
\usetheme{Luebeck}
\renewcommand{\figurename}{図}
\renewcommand{\tablename}{表}
%\usepackage{xeCJK}
%\setCJKmainfont{MS Gothic}
%

\input{../../Lab/stylefiles/rules/MyPackages}
% Font
\input{../../Lab/stylefiles/rules/MyBold}
\input{../../Lab/stylefiles/rules/MyCal}
\input{../../Lab/stylefiles/rules/MyScr}
\input{../../Lab/stylefiles/rules/MyBB}
\input{../../Lab/stylefiles/rules/MyTilde}
\input{../../Lab/stylefiles/rules/MyBar}
\input{../../Lab/stylefiles/rules/MyHat}
\input{../../Lab/stylefiles/rules/MyRoman}
% Environment
\input{../../Lab/stylefiles/rules/MyOperations}
\input{../../Lab/stylefiles/rules/MyNewEnvironment}
% References
\input{../../Lab/stylefiles/rules/MyJournals}
\bibliographystyle{IEEEtran}
\usepackage{graphicx}
\usepackage{lscape}
\usepackage{beamerthemesplit}
\setbeamertemplate{navigation symbols}{}
\setbeamertemplate{footline}[frame number]
\usetheme{Luebeck}
\renewcommand{\figurename}{図}
\renewcommand{\tablename}{表}

\newcommand\blfootnote[1]{%
  \begingroup
  \renewcommand\thefootnote{}\footnote{#1}%
  \addtocounter{footnote}{-1}%
  \endgroup
}




\title[Assignment 1]{Assignment 1}
\author[Bohulu]{ \underline{Bohulu Kwame Ackah} \\ b1841008}
\institute[UEC]{Graduate School of Informatics and Engineering\\ The University of Electro-Communications}
\date[Week 3]{\today}

\begin{document}
\frame{\titlepage}

\addtobeamertemplate{navigation symbols}{}{%
    \usebeamerfont{footline}%
    \usebeamercolor[fg]{footline}%
    \hspace{1em}%
    \insertframenumber/\inserttotalframenumber
}




%Slide 1

\begin{frame}
\frametitle{Other Models}
\begin{itemize}
\item Shifted Hazard Model : Obtained by replacing $t$ in the hazard model with $(t-t_0$ to fit observed data, with $t_0$ being adjustable

\item Piecewise model estimation: By using the piecewise-linear model, an approximation  of the hazard function might be obtained

\begin{itemize}
\item Accuracy is improved by increasing the number of time segments.
\end{itemize}

\item Power series Estimation : Another estimate can be obtained using the power series models
\begin{itemize}
\item Let $\lambda(t)=K_{0}+K_{1} t+K_{2} t^{2}+\ldots+K_{n} t^{m}$. The reliability function is then

$R(t)=\exp \left[-\left(K_{0} t+K_{1} \frac{t^{2}}{2}+K_{2} \frac{t^{3}}{3}+\ldots+K_{m} \frac{t^{2+1}}{n+1}\right)\right]$
\end{itemize}
\item Most cases don't need complex models
\begin{itemize}
\item Real Challenge: Keeping the models of individual compenents as simple as possible to prevent analysis of the complete sytem from becoming too complicated\end{itemize}

\end{itemize}
\end{frame}

%slide 2
\begin{frame}
\frametitle{Modelling the Wearout Region}

\begin{itemize}
\item With reference to the bathtub hazard function, the wearout region corresponds to is the rapidly rising part towards the end of the useful life 
\begin{itemize}
\item Alternately modelled using the normal distribution around the mean wearout life $W$ of a component with suitably chosen paramenters
\end{itemize}

\item Using the normal distribution, the associated density function will be
$f(t)=\frac{1}{\sigma \sqrt{2 \pi}} \exp \left[-\frac{(t-W)^{2}}{2 \sigma^{2}}\right]$
where $t$ and $\sigma$ is the age of the component and standard deviation respectively

\item We consider a duration $t$ in the lifetime of the component which begins at time $T$. 

\item Assume failures occur due to random events during interval $(T,T+t)$. The components reliability is then 
$$ R_{t}(t)=e^{-\lambda t}$$ where $\lambda$ corresponds to the useful lifetime of the component.


\end{itemize}


\end{frame}


%slide 3
\begin{frame}
\frametitle{Modelling the Wearout Region-2}

\begin{itemize}
\item Assuming Failure is actually due to wearout during time interval $(T,T+t)$ then then the associated probability given survival up till $T$ is
$$Q_{w}(t)=\frac{\int_{T}^{r+1} f(\xi) d \xi}{\int_{r}^{\infty} f(\xi) d \xi}$$ 

\item The probability of no wearout failure then becomes
$$R_{w}(t)=1-Q_{w}(t)=\frac{\int_{r+1}^{\infty} f(\xi) d \xi}{\int_{r}^{\infty} f(\xi) d \xi}$$

\item Considering both chance failures and wearout failures, the probability that they do not occur during the same interval is given by
$$R(t)=R_{c}(t) R_{w}(t)=e^{-\lambda t} \frac{R_{\text {we }}(T+t)}{R_{\text {we }}(T)}$$ which simplifies to $e^{-\lambda t}R_{\text {we }}(t)$ at $T=0$
\end{itemize}


\end{frame}


%slide 4
\begin{frame} 
\frametitle{Modelling the Wearout Region-3}

\begin{itemize}
\item To achieve high values of reliability we need to ensure that

\begin{itemize}
\item The system is subjected to high burn and debugging procedures

\item Strict preventive maintenance and replacement schedules are to be adhered to to prevent components from entering wearout region.

\end{itemize}


\end{itemize}


\end{frame}


%slide 5
\begin{frame}
\frametitle{Reliability and Maintenance}

\begin{itemize}
\item Scheduled maintenance is done at constant intervals of time with the aim of prolonging the life of components etc.

\item Forced Maintenance is done only when there is an in-service failure and thus may be thought of as repair

\item Components that have a increasing hazard are the only ones that should be considered for scheduled maintenance

\item After incorporating maintenance, the density function may be written as 

$$f^{*}_T(t)=\sum_{k=0}^{\infty} f_{1}\left(t-k T_{M}\right) R^{k}\left(T_{M}\right)$$
where 
$\begin{aligned} f_{T}(t)=& \text { failure density function },R(t) =\text{component reliability function} \\ T_{M}=& \text { fixed time interval between maintenances } \\ & f_{1}(t)=\left\{\begin{array}{ll}f_{T}(t) & \text { for } 0<t \leq T_{M} \\ 0 & \text { otherwise }\end{array}\right.\end{aligned}$
\end{itemize}


\end{frame}

%slide 6
\begin{frame}
\frametitle{Ideal Repair}

\begin{itemize}
\item 2 conditions are assumed for ideal repair
\begin{itemize}
\item duration of repair after failure is much smaller when compared to time between failures and canbe assumed to be zero

\item Component restored to `as new' condition post repair
\end{itemize}
\item Time to $kth$ failure may be calculated as $$f_{k}(t)=\int_{0}^{1} f_{k-1}(\tau) f_{1}(t-\tau) d \tau ; \quad k \geq 2$$

\item let $L(t)$ be the density function of some failure occuring with ideal repair. Then
$$L(t)=f_{1}(t)+\sum_{k=2}^{\infty} \int_{0}^{t} f_{k-1}(\tau) f_{1}(t-\tau) d \tau$$

\end{itemize}


\end{frame}

%slide 7
\begin{frame}
\frametitle{Ideal Repair-2}

\begin{itemize}
\item if time between failures are exponentially distributed, then $f_k(t)$ becomes special case of Erlangian distribution

\item $f_k(t)$ becomes $$f_{k}(t)=\lambda^{k} \frac{t^{k-1}}{(k-1) !} e^{-\lambda t}$$

\item Finally for this special case $L(t)$ becomes $$\begin{aligned}
L(t)=\sum_{k=1}^{*} f_{k}(t) &=\left(\lambda e^{-\lambda t}\right) \sum_{k=1}^{\infty} \frac{(\lambda t)^{k-1}}{(k-1) !} \\
&=\lambda e^{-\lambda t} e^{\lambda t}=\lambda
\end{aligned}$$

\end{itemize}


\end{frame}

%slide 8
\begin{frame}
\frametitle{Ideal Repair and Preventative Maintenance}

\begin{itemize}
\item Combining Ideal repair and preventive maintenance if possible, will reduce the frequency of repairs

\item Assuming ideal maintenance at periodicc intervals of $T_M$ the frequence of repair $f_R$ will be

$$f_{R}=\frac{1}{T_{M}} \int_{0}^{T_{M}} L(t) d t$$

\item Maintenance is only useful if the MTTF which is the reciprocal of $f_{R}$ increases.

\end{itemize}


\end{frame}







\end{document}
