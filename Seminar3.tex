\documentclass[24 pts]{article}
\usepackage{xeCJK}
\usepackage{amssymb}
\usepackage{amsmath}
\usepackage{amsthm}
\setCJKmainfont[BoldFont= Yu Mincho Demibold]{MS Mincho}
\title{整数リング上において置換多項式を使用するターボ符号のためのインタリーバ}
\date{11/17/2016}
\author{Kwame Ackah Bohulu}
\begin{document}
\maketitle




\section{効果的な自由距離$(d_{ef})$を使用して、良いインタリーバを探索する。}
決定論インタリーバでは大きな$d_{ef}$が良い性能を保証するわけではないが、小さい$d_{ef}$だと通常、悪い性能になる関係がある。このような悪い置換多項式を選ばないように、$d_{ef}$を基準とする。
ランダムインタリーバと二次インタリーバの場合は、入力重み2エラーイベントが抑制できないが,いえるのは入力重み2エラーイベントが起きる確率は,フレーメサイズが無限にちかづくほど,ゼローになっていく。S-ランダムインタリーバの場合、それぞれの要素符号に起きるSより小さい距離を持つ入力重み2エラーイベントが防げる。$t\leq S$の場合、S-ランダムインタリーバは$(x,x+t)$を$(y,z)$にマッピングして、$|y-z|>S$。ところが、ある要素符号に起こる入力重み2エラーイベントはtが(cycle length)の倍数の値だけなので、S-ランダムインタリーバの能力がむだになる。
置換多項式に基づいてインタリーバを使う場合、多項式の係数をうまく選べば、ある要素符号によく起きる重み2エラーイベントが避けられる。そうすると、それより大きい入力重み2エラーイベントも避けられる。
1番目の要素符号に起きる入力重み2エラーイベントの長さを$t+1$とし、tは$\tau$の倍数で、tのオーダーは$o_t$とする。2番目の要素符号に起きる入力重み2エラーイベントの長さ-1は以下のようになる。

\begin{equation}\tag{8}
\Delta(x,t)=P(x+t)-P(x)=2btx+bt^2+at=c_1x+bt^2+at 
\end{equation}
\paragraph{}
性質2.9より、$x$の係数は$c_1=2bt$のオーダーは$o_{c1}=o_2+o_b+o_t$である。$x\in\{0,1,2,...,N-1\}$のとき、式(8)での第一項は$k\cdot p_N^{o_{c1}}とし,k=0,1,2,...,p_N^{o_N-o_{c1}}-1$\\それぞれの値は$p_N^{o_{c1}}$回をとる。
$x$に従って$c_1x$の図を描くと$p_N^{(o_N-o_{c1})}$の水平線が出る。$bt^2+at$は水平線のオフセットを与える。短い入力重み2エラーイベントを防止するために、$t$が$\tau$の小さい倍数の場合、$\Delta(x,t)$が$\tau$の倍数の値を0から離れてほしい。このためには、ベクトル$o_{c1}$を大きくして、$\Delta(x,t)$の図にある水平線の数が少なくなって、$\Delta(x,t)$を0から離れる係数をうまく選ばれる。
$o_{c1}$はもう大きいため、0の上か下からの最初線を着目する。着目される線から0までの距離は以下のように書ける。
\begin{equation}\tag{9}
s=\pm \Delta(x,t) \ mod \ p_N^{o_{c1}} =( bt^2+at) \ mod \ p_N^{o_{c1}}
\end{equation}
\paragraph{}
a,b,$\tau$が与えられたとき、$L_{(a,b,\tau)}$は以下のように定義して、良いインタリーバを選ぶ基準とする。
$$L_{(a,b,\tau)}=min⁡(|s|+|t|)$$
要素符号が与えられたとき、$L_(a,b,\tau)$から$d_{ef}$が計算できる。良いaとbを探索するとき、範囲を制限したらよい。以下の補題でaとbの範囲が制限できる。
%page 5%
\paragraph{}
補題4.1\\
入力重み2エラーイベントの解析では、$b$を$b_1$ $\cdot$  $b_0$=$b_1 \cdot p_N^{o_{b1}}$のようにかけばb1を1とすることができる。

\begin{proof}
$b_1=1$と仮定すると、$b_1$とNは互いに素である。ある置換多項式$P_1(x)=p_N^{o_b} x^2+at $が与えたら、(9)は$$s_1=p_N^{o_b}t^2 + at \ mod \ p_N^{o_b+o_t+o_2}$$
もう一つの置換多項式$P_2(x)=b_1p_N^{o_b} x^2+at$が与えたら、(9)は
$$s_2=b_1p_N^{o_b}t^2 + at\ mod \ p_N^{o_b+o_t+o_2}$$
$s_2-s_1$を計算すると以下の式が出る。
\begin{equation}\tag{10}
s_2-s_1=(b_1-1)p_N^{o_b}t^2 + at \ mod \ p_N^{o_b+o_t+o_2}
\end{equation}

\paragraph{2はNの因数の場合}:
$b_1$とNは互いに素であるので$b_1$は奇数で、$b_1-1$は偶数である。式(10)の右辺のオーダーは少なくとも$o_2+o_b+2o_t$。
$$s_2-s_1=0 \ mod \ p_N^{o_b+o_t+o_2}$$
\paragraph{2はNの因数でない場合}:
式(10)の右辺のオーダーは少なくとも$o_b+2o_t$であり、\ mod \ $o_b+o_t$で計算する。
$$s_2-s_1=0 \ mod \ p_N^{o_b+o_t+o_2}$$
$P_1(x)$と$P_2(x)$の入力重み2エラーイベントの位置以外は同じ入力重み2エラーイベントを持っている。この観点から、$P_1(x)$と$P_2(x)$は均しいである。
\end{proof}
\paragraph{}
補題4.2\\
入力重み2エラーイベントの解析では、$b= b_1 \cdot p_N^{o_{b1}}$があたえられたとき、$a$は$1\leq a \leq p_N^{o_{b1}}$となるaだけ考えば十分である。

\begin{proof}
補題4.1の結果より$b=p_N^{o_{b}}$。
\paragraph{2はNの因数でないとき}:
$a_0=a\ mod \ p_N^{o_{b}+o_{2}}$とする。すると、$a=a_0+lp_N^{o_{b}+o_{2}}$。
\begin{equation}\tag{11}
\begin{split}
&s=\pm ({b}t^2 + (a_0+lp_N^{o_{b}+o_{2}})t)\ mod \ p_N^{o_b+o_t+o_2}\\
&=\pm {b}t^2 + (a_0)t\ mod \ p_N^{o_b+o_t+o_2}
\end{split}
\end{equation}
これは$L_(a,b,\tau)=L_(a_0,b,\tau)$を意味する。
\paragraph{2はNの因数のとき}:
一般性を失わずに、上の証明で$1 \leq a < p_N^{o_b+o_2}$を仮定することができる。$a_0=p_N^{o_b+o_2}-a$とすると、
\begin{equation}\tag{12}
\begin{split}
&s=\pm ({b}t^2 + (a_0t)\ mod \ p_N^{o_b+o_2+o_t}\\
&s=\pm ({b}t^2 + (p_N^{o_b+o_t+o_2}-a)t)\ mod \ p_N^{o_b+o_2+o_t}\\
&=\pm (b(-t)^2 +(a(-t))\ mod \ p_N^{o_b+o_2+o_t}
\end{split}
\end{equation}
また、$L_(a,b,\tau)=L_(a_0,b,\tau)$
\end{proof}
7/5と5/7要素符号の場合の結果をテーブル\ref{テーブル:1}に書かれている。
\begin{table}[h!]
\begin{center}
\begin{tabular}{|c|c|c|c|c|c|c|c|c|}
\hline
a & 1 & 3 & 5 & 7 & 9 & 11 & 13 & 15 \\
\hline
L(5/7) & 12 & 18 & 12 & 24 & 24 & 18 & 12 & 6 \\
\hline
L(7/5) & 4 & 8 & 12 & 16 & 16 & 8 & 12 & 32 \\
\hline
\end{tabular}
\caption{$\tau(7/5) =2,\tau(5/7) =3,N=2^n,p_N=[\, 2]\,,o_N=[\, n]\,,o_b=[\, 4]\,b=16$}
\label{テーブル:1}
\end{center}
\end{table}
\paragraph{}
%page 6%



\section{結果}
フレームサイズNと要素符号にが与えられたら、良い置換多項式に基づいてインタリーバを探すことは、多項式のaとbを計算することになる。最初に、$o_b$の値を決める。前の分析で$p_N^{o_b}$を大きくしなければならないですが、特別な入力重み4エラーイベントと入力重み6エラーイベントで成約を拘束しなければならない。$o_b$が決めたら、$b=p_N^{o_b}$とし、定理4.8の範囲ですべてのaを計算する。
\paragraph{}
6種類の要素符号が選ばれて、テーブル2に書かれている。フレームサイズを$N=2^n$とし,Nのベースを$p_N=2$になり、Nのオーダーはスカラーになる。$N=2^8$の場合、要素符号に対して最良な置換多項式そして、入力重み2エラーイベントに対する最低距離と多重度がテーブル\ref{テーブル:2}に書かれている。

\begin{table}[h!]
\begin{center}
\begin{tabular}{|c|c|c|c|c|}
\hline
要素符号 & Cycle length($\tau$) & 最適多項式 & $d_{min}$(多重度) & 図 \\
\hline
7/5 & 2 & $15x+16x^2$ & 18(512) & 6 \\
\hline
5/7 & 3 & $15x+32x^2$ & 28(512) & 7 \\
\hline
37/21 & 4 & $7x+8x^2$ & 24(56) & 8 \\
\hline
21/37 & 5 & $15x+32x^2$ & 28(512) & 9 \\
\hline
37/25 & 6 & $15x+16x^2$ & 24(512) & 10 \\
\hline
23/35 & 7 & $15x+32x^2$ & 36(512) & 11 \\
\hline
\end{tabular}
\caption{様々な要素符号に対して最適な置換多項式、フレーメサイズ256}
\label{テーブル:2}
\end{center}
\end{table}

シムレーションで置換多項式に基づいてインタリーバをSーランダムインタリーバと二次インタリーバと比べた結果は、図6-11で示される。置換多項式に基づいたインタリーバは常に二次インタリーバとSーランダムインタリーバより良い性能をもつ。
\paragraph{}
要素符号をRC 5/7符号、フレームサイズNを1024と16384とし、それぞれのインタリーバの最良の置換多項式は$P(x)=31x+64x^2$と$P(x)=15x+32x^2$に基づく。シムレーションでの結果は図12と13に示される。長いフレームサイズの場合、置換多項式に基づいたインタリーバの性能は、二次インタリーバより良いですが、Sーランダムインタリーバほどよくないということがわかる。


\section{結論}
この論文には、置換多項式に基づいてインタリーバがしょうかいされた。インタリーバの生成多項式のパラメータが与えたら、多項式を計算することで,重要なエラーイベントの集合の$d_{ef}$が探索でき、本当の$d_{ef}$も近似できる。そして、近似値に対して、良いインタリーバの制限された探索ができる。紹介されましたインタリーバをS-ランダムインタリーバと二次インタリーバと比べられた。短いフレームサイズの場合,S-ランダムインタリーバより良い性能を持つインタリーバが見つけられた。長いフレームサイズの場合、紹介されたインタリーバはS-ランダムインタリーバと近い性能を持つ。二次インタリーバと比べた場合、どんなフレームサイズでも置換多項式に基づいてインタリーバの性能がたかいです。





\end{document}