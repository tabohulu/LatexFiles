\documentclass[24 pts]{article}
\usepackage{xeCJK}
\usepackage{amssymb}
\usepackage{amsmath}
\usepackage{amsthm}
\setCJKmainfont[BoldFont= Yu Mincho Demibold]{MS Mincho}
\title{整数リング上において置換多項式を使用するターボ符号のためのインタリーバ}
\date{}
\begin{document}
\maketitle

\section{置換多項式の探索(\textbf{たんさく})}
置換多項式に基づいてインタリーバの場合は、良い係数をうまく選べば$\Delta(x,t)=0$の線にかなり近い点がなくなる。
置換多項式に基づいてインタリーバはそれぞれの性能が違う。フレームサイズと要素符号が与えられたら、最適置換多項式を見つけようとするのである。固定(\textbf{こてい})フレームサイズが与えたら、残りの変数は多項式の次数と係数である。この論文では$$P(x)= bx^2+ax+c$$のような二次多項式に着目(\textbf{ちゃくもく})する。一つ目の理由は可能な限り低い複雑さを持ちたいためである。置換多項式の中で最も簡単な種類\textbf{しゅるい})は$$P(x)=ax+c$$(線形インタリーバ)の形を持つ一次多項式である。しかし、論文[6]で示されるように、線形インタリーバは悪い入力重み4エラーイベント性質を持つため、中間から長いフレームサイズの場合では高いエラーフローを起こす。そのため、二次多項式が注目\textbf{ちゃくもく})される。二番目の理由は二次多項式の分析が相対的(\textbf{そうたいてきに})に簡単だからである。
定数項\textbf{(ていすうこう)}$c$はインタリーブされたシーケンスの周期的回転(\textbf{しゅうきてきかいてん})にちょうど対応する。境界効果(\textbf{きょうかいこうか})を無視(\textbf{むし})したら、連結(\textbf{れんけつ})されたシステムの性能に影響を与えず、置換多項式の条件と関係ないので、$0$とし、
$$P(x)= bx^2+ax$$のような多項式を考える。

\paragraph{}
多項式の良い係数を選ぶために、ターボ符号のエラーイベントの部分集合の最低距離を基準(\textbf{きじゅん})とする。その部分集合は入力重み2mのエラーイベントである。そのエラーイベントは簡単に見つけて数えることができるからである。もちろん、そのエラーイベントはすべてのあり得る(\textbf{ありえる})エラーイベントの場合を示さないが、置換多項式に基づいたインタリーバの構造(\textbf{こうぞう})のため、特に短いフレーメサイズのとき、そのエラーイベントの多重度は高く、通常(\textbf{つうじょう})の場合、TCの性能に影響を与える。
この論文ではそれぞれのRSC符号の要素符号が(tail-biting trellis)[16]
%page 2%
を使っている仮定(\textbf{かてい})を使う。trellisの末尾(\textbf{まつび})はtrellisの先頭(\textbf{せんとう})と直接(\textbf{ちょくせつ})に繋いでいて(\textbf{つないでいて})、(flushing bits)を使わない。(tail-biting trellis)では要素符号におけるエラーイベントの周期性動き(\textbf{ちゅうきせいうごき})もエラーイベントである。そして、trellisの末尾の近くから始まるエラーイベントがtrellisの先頭にかかることが可能である。そのため、エラーイベントが mod Nといえる。その仮定(\textbf{かてい})を使うと、終端(\textbf{しゅうたん})にする境界効果(\textbf{きょうかいこうか})を無視(\textbf{むし})することができる。残念ながら、RSC符号を要素符号とするので、常に(\textbf{つねに})(tail-biting trellis)が存在(\textbf{そんざい})するわけではない[17]。終端(\textbf{しゅうたん})を使わなければならない。あるエラーイベントは終端でなくなる。そして、終端でエラーイベントを起こすときもある。ゆえに、エラーイベントをmod Nで探せば、エラーを導かける(\textbf{みちびかける})。しかし、終端(\textbf{しゅうたん})で壊された(\textbf{こわされた})mod Nエラーイベントの割合(\textbf{わりあい})が少ないし、終端にすって生じる(\textbf{しょうじる})の重みエラーイベントの多重度はたいてい低いので、非常に短くないフレームサイズのとき、mod N エラーイベントは性能を左右(\textbf{さゆう})する。ゆえに、エラーイベントをmod Nで数えてもかまわない。 

\subsection{入力重み2mエラーイベント}
長いランダムインタリーバは均一な(\textbf{きんいつな})インタリーバで近似(\textbf{きんじ})できる。均一なインタリーバというのは、与えられた入力を同じ確率で出力ポジションに並び替える確率的なデバイスのことである。[14]。均一なインタリーバモデルを使って、最高SNR領域(\textbf{りょういき})での復号性能は入力重み2エラーイベントに左右される。入力重み2エラーイベントに対して最小距離は(minimum effective distance)$d_{ef}$と呼び[2]、要素符号が良いエラーフロー性能をえるために、設計条件として、使われている。
置換多項式は線形インタリーバの一般化のように考えることができる。線形インタリーバと同じように、置換多項式によく起きるエラーイベントは入力重み$2m$エラーイベントである $(m=1,2,...)$
しかし、多項式の係数の$a$と$b$をうまく選べばそのようなエラーイベントを制御(\textbf{せいぎょう})することができる。要素符号が与えたら、良い性能をもつ二つの係数が見つけられる。
代表的な入力重み$2m$エラーイベントは図3に示される。\\
そのエラーイベントはそれぞれの要素符号にあるm個の入力重み2エラーイベントで作られていて、インタリーバで繋がっている。その図には1番目の要素符号に起きるi番目のエラーイベントは$x_i$から始まって、$t_i+1$の長さを持つ。2番目の要素符号に起きるi番目のエラーイベントは$s_i+1$の長さを持つ。\\
それぞれの入力重み$2$エラーイベントは最初と最後のポジションを示す整数の組で表される(\textbf{あらわされる})。二つの要素符号は同じRSC符号を使用するので、すべての$t_i$と$s_i$は畳み込み符号の(cycle length)$\tau$の倍数である。この論文では、(cycle length)$\tau$というのは入力シーケンスが[\,1,0,0,0......]\,のとき、符号器の出力の周期(\textbf{しゅうき})である。
\paragraph{例}
要素符号 = $\frac{1+D^2}{1+D+D^2}$,(\textbf{はちしんすう})$8$進数で$5/7$\\
入力=[1,0,0,0,...] 出力=[1,1,1,0,1,1,0,1,1,0]\\
%page 3%
周期=[1,1,0], (cycle length)$\tau=3$\\
 (cycle length)=最低入力重み2エラーイベントの距離-1。エラーパタンを以下の長さ2mのベクトルのように定義する。
$$[\,t_1, t_2,...., t_m, s_1,s_2,.., s_m]\,$$
入力重み2エラーイベントで、以下のm式が書ける。
\begin{equation}\tag{3.1}
P(x_2)-P(x_1)=s_1
\end{equation}

\begin{equation}\tag{3.2}
P(x_3)-P(x_1+t_1)=s_2
\end{equation}

\begin{equation}\tag{3.3}
P(x_4)-P(x_2+t_2)=s_3\\
\end{equation}

\begin{equation}\tag{3.m}
P(x_m+t_m)-P(x_{m-1}+t_{m-1})=s_m
\end{equation}
$t_i$,$s_i$の値は$\tau$の小さい倍数,$x_i=\ {0,1,…N-1\ }$
\paragraph{}
エラーイベントの見つける方法を簡単にするために、式3を分析しやすい形に変換する。すると以下のようになる。
\begin{equation}\tag{4}
\begin{split}
s_1-s_2+s_3-s_4...&=2b(x_1t_1-x_2t_2+x_3t_3-x_4t_4...)\\
&+b((t_1)^2-(t_2)^2+(t_3)^2-(t_4)^2...)\\
&+a(t_1-t_2+t_3-t_4...)
\end{split}
\end{equation}
または、もっと簡単な形で
\begin{equation}\tag{5}
\begin{split}
\sum_{i=1}^m (-1)^{i-1}s_i=2b\sum_{i=1}^m (-1)^{i-1}x_it_i\\
&+b\sum_{i=1}^m (-1)^{i-1} {t_i}^2+a\sum_{i=1}^m (-1)^{i-1}t_i
\end{split}
\end{equation}
\paragraph{}
図3のような入力重み$2m$エラーイベントが現れるために、式5は式3の残りのm-1式と一緒に使わなければなりません。この論文には、入力重み2mエラーイベントを見つけるために式3-5を使われる。
あるエラーパターンが与えられ、第一オーダーのエラーイベントが重ならなかったら、エラーイベントのハミング距離を一意に決定(\textbf{いちいにけってい})できる。例で説明する。\\
要素符号は5/7のRSC符号とする。その符号の$\tau=3$。$t_i$と$s_i$は$\tau$の倍数なのでk$\tau$の一般形を持つ。入力は、$1+D^{k\tau}$とする。出力は以下のようになる。
\begin{equation}\tag{6}
\begin{split}
(1+D^{3k})\frac{1+D^2}{1+D+D^2}\\
&=(1+D^3+D^(2.3)+...+D^{3(k-1)})\\
&\times(1+D^{3})\frac{1+D^2}{1+D+D^2}\\
&=(1+D^3+D^(2.3)+...+D^{3(k-1)})\\
&\times(1+D+D^2+D^3)
\end{split}
\end{equation}
%page 4%
\paragraph{}
入力の$1+D^3$にたいして出力シーケンスの重みは$w_0=2$になる。(最初と最後の1を入れずに)そのうえ、入力の$1+D^{3k}$にたいして出力シーケンスの重みは$2+w_0k$になる。エラーエベントの全出力重みは以下のようになる。

\begin{equation}\tag{7}
6m+ \left( \frac{\sum \left|t_i\right|}{\tau}+\frac{\sum \left|s_i\right|}{\tau}\right)w_0
\end{equation}
この論文では式(7)を使ってエラーパターンのハミング距離を計算する。

\subsection{効果的な自由距離(\textbf{こうかてきなじゆうきょり})$(d_{ef})$を使用して、良いインタリーバを探索(\textbf{たんさく})する。}
決定論インタリーバでは大きな$d_{ef}$が良い性能を保証(\textbf{ほしょう})するわけではないが、小さい$d_{ef}$だと通常、(\textbf{つうじょう})悪い性能になる関係がある。このような悪い置換多項式を選ばないように、$d_{ef}$を基準(\textbf{きじゅん})とする。
ランダムインタリーバと二次インタリーバの場合は、入力重み2エラーイベントが抑制できない(\textbf{よくせい})がいえるのは入力重み2エラーイベントが起きる確率はフレーメサイズが無限にちかづくほどゼローになっていく。S-ランダムインタリーバの場合、それぞれの要素符号に起きるSより小さい距離を持つ入力重み2エラーイベントが防げる(\textbf{ふせげる})。$t\leq S$の場合、S-ランダムインタリーバは$(x,x+t)$を$(y,z)$にマッピングして、$|y-z|>S$。ところが、ある要素符号に起こる入力重み2エラーイベントはtが(cycle length)の倍数の値だけなので、S-ランダムインタリーバの能力がむだになる。
置換多項式に基づいてインタリーバを使う場合、多項式の係数をうまく選べば、ある要素符号によく起きる重み2エラーイベントが避けられる(\textbf{よけられる})。そうすると、それより大きい入力重み2エラーイベントもわけられる。
1番目の要素符号に起きる入力重み2エラーイベントの長さを$t+1$とし、tは$\tau$の倍数で、tのオーダーは$o_t$とする。2番目の要素符号に起きる入力重み2エラーイベントの長さ-1は以下のようになる。

\begin{equation}\tag{8}
\Delta(x,t)=P(x+t)-P(x)=2btx+bt^2+at=c_1x+bt^2+at 
\end{equation}
\paragraph{}
性質2.9より、$x$の係数は$c_1=2bt$のオーダーは$o_{c1}=o_2+o_b+o_t$である。$x\in\{0,1,2,...,N-1\}$のとき、式(8)での第一項(\textbf{だいいっこう})は$k\cdot p_N^{o_{c1}}とし,k=0,1,2,...,p_N^{o_N-o_{c1}}-1$\\それぞれの値は$p_N^{o_{c1}}$回をとる。
$x$に従って$c_1x$の図を描くと$p_N^{(o_N-o_{c1})}$の水平線(\textbf{すいへいせん})が出る。$bt^2+at$は水平線のオフセットを与える。短い入力重み2エラーイベントを防止するために、$t$が$\tau$の小さい倍数の場合、$\tau$の倍数の値$\Delta(x,t)$もを0から離れてほしい。このためには、ベクトル$o_{c1}$を大きくして、$\Delta(x,t)$の図にある水平線の数が少なくなって、$\Delta(x,t)$を0から離れる係数をうまく選ばれる。
$o_{c1}$はもう大きいため、0の上か下からの最初線を着目する。着目される線から0までの距離は以下のように書ける。
\begin{equation}\tag{9}
s=\pm \Delta(x,t) \ mod \ p_N^{o_{c1}} =( bt^2+at) \ mod \ p_N^{o_{c1}}
\end{equation}
\paragraph{}
a,b,$\tau$が与えられたとき、$L_{(a,b,\tau)}$は以下のように定義して、良いインタリーバを選ぶ基準とする。
$$L_{(a,b,\tau)}=min⁡(|s|+|t|)$$
要素符号が与えられたとき、$L_(a,b,\tau)$から$d_{ef}$が計算できる。良いaとbを探索するとき、範囲(\textbf{はんい})を制限(\textbf{せいげん})したらよい。以下の補題(\textbf{ほだい})でaとbの範囲が制限できる。
%page 5%
\paragraph{}
補題4.1\\
入力重み2エラーイベントの解析(\textbf{かいせき})では、$b$を$b_1$ $\cdot$  $b_0$=$b_1 \cdot p_N^{o_{b1}}$のようにかけばb1を1とすることができる。

\begin{proof}
$b_1=1$と仮定(\textbf{かてい})すると、$b_1$とNは互いに素(\textbf{たがいにそ})である。ある置換多項式$P_1(x)=p_N^{o_b} x^2+at) $が与えたら、(9)は$$s_1=p_N^{o_b}t^2 + at \ mod \ p_N^{o_b+o_t+o_2}$$
もう一つの置換多項式$P_2(x)=b_1p_N^{o_b} x^2+at)$が与えたら、(9)は
$$s_2=b_1p_N^{o_b}t^2 + at\ mod \ p_N^{o_b+o_t+o_2}$$
$s_2-s_1$を計算すると以下の式が出る。
\begin{equation}\tag{10}
s_2-s_1=(b_1-1)p_N^{o_b}t^2 + at \ mod \ p_N^{o_b+o_t+o_2}
\end{equation}

\paragraph{2はNの因数(\textbf{いんすう})の場合}$b_1$とNは互いに素であるので$b_1$は奇数(\textbf{きすう})で、$b_1-1$は偶数(\textbf{ぐうすう})である。式(10)の右辺(\textbf{うへん})のオーダーは少なくとも$o_2+o_b+2o_t$。
\paragraph{2はNの因数でない場合}式(10)の右辺のオーダーは少なくとも$o_b+2o_t$であり、\ mod \ $o_b+o_t$で計算する。両方の場合に
$$s_2-s_1=0 \ mod \ p_N^{o_b+o_t+o_2}$$
$P_1(x)$と$P_2(x)$の入力重み2エラーイベントの位置以外は同じ入力重み2エラーイベントを持っている。この観点(\textbf{かんてん})から、$P_1(x)$と$P_2(x)$は均しい(\textbf{ひとしい})である。
\end{proof}
\paragraph{}
補題4.2\\
入力重み2エラーイベントの解析(\textbf{かいせき})では、$b= b_1 \cdot p_N^{o_{b1}}$があたえられたとき、$a$は$1\leq a \leq p_N^{o_{b1}}$となるaだけ考えば十分である。

\begin{proof}
補題4.1の結果より$b=p_N^{o_{b}}$。\paragraph{2はNの因数でないとき}$a_0=a\ mod \ p_N^{o_{b}+o_{2}}$とする。すると、$a=a_0+lp_N^{o_{b}+o_{2}}$。
\begin{equation}\tag{11}
\begin{split}
&s=\pm ({b}t^2 + (a_0+lp_N^{o_{b}+o_{2}})t)\ mod \ p_N^{o_b+o_t+o_2}\\
&=\pm {b}t^2 + (a_0)t\ mod \ p_N^{o_b+o_t+o_2}
\end{split}
\end{equation}
これは$\mathbf{L}_(a,b,\tau)=\mathbf{L}_(a_0,b,\tau)$を意味する。
2はNの因数(\textbf{いんすう})でないとき、上記(\textbf{じょうき})の証明は十分である。もし2はNの因数ならば、一般性を失わずに(\textbf{うしなわずに})、上の証明で$1 \leq a < p_N^{o_b+o_2}$を仮定することができる。$a_0=p_N^{o_b+o_2}-a$とすると、
\begin{equation}\tag{12}
\begin{split}
&s=\pm ({b}t^2 + (a_0t)\ mod \ p_N^{o_b+o_2+o_t}\\
&s=\pm ({b}t^2 + (p_N^{o_b+o_t+o_2}-a)t)\ mod \ p_N^{o_b+o_2+o_t}\\
&=\pm (b(-t)^2 +(a(-t))\ mod \ p_N^{o_b+o_2+o_t}
\end{split}
\end{equation}
また、$\mathbf{L}_(a,b,\tau)=\mathbf{L}_(a_0,b,\tau)$
\end{proof}
7/5と5/7要素符号の場合の結果をテーブル\ref{テーブル:1}に示す。
\begin{table}[h!]
\begin{center}
\begin{tabular}{|c|c|c|c|c|c|c|c|c|}
\hline
a & 1 & 3 & 5 & 7 & 9 & 11 & 13 & 15 \\
\hline
L(5/7) & 12 & 18 & 12 & 24 & 24 & 18 & 12 & 6 \\
\hline
L(7/5) & 4 & 8 & 12 & 16 & 16 & 8 & 12 & 32 \\
\hline
\end{tabular}
\caption{$\tau(7/5) =2,\tau(5/7) =3,N=2^n,p_N=[\, 2]\,,o_N=[\, n]\,,o_b=[\, 4]\,b=16$}
\label{テーブル:1}
\end{center}
\end{table}
\paragraph{}
%page 6%
$o_b$が与えられたとき、$d_{ef}$で良いaとbを選ぶのは十分可能であるように思える(\textbf{おもえる})。しかし、$d_{ef}$のみでは$o_b$を選ぶ十分な情報でない。例えば式(9)を見ると、$o_b$が大きければもっと良いaを選ぶことができる。しかし、シミュレーションより、置換多項式の性能は$o_b$に従ってある値まで良くなって、その値を超えると性能が悪くなる。それを説明して、これより正確パラメータを選ぶ方法を見つけるために、より高い入力重みエラーイベントを調べなければならない。

\subsection{より高い入力重みエラーイベント}
良い置換多項式を探すとき、入力重み2エラーイベントの$d_{ef}$で決める。$d_{ef}$を見つけるために、$m$が小さい値しか注目しません。大きい値はほとんど大きいハミング距離と関係があるからである。エラーイベントを見つける一つの方法はエラーパタン$[\,t_1,...,t_m,s_1,....,s_m]\,$をきめて$x_1$を計算する。\\エラーパターンと$x_1$が決めたら、残りの$x_i$は、式(3)の残りの$ m-1$式で計算できる。最後に$x_i,t_i,s_i$の$3m$値をまだ使っていない(3)か(4)式に使って、エラーイベントが正しいかどうかを確かめる。
\paragraph{}
置換多項式に基づいてインタリーバは高度に構造化(\textbf{こうどにこうぞうか})ので$x_1$を0から$p_N^{(o_N-o_2-o_b)}-1$から確認したら十分である。入力重み2mエラーイベントは周期的な構造を持つ。一番目の要素符号に起こるエラーイベントはすべての2m末尾点を$p_N^{(o_N-o_2-o_b)}$の倍数で循環的に動かしたら、ぜんたいのTCで、正しい入力重み2mエラーイベントをまた得る。ゆえに、エラーイベントの探索で$x_1$を0から$p_N^{(o_N-o_2-o_b)}-1$から確認したら十分である。
$p_N^{(o_N-o_2-o_b)}$が小さい場合、この方法は有能(\textbf{ゆうのう})である。エラーパターンが与えられ、$x_1$を見つけるとき、入力重み2mエラーイベントの制約(\textbf{せいやく})をーつの式に変えたい。つまり、式(3)での$m-1$残りの式を式(5)で $i=2,…m$をキャンセルする。
以下の問題を解決できれば、(3)と(5)を一つの式にすることができる。\\
問題:N,a,b,sが与え、$ P(y)-P(x)=s$なら、xをyの関数とします。\\上記の問題を解決する前、ほかの定義が必要。置換多項式$ P(x)= bx^2+ax$と定数したsが与えたら、シーケンス$\{y_i\}$は以下のように定義します。

\begin{align*}
P(y_0)-P(0)&=s\\
P(y_1)-P(1)&=s\\
P(y_2)-P(2)&=s\\
P(y_3)-P(3)&=s\\
\tag{16}
\end{align*}
そして、$\Delta_{k }(i)$を再帰的に(\textbf{さいきてきに})以下のように定義する。
\begin{align*}
\Delta_1(i)&=y_i-1\\
\Delta_2 (i)&= \Delta_0 (i+1)- \Delta_0 (i)etc
\end{align*}
注意:$y_i$ と$\Delta_k$ (i)はa,bとsに関する関数である。\\すると以下の定理が出ます。
\paragraph{}
定理4.3\\
N,$P(x)$とsが与えられたとき、すべての iで$\Delta_k (i)$は同じオーダを持つ。$\Delta_k (i)$のオーダーは $o_{\Delta_k}$と示したら、 $k$>0場合$o_{\Delta_0}= o_s$。そして、$o_{\Delta_k}=o_{\Delta_{k-1}}+o_b+o_{2k}$。すべてのkでは$o_{\Delta_k}=ko_b+ko_2+\sum_{n=1}^k o_n  +o_s$である。

\paragraph{}
$\Delta_k$のオーダーは$k$に従って厳密に増加(\textbf{げんみつにぞうか})ので、最終的(\textbf{さいしゅうてき})に$o_N$よりおおきくなる。もし$K$は最大数で $o_\Delta k\ngeq o_N$であれば、 $\Delta_{K+1}(i)=0\ mod\ N$。 $\Delta_{K+1}(i)$のていぎより$\Delta_K(i)$はすべての$i$で定数である。この結果は以下の系で要約された。
系4.4\\
N,P(x)とsが与え、Kが一番大きい数で、$o_(\Delta_k)\ngeq o_N$の場合、すべてのiで $\Delta_K (i)= \Delta_K$は定数である。
注意:$K,N,a,b$とsの関数である。
\paragraph{}
系4.4でのKを見つけられたら、すべての k>K場合 $\Delta_K (i)=0$。その上、もしすべての $0\leq k\leq$ Kで,$\Delta_K (0) $が知られていたら、 すべてのiで$\Delta_K (i)$の定義で $\Delta_K (i)$を計算すくことがでる。
わかりやすくするために、$\Delta_k$ を$\Delta_k (0)$と代表する。sを指す必要があったら、$\Delta_k(s)$を使う。\\これで、$P(y)-P(x)=s$ の場合、xとyの関係を見つける道具を持っている。以下の定理で要約された(\textbf{ようやくされた})。
\paragraph{}
定理4.5
N,P(x)とsが与え、$P(y)-P(x)=s$であるならば
\begin{equation}\tag{17}
y=x+\Delta_0 (s) +\Delta_1 (s)+\frac{(x(x-1)}{2!} \Delta_2 (s)+\frac{(x(x-1)(x-2)}{3!} \Delta_3(s)+⋯		
\end{equation}	

x<kの場合${x \choose k}=0$と定義すると、(17)が以下のように書ける。
\begin{equation}\tag{18}
y=F(x,s)\triangleq x+\sum_{k=0}^\infty {x \choose k} \Delta_k (s)	
\end{equation}					
F(x,s)を整数値の多項式をするために、Nが与えたら、$D(k)=\prod_{i=2}^k \frac{i}{p_N^{o_i}}$ を定義する。
$\frac{\Delta_k(s)}{k!}$はいつも整数であるわけではないからである。

\subsubsection{式を解くことで、エラーイベントを探索(\textbf{たんさく})する。}
式(4)が正解(\textbf{せいかい})であったら、エラーイベントを作ることがでる。定理4.5を式(3)の残り$m-1$で使えば、以下の式が出る。
\begin{align*}
x_2&=F(x_1, s_1 )\\
x_3&=F(x_1 +t_1,s_2 )\\
x_4&=F(x_2 +t_2,s_3 )\\
x_m&=F(x_{m-2} +t_{m-2},s_{m-1} )	\tag{20}
\end{align*}
そして、(20)を(4)で使えば、一般的な$x_1$に対する多項式になる。 一般の多項式では、解答、または、いくつの解答があるかを探すことの複雑さは高いである。しかし,$m$の値が小さい場合もっと簡単な方法がある。これから、$m=1,2,3$の場合を解く。
\newpage
\subsubsection{$m=1$、入力重み2エラーイベント}
上記の場合は、式(20)を使わずに、式(4)は以下のようになる。
\begin{equation}\tag{21}
2bt_1 x_1+b{t_1}^2+at_1-s_1=0	
\end{equation}				
$x_1$に対する線形多項式としたら、以下のようになります。
\begin{equation}\tag{22}
c_1 x+c_0
\end{equation}
						
もし$o_{c0}  \geq o_{c1}$場合のみ(22)の解答がある。その条件が満たされる場合、$p_N^{o_{c1}}$で(22)を分割(\textbf{ぶんかつ})でき、結果は特別な解答を持つ一次多項式\ mod \ $p_N^{o_N-o_{c1}}$になる。多くの場合ではエラーイベントの位置(\textbf{いちい})より、エラーイベントのハミング距離とそのハミング距離の多重度に興味がある。ゆえに、式(22)を解答するかわりに$o_{c0} \geq o_{c1}$を確認する。満たされていたら、$p_N^{o_{c1}}$の多重度を対応するスペクトラム線と足す。

\subsubsection{$m=2$、入力重み4エラーイベント}
(20)使って、(4)は以下のようになる。

\begin{equation}\tag{23}
2b[\,x_1t_1-(x_1\sum_{k=0}^\infty {x_1 \choose k}\Delta_k(s_1))t_2]\, b(t_1^2-t_1^2)+a(t_1-t_2)-(s_1-s_2)=0
\end{equation}
$x_1$に関する条項を収集すると以下のようになる。
\begin{equation}\tag{24}
\begin{split}
(-2b\Delta_0 (s_1 ) t_2&+b(t_1^2-t_2^2)+a(t_1-t_2)-(s_1-s_2))\\
&+〖2b(t_1-t_2-\Delta_1 (s_1 ) t_2)x_1\\
&-2bt_2 \sum_{k=2}^\infty \frac{\Delta_k (s_1 )}{k!} \prod_{m=0}^{k-1}(x_1 -m)=0
\end{split}   
\end{equation}
式(24)は分数係数を持つような$x_1$に対する多項式である。$s_1$が与えられ、系4.4でのKを見つけることができる。(24) をD(K)と掛けたら(\textbf{かけたら})、以下のようになります。
\begin{equation}\tag{25}
c_0+c_1 x_1+c_2 (x_1 )=0
\end{equation}

\begin{align*}
c_0&=(-2b\Delta_0 (s_1 ) t_2+b(t_1^2-t_2^2)+a(t_1-t_2)-(s_1-s_2))D(K)\\
c_1&=2b(t_1-t_2-\Delta_1 (s_1 ) t_2)D(K)\\
c_2 (x_1 )&=-2bt_2 \sum_{k=2}^\infty \frac{\Delta_k (s_1 )}{k!} \prod_{m=0}^{k-1}(x_1 -m)=0.\tag{26}
\end{align*}
\paragraph{}
$c_0$ と$c_1$は整数であり、$c_2 (x_1 )$は$x_1$に対する多項式である。
式(26) のオーダーは少なくとも、$3o_b+3o_2+o_s1+o_{t2}$。系2.5で、$\frac{c_1 x_1+c_2 (x_1 )}{p_N^{o_{c1} }}$は置換多項式である。$t_1\neq t_2$場合、系2.5を使うために、以下の条件がある。
\begin{equation} \tag{27}
o_{c1}\ll3o_b+3o_2+o_{s1}+o_{t2}\end{equation}							
エラーパターン$[\,t_1 , t_2  , s_1 ,  s_2]\,$が与えられたとき、式(27)の条件があっているかどうかを確認する。そして、入力重み2エラーイベントと同じように$o_{c0}  \geq o_{c1}$の確認しかしない。正しければ、$p_N^{o_{c1}}$同じエラーパタンを持っているエラーイベントがある。もしスペクトラムに着目されたら、$\Delta_0 (s_1 )$と$\Delta_1 (s_1 )$を解くことになる。

\subsubsection{$m=3$、入力重み6エラーイベント}
この場合はの(25)は入力重み4エラーイベントと同じ形になる。
$$c_1=2b(t_1-t_2+t_3-\Delta_1(s_1)t_2+\Delta_1(s_2)t_3)D$$
$D=D(max(K(s_1),K(s_2)))$
そして、$c_2(x_1)$の係数のオーダーは少なくとも$min(3o_b+3o_2+o_{s_1}+o_{t_1},3o_b+3o_2+o_{s_2}+o_{t_3})$。系2.5を使用する条件は
\begin{equation}\tag{28}
o_{c1}=min(3o_b+3o_2+o_{s_1}+o_{t_1},3o_b+3o_2+o_{s_2}+o_{t_3})
\end{equation}
あっていたら、$\frac{c_1 x_1+c_2 (x_1 )}{p_N^{o_{c1}} }$は置換多項式である。もしスペクトラムに着目したら、$\Delta_0 (s_1 ), \Delta_0 (s_2 ), \Delta_1 (s_1 )$と$\Delta_1 (s_2 )$を解くことになる。

\subsection{$o_b$の上界}
入力重み2エラーイベントの分析より、$N$が与えたとき、$o_b$が大きければ良いインタリーバを選ぶことができ、よい性能を得る。しかし、シミュレーションより、置換多項式の性能は$o_b$に従ってある値まで良くなって、その値を超えると性能が悪くなる。
定理2.5で$o_b$の上界を見つけることができる。
\subsubsection{入力重み4エラーイベントでの$o_b$の上界}
(25)から始まる。条件(27)があっていて、$o_{c0}  \geq o_{c1}$のとき、(25)で$x_1$  の解答があるなら、与えたエラーパターンにたいして$x_1$から始まるエラーイベントがある。0からN-1のすべて$x_1$が解答である特別な場合がある。$t_1= t_2=t,s_1=s_2=s$ のとき、(26)にある$c_0$と$c_1$は以下のようになる。
\begin{equation*}
\begin{split}
c_0&=-2b\Delta_0(s)tD(K)\\
c_1&=-2b\Delta_1(s)tD(K)
\end{split}
\end{equation*}
$o_{c0}  \geq o_{c1}$ということが簡単に見える。$o_{c0}=o_N$のとき、(25)は全0の多項式になる。

\subsubsection{入力重み6エラーイベントでの$o_b$の上界}
(25)から始まる。条件(28)があっていたら、$c_0$と$c_1$は以下のようになる。
\begin{equation}\tag{30}
\begin{split}
c_0=&[\, 2b[\,t_3\Delta_0(s_2)-t_2\Delta_0(s_1)]\,+2bt_3t_1\Delta_1(s_2)\\
&+2bt_1t_3+b(t_1^2-t_2^2-t_3^2)\\
&+a(t_1-t_2+t_3)-(s_1-s_2+s_3) ]\,D
\end{split}
\end{equation}

\begin{equation}\tag{31}
c_1=[\, 2b(t_1-t_2+t_3)+2b(t_3\Delta_(s_2)-t_2\Delta(s_1))]\,
\end{equation}
$D=D(max⁡(k(s_1) ,k(s_2)))$で$c_2 (x_1)$の係数のオーダーは$o_c1$より大きいすべての$x_1$がエラーパタンの解答である場合に興味ある。\newline
[\,2t,t,-t,s,-s,2s]\,の形を持つエラーパタンが大変重要である。最小ハミング距離に対応するエラーパタンは$t=\tau$ と
$s=\pm t$のときである。上記のエラーパタンで$t_1-t_2+t_3=s_1-s_2+s_3=0$それで、$c_0$と$c_1$は以下のようになる。

\begin{equation}\tag{32}
c_0=\left[\,-2bt[\,\Delta_0(-s)+\Delta_0(s)]\,-4bt^2\Delta_1(-s)\right]\,D
\end{equation}

\begin{equation}\tag{33}
c_1=-2bt(\Delta_1(-s)+\Delta_1(s))D
\end{equation}
ここから進めるために二つの補題が必要である。
補題4.6
$\Delta_0 (-s)+\Delta_0 (s)$のオーダーは$o_b +o_2+2o_b$
補題4.7
$\Delta_1 (-s)+\Delta_1 (s)$のオーダーは少なくとも$o_b +2o_2+2o_b$

それで$c_0$と$c_1$のオーダーを計算することができる。

$$o_{c0}\geq2o_b+2o_2+3o_\tau$$
$$o_{c1}\geq2o_b+3o_2+3o_\tau$$

ベクトル$o_N$にあるそれぞれのメンバー$2o_b+2o_2+3o_\tau$に対応するメンバーより大きくないとき$c_0$と$c_1$は$0modN$になって、すべての$x_1$は式の解答である。$o_b$の上界とすることができる。

\subsection{aとbを探索するときの範囲}
補題4.1と補題4.2で、入力重み2エラーイベントのとき、$o_b$が決めたら$b=b_0 \cdot p_N^{o_b}$としてaは1から$b_0$しか注目しない。残念ながら一般的な場合でもaの範囲の結果はだいたい同じである。
\paragraph{}
定理4.8
二次順列多項式に基づいたインタリーバで、aの範囲は1から2bを注目しなければない。\\
入力重み4エラーイベントでは一般的に、$o_b$が与えたら、すべてのb、そして$1\leq a\leq 2b$さがさなければならない。これは退屈(\textbf{たいくつ})である。しかし、ある条件で補題4.1よりのbを使うことができる。ゆえに、入力重み2エラーイベントのスペクトラムを使って、多項式を探索するとき、$b=p_N^{o_b}$と$1\leq a\leq 2b$を着目する。
\section{結果}
フレームサイズNと要素符号にが与えられたら、良い置換多項式に基づいてインタリーバを探すことは、多項式のaとbを計算することになる。最初に、$o_b$の値を決める。前の分析で$p_N^{o_b}$を大きくしなければならないですが、特別入力重み4エラーイベントと入力重み6エラーイベントで成約を拘束しなければならない。$o_b$が決めたら、$b=p_N^{o_b}$として定理4.8の範囲ですべてのaを計算する。
\paragraph{}
6種類の要素符号が選ばれて、テーブル2に書かれている。フレームサイズを$N=2^n$とし,Nのベースを$p_N=2$になり、Nのオーダーはスカラーになる。$N=2^8$の場合、要素符号に対して最良な置換多項式そして、入力重み2エラーイベントに対する最低距離と多重度がテーブル\ref{テーブル:2}に書かれている。

\begin{table}[h!]
\begin{center}
\begin{tabular}{|c|c|c|c|c|}
\hline
要素符号 & Cycle length($\tau$) & 最適多項式 & $d_{min}$(多重度) & 図 \\
\hline
7/5 & 2 & $15x+16x^2$ & 18(512) & 6 \\
\hline
5/7 & 3 & $15x+32x^2$ & 28(512) & 7 \\
\hline
37/21 & 4 & $7x+8x^2$ & 24(56) & 8 \\
\hline
21/37 & 5 & $15x+32x^2$ & 28(512) & 9 \\
\hline
37/25 & 6 & $15x+16x^2$ & 24(512) & 10 \\
\hline
23/35 & 7 & $15x+32x^2$ & 36(512) & 11 \\
\hline
\end{tabular}
\caption{様々な要素符号に対して最適な置換多項式、フレーメサイズ256}
\label{テーブル:2}
\end{center}
\end{table}

シムレーションで置換多項式に基づいてインタリーバをSーランダムインタリーバと二次インタリーバと比べた結果は、図6-11で示される。置換多項式に基づいたインタリーバは常に二次インタリーバとSーランダムインタリーバより良い性能をもつ。
\paragraph{}
要素符号をRC5/7符号、フレームサイズNを1024と16384とし、それぞれのインタリーバの最良置換多項式は$P(x)=31x+64x^2$と$P(x)=15x+32x^2$に基づく。シムレーションでの結果は図12と13に示される。長いフレームサイズの場合、置換多項式に基づいたインタリーバの性能は、二次インタリーバより良いですが、Sーランダムインタリーバほどよくないということがわかる。

\section{結論}
この論文には、置換多項式に基づいてインタリーバがしょうかいされた。インタリーバの生成多項式のパラメータが与えたら、多項式を計算することで,大切なエラーイベントの集合の$d_{ef}$が探索でき、本当の$d_{ef}$も近似できる。そして、近似値に対して、良いインタリーバの制限された探索ができる。紹介されましたインタリーバをS-ランダムインタリーバと二次インタリーバと比べられた。短いフレームサイズの場合,S-ランダムインタリーバより良い性能を持つインタリーバが見つけられた。長いフレームサイズの場合、紹介されたインタリーバはS-ランダムインタリーバと近い性能を持つ。二次インタリーバと比べた場合、どんなフレームサイズでも置換多項式に基づいてインタリーバの性能がたかいです。


\end{document}