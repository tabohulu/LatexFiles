\documentclass[20 pts]{article}
\usepackage{xeCJK}
\usepackage{amsfonts}
\usepackage{amssymb}
\usepackage{amsmath}
\usepackage{bm}
\setCJKmainfont{SimSun}
\title{ターボ符号器における決定論的インタリーバの設計}
\author{Kwame Ackah Bohulu, 1631133}
\date{\today}
\begin{document}
\maketitle


\section{概要}
 
 
 
ターボ符号は、加法性ガウス雑音通信路において、通信路容量を達成できる誤り訂正符号
の1つとして知られているが、その性能は用いるインターリーバに大きく左右される。
決定論的インタリーバはアルゴリズムを用いてインタリーブを行うため,ランダムインターリーバに比べて、
メモリ消費が少なく、ハードウェア構成が簡単の上、並列復号などの高速復号が可能のため、
実際のターボ符号で広く用いられている.ところが、現在知られている決定論的インターリーバは、
フレームサイズが長い場合、ランダムインターリーバより性能が劣る。

本研究では、決定論的インタリーバの中で、フレームサイズが短い場合、
ランダムインターリーバより性能が優れ、設計が簡単な線形インターリーバの改良を行う。
本稿ではまず、フレームサイズが長い場合、線形インターリーバは高SNR(Signal to Noise Ratio)で
ビット誤り率 (BER: Bit Error Rate) 特性の劣化を引き起こすエラーフロアの発生原因について
解析する.そして,線形インタリーバの係数を決められた定数内で周期的に変化させ,
この係数に従ってすべての位置をシフトさせ,エラーフロアの発生原因である重み$2$のエラーイベントの
発生を和らげる新しいインターリーバ:multi-Shift interleaver(MSI)を提案する。
また、性能の良いMSIを効率よく探索するために,定数制約探索法を用いてMSIの最適
パラメーターを探索する.中程度と長いフレームサイズにおいて,提案するインタリーバが
線形インタリーバーより優れたBER を達成できることをシミュレーションにより確認した.
\end{document}