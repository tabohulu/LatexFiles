\documentclass[24 pts]{article}
\usepackage{xeCJK}
\usepackage{amssymb}
\usepackage{amsmath}
\usepackage{amsthm}
\usepackage{relsize}
\newcommand{\me}{\mathrm{e}}
\setCJKmainfont[BoldFont= Yu Mincho Demibold]{MS Mincho}
\title{情報伝送基礎レポート6}
\date{}
\author{クワメ・アカー・ボフル 1631133}
\begin{document}
\maketitle
\newpage
\textbf{定理1:}
ある信号$x(t)$が 
\[
x(nT)=
\begin{cases}
1&,n=0\\
0,&n\neq0
 \end{cases}
\]
を満たされる重要十分条件は、その信号のFourier変換
 $X(f)$が \begin{equation}\mathlarger{\mathlarger{\sum}}_{m=-\infty}^\infty X(f+m/T)=T\end{equation}を満たされることである。
\begin{proof}
一般的に、$x(t)$は$X(f)$の逆Fourier変換である。
$$x(t)=\int_{-\infty}^{\infty} X(f)\me^{j2\pi ft} df$$
$t=nT$のとき、
\begin{equation}x(nT)=\int_{-\infty}^{\infty} X(f)\me^{j2\pi fnT} df\end{equation}
上記の式を1/Tに対して有限範囲に書き換えるといかのようになる。
\begin{equation}
\begin{split}
x(nT)=&\sum_{-\infty}^{\infty} \int_{(2m-1)/2T}^{(2m+1)/2T}  X(f)\me^{j2\pi fnT} df\\
= &\int_{-1/2T}^{1/2T} \left [\, \sum_{-\infty}^{\infty} X(f+m/T)\right]\, \me^{j2\pi fnT} df, B(f)= \sum_{-\infty}^{\infty} X(f+m/T)\\
 =&\int_{-1/2T}^{1/2T} B(f)\me^{j2\pi fnT} df
\end{split}
\end{equation}
$B(f)$は$1/T$に対して周期関数なので、フーリエ級数係数$\{ b_n\}$に従って展開と
\begin{equation} B(f) =\sum_{-\infty}^{\infty} b_n\me^{j2\pi nfT}
, b_n =T\int_{-1/2T}^{1/2T} B(f)  \me^{-j2\pi nfT}df \end{equation}
式(4)で定義した$b_n$と式(3)を比べると
$$ b_n=Tx(-nT)$$
ゆえに、しき(1)が満たされる必要十分条件は
\[
x(nT)=
\begin{cases}
T&,n=0\\
0,&n\neq0
 \end{cases}
\]
式(4)で定義した$B(f)$に入れると
$$ B(f)= \sum_{-\infty}^{\infty} X(f+m/T)=T$$
\end{proof}
\end{document}