\documentclass[11pt, oneside, dvipdfmx]{book}
\newcommand{\folder}{/usr/local/share/texmf}
%Page setting
\setlength{\textwidth}{460pt}
\setlength{\topmargin}{-1truemm}
\setlength{\textheight}{650pt}
\setlength{\oddsidemargin}{-0.2truemm}
\setlength{\evensidemargin}{-0.2truemm}

%%%%%%%%%%%%
% 	 UsePackege	%
%%%%%%%%%%%%
\usepackage{amssymb}
\usepackage{amsmath}
\usepackage{amsthm}
\usepackage{ascmac}
\usepackage{lscape}
\usepackage{pifont}
\usepackage{cite}
\usepackage{ifthen}
\usepackage{framed}
\usepackage{mathrsfs}
\usepackage{booktabs}
\usepackage{algorithm}
\usepackage{algorithmic}
\usepackage{comment}
\usepackage{longtable}
% Font
%%%%%%%%%%%%%%
% 	Bold Number	%
%%%%%%%%%%%%%%
\newcommand{\bzero}{{\bf 0}}
\newcommand{\bone}{{\bf 1}}
%%%%%%%%%%%%%%
% 	Bold English	%
%%%%%%%%%%%%%%
% a
\newcommand{\ba}{{\mbox{\boldmath$a$}}}
\newcommand{\bA}{{\mbox{\boldmath$A$}}}
\newcommand{\sba}{{\mbox{\scriptsize\boldmath $a$}}}
% b
\newcommand{\bb}{{\mbox{\boldmath$b$}}}
\newcommand{\bB}{{\mbox{\boldmath$B$}}}
\newcommand{\sbb}{{\mbox{\scriptsize\boldmath $b$}}}
% c
\newcommand{\bc}{{\mbox{\boldmath$c$}}}
\newcommand{\bC}{{\mbox{\boldmath$C$}}}
\newcommand{\sbC}{{\mbox{\scriptsize\boldmath $C$}}}
\newcommand{\sbc}{{\mbox{\scriptsize\boldmath $c$}}}
% d
\newcommand{\bd}{{\mbox{\boldmath$d$}}}
\newcommand{\bD}{{\mbox{\boldmath$D$}}}
\newcommand{\sbd}{{\mbox{\scriptsize\boldmath $d$}}}
% e
\newcommand{\be}{{\mbox{\boldmath$e$}}}
\newcommand{\bE}{{\mbox{\boldmath$E$}}}
\newcommand{\sbe}{{\mbox{\scriptsize\boldmath $e$}}}
% f
\newcommand{\bbf}{{\mbox{\boldmath$f$}}}
\newcommand{\bF}{{\mbox{\boldmath$F$}}}
% g
\newcommand{\bg}{{\mbox{\boldmath$g$}}}
\newcommand{\bG}{{\mbox{\boldmath$G$}}}
% h
\newcommand{\bh}{{\mbox{\boldmath$h$}}}
\newcommand{\bH}{{\mbox{\boldmath$H$}}}
\newcommand{\sbH}{{\mbox{\scriptsize\boldmath$H$}}}
\newcommand{\sbh}{{\mbox{\scriptsize\boldmath$h$}}}
% i
\newcommand{\bi}{{\mbox{\boldmath$i$}}}
\newcommand{\sbi}{{\mbox{\scriptsize \boldmath$i$}}}
\newcommand{\bI}{{\mbox{\boldmath$I$}}}
% i
\newcommand{\bj}{{\mbox{\boldmath$j$}}}
\newcommand{\sbj}{{\mbox{\scriptsize \boldmath$j$}}}
\newcommand{\bJ}{{\mbox{\boldmath$J$}}}
% k
\newcommand{\bk}{{\mbox{\boldmath$k$}}}
\newcommand{\bK}{{\mbox{\boldmath$K$}}}
% l
%\newcommand{\ell}{{\mbox{\boldmath $l$}}}
\newcommand{\bl}{{\mbox{\boldmath$l$}}}
\newcommand{\bL}{{\mbox{\boldmath$L$}}}
% m
\newcommand{\bm}{{\mbox{\boldmath$m$}}}
\newcommand{\sbm}{{\mbox{\scriptsize \boldmath$m$}}}
\newcommand{\bM}{{\mbox{\boldmath$M$}}}
% n
\newcommand{\bn}{{\mbox{\boldmath$n$}}}
\newcommand{\bN}{{\mbox{\boldmath$N$}}}
\newcommand{\sbn}{{\mbox{\scriptsize \boldmath$n$}}}
\newcommand{\sbN}{{\mbox{\scriptsize\boldmath $N$}}}
% o
\newcommand{\bo}{{\mbox{\boldmath$o$}}}
\newcommand{\bO}{{\mbox{\boldmath$O$}}}
% p
\newcommand{\bp}{{\mbox{\boldmath$p$}}}
\newcommand{\bP}{{\mbox{\boldmath$P$}}}
\newcommand{\sbp}{{\mbox{\scriptsize \boldmath$p$}}}
% q
\newcommand{\bq}{{\mbox{\boldmath$q$}}}
\newcommand{\bQ}{{\mbox{\boldmath$Q$}}}
% r
\newcommand{\br}{{\mbox{\boldmath$r$}}}
\newcommand{\bR}{{\mbox{\boldmath$R$}}}
\newcommand{\sbr}{{\mbox{\scriptsize\boldmath$r$}}}
% s
\newcommand{\bs}{{\mbox{\boldmath$s$}}}
\newcommand{\sbs}{{\mbox{\scriptsize \boldmath$s$}}}
\newcommand{\bS}{{\mbox{\boldmath$S$}}}
% t
\newcommand{\bt}{{\mbox{\boldmath$t$}}}
\newcommand{\bT}{{\mbox{\boldmath$T$}}}
% u
\newcommand{\bu}{{\mbox{\boldmath$u$}}}
\newcommand{\bU}{{\mbox{\boldmath$U$}}}
\newcommand{\sbu}{{\mbox{\scriptsize\boldmath $u$}}}
% v
\newcommand{\bv}{{\mbox{\boldmath$v$}}}
\newcommand{\bV}{{\mbox{\boldmath$V$}}}
% w
\newcommand{\bw}{{\mbox{\boldmath$w$}}}
\newcommand{\bW}{{\mbox{\boldmath$W$}}}
\newcommand{\sbW}{{\mbox{\scriptsize\boldmath $W$}}}
\newcommand{\sbw}{{\mbox{\scriptsize\boldmath $w$}}}
% x
\newcommand{\bx}{{\mbox{\boldmath$x$}}}
\newcommand{\bX}{{\mbox{\boldmath$X$}}}
\newcommand{\sbx}{{\mbox{\scriptsize\boldmath $x$}}}
\newcommand{\sbX}{{\mbox{\scriptsize\boldmath $X$}}}
%y
\newcommand{\by}{{\mbox{\boldmath$y$}}}
\newcommand{\bY}{{\mbox{\boldmath$Y$}}}
\newcommand{\sby}{{\mbox{\scriptsize\boldmath $y$}}}
\newcommand{\sbY}{{\mbox{\scriptsize\boldmath $Y$}}}
%z
\newcommand{\bz}{{\mbox{\boldmath$z$}}}
\newcommand{\bZ}{{\mbox{\boldmath$Z$}}}
\newcommand{\sbz}{{\mbox{\scriptsize\boldmath$z$}}}
\newcommand{\sbZ}{{\mbox{\scriptsize\boldmath$Z$}}}


%%%%%%%%%%%
% 	Bold Grik	%
%%%%%%%%%%%
% alpha
\newcommand{\balpha}{{\mbox{\boldmath$\alpha$}}}
\newcommand{\sbalpha}{{\mbox{\scriptsize\boldmath$\alpha$}}}
% beta
\newcommand{\bbeta}{{\mbox{\boldmath$\beta$}}}
\newcommand{\sbbeta}{{\mbox{\scriptsize\boldmath$\beta$}}}
% gamma
\newcommand{\bgamma}{{\mbox{\boldmath$\gamma$}}}
\newcommand{\bGamma}{{\mbox{\boldmath$\Gamma$}}}
\newcommand{\sbgamma}{{\mbox{\scriptsize\boldmath$\gamma$}}}
% delta
\newcommand{\bdelta}{{\mbox{\boldmath$\delta$}}}
\newcommand{\bDelta}{{\mbox{\boldmath$\Delta$}}}
\newcommand{\sbdelta}{{\mbox{\scriptsize\boldmath$\delta$}}}
% epsilon
\newcommand{\bepsilon}{{\mbox{\boldmath$\epsilon$}}}
\newcommand{\bvarepsilon}{{\mbox{\boldmath$\varepsilon$}}}
\newcommand{\sbvarepsilon}{{\mbox{\scriptsize\boldmath$\varepsilon$}}}
% zeta
\newcommand{\bzeta}{{\mbox{\boldmath$\zeta$}}}
% eta
\newcommand{\etab}{{\mbox{\boldmath$\eta$}}}
% theta
\newcommand{\btheta}{{\mbox{\boldmath$\theta$}}}
\newcommand{\bTheta}{{\mbox{\boldmath$\Theta$}}}
\newcommand{\bvartheta}{{\mbox{\boldmath$\vartheta$}}}
% iota
\newcommand{\biota}{{\mbox{\boldmath$\iota$}}}
\newcommand{\sbiota}{{\mbox{\scriptsize\boldmath$\iota$}}}
% kappa
\newcommand{\bkappa}{{\mbox{\boldmath$\kappa$}}}
% lambda
\newcommand{\blambda}{{\mbox{\boldmath$\lambda$}}}
\newcommand{\bLambda}{{\mbox{\boldmath$\Lambda$}}}
% mu
\newcommand{\bmu}{{\mbox{\boldmath$\mu$}}}
\newcommand{\sbmu}{{\mbox{\scriptsize\boldmath$\mu$}}}
% nu
\newcommand{\bnu}{{\mbox{\boldmath$\nu$}}}
\newcommand{\sbnu}{{\mbox{\scriptsize\boldmath$\nu$}}}
% xi
\newcommand{\bxi}{{\mbox{\boldmath$\xi$}}}
\newcommand{\bXi}{{\mbox{\boldmath$\Xi$}}}
% pi
\newcommand{\bpi}{{\mbox{\boldmath$\pi$}}}
\newcommand{\bPi}{{\mbox{\boldmath$\Pi$}}}
\newcommand{\bvarpi}{{\mbox{\boldmath$\varpi$}}}
% rho
\newcommand{\brho}{{\mbox{\boldmath$\rho$}}}
\newcommand{\bvarrho}{{\mbox{\boldmath$\varrho$}}}
% sigma
\newcommand{\bsigma}{{\mbox{\boldmath$\sigma$}}}
\newcommand{\bSigma}{{\mbox{\boldmath$\Sigma$}}}
\newcommand{\bvarsigma}{{\mbox{\boldmath$\varsigma$}}}
% tau
\newcommand{\btau}{{\mbox{\boldmath$\tau$}}}
% upsilon
\newcommand{\bupsilon}{{\mbox{\boldmath$\upsilon$}}}
\newcommand{\bUpsilon}{{\mbox{\boldmath$\Upsilon$}}}
% phi
\newcommand{\bphi}{{\mbox{\boldmath$\phi$}}}
\newcommand{\bPhi}{\mathbf \Phi}
\newcommand{\bvarphi}{{\mbox{\boldmath$\varphi$}}}
% chi
\newcommand{\bchi}{{\mbox{\boldmath$\chi$}}}
% psi
\newcommand{\bpsi}{{\mbox{\boldmath$\psi$}}}
\newcommand{\bPsi}{\mathbf \Psi}
% omega
\newcommand{\bomega}{{\mbox{\boldmath$\omega$}}}
\newcommand{\bOmega}{{\mbox{\boldmath$\Omega$}}}
% nabla
\newcommand{\bnabla}{{\mbox{\boldmath$\nabla$}}}

% calligraphic letters
\newcommand{\cA}{{\cal A}}
\newcommand{\cB}{{\cal B}}
\newcommand{\cC}{{\cal C}}
\newcommand{\cD}{{\cal D}}
\newcommand{\cE}{{\cal E}}
\newcommand{\cF}{{\cal F}}
\newcommand{\cG}{{\cal G}}
\newcommand{\cH}{{\cal H}}
\newcommand{\cI}{{\cal I}}
\newcommand{\cJ}{{\cal J}}
\newcommand{\cK}{{\cal K}}
\newcommand{\cL}{{\cal L}}
\newcommand{\cM}{{\cal M}}
\newcommand{\cN}{{\cal N}}
\newcommand{\cO}{{\cal O}}
\newcommand{\cP}{{\cal P}}
\newcommand{\cQ}{{\cal Q}}
\newcommand{\cR}{{\cal R}}
\newcommand{\cS}{{\cal S}}
\newcommand{\cT}{{\cal T}}
\newcommand{\cU}{{\cal U}}
\newcommand{\cV}{{\cal V}}
\newcommand{\cW}{{\cal W}}
\newcommand{\cX}{{\cal X}}
\newcommand{\cY}{{\cal Y}}
\newcommand{\cZ}{{\cal Z}}
% calligraphic letters
\newcommand{\mA}{{\mathscr A}}
\newcommand{\mB}{{\mathscr B}}
\newcommand{\mC}{{\mathscr C}}
\newcommand{\mD}{{\mathscr D}}
\newcommand{\mE}{{\mathscr E}}
\newcommand{\mF}{{\mathscr F}}
\newcommand{\mG}{{\mathscr G}}
\newcommand{\mH}{{\mathscr H}}
\newcommand{\mI}{{\mathscr I}}
\newcommand{\mJ}{{\mathscr J}}
\newcommand{\mK}{{\mathscr K}}
\newcommand{\mL}{{\mathscr L}}
\newcommand{\mM}{{\mathscr M}}
\newcommand{\mN}{{\mathscr N}}
\newcommand{\mO}{{\mathscr O}}
\newcommand{\mP}{{\mathscr P}}
\newcommand{\mQ}{{\mathscr Q}}
\newcommand{\mR}{{\mathscr R}}
\newcommand{\mS}{{\mathscr S}}
\newcommand{\mT}{{\mathscr T}}
\newcommand{\mU}{{\mathscr U}}
\newcommand{\mV}{{\mathscr V}}
\newcommand{\mW}{{\mathscr W}}
\newcommand{\mX}{{\mathscr X}}
\newcommand{\mY}{{\mathscr Y}}
\newcommand{\mZ}{{\mathscr Z}}
% calligraphic letters
\newcommand{\bbA}{{\mathbb A}}
\newcommand{\bbB}{{\mathbb B}}
\newcommand{\bbC}{\mathbb{C}}
\newcommand{\bbD}{\mathbb{D}}
\newcommand{\bbE}{{\mathbb E}}
\newcommand{\bbF}{{\mathbb F}}
\newcommand{\bbG}{{\mathbb G}}
\newcommand{\bbH}{{\mathbb H}}
\newcommand{\bbI}{{\mathbb I}}
\newcommand{\bbJ}{{\mathbb J}}
\newcommand{\bbK}{{\mathbb K}}
\newcommand{\bbL}{\mathbb{L}}
\newcommand{\bbM}{{\mathbb M}}
\newcommand{\bbN}{\mathbb{N}}
\newcommand{\bbO}{{\mathbb O}}
\newcommand{\bbP}{\mathbb{P}}
\newcommand{\bbQ}{{\mathbb Q}}
\newcommand{\bbR}{\mathbb{R}}
\newcommand{\bbS}{\mathbb{S}}
\newcommand{\bbT}{{\mathbb T}}
\newcommand{\bbU}{{\mathbb U}}
\newcommand{\bbV}{\mathbb{V}}
\newcommand{\bbW}{{\mathbb W}}
\newcommand{\bbX}{\mathbb{X}}
\newcommand{\bbY}{\mathbb{Y}}
\newcommand{\bbZ}{{\mathbb Z}}
%%%%%%%%%%%%%%
% 	Tilde English	%
%%%%%%%%%%%%%%
% a
\newcommand{\tila}{\tilde{a}}
\newcommand{\tilA}{\tilde{A}}
% b
\newcommand{\tilb}{\tilde{b}}
\newcommand{\tilB}{\tilde{B}}
% c
\newcommand{\tilc}{\tilde{c}}
\newcommand{\tilC}{\tilde{C}}
% d
\newcommand{\tild}{\tilde{d}}
\newcommand{\tilD}{\tilde{D}}
% e
\newcommand{\tile}{\tilde{e}}
\newcommand{\tilE}{\tilde{E}}
% f
\newcommand{\tilf}{\tilde{f}}
\newcommand{\tilF}{\tilde{F}}
% g
\newcommand{\tilg}{\tilde{g}}
\newcommand{\tilG}{\tilde{G}}
% h
\newcommand{\tilh}{\tilde{h}}
\newcommand{\tilbh}{\tilde\bh}
\newcommand{\tilH}{\tilde{H}}
% i
\newcommand{\tili}{\tilde{i}}
\newcommand{\tilI}{\tilde{I}}
% i
\newcommand{\tilj}{\tilde{j}}
\newcommand{\tilJ}{\tilde{J}}
% k
\newcommand{\tilk}{\tilde{k}}
\newcommand{\tilK}{\tilde{K}}
% l
\newcommand{\till}{\tilde{l}}
\newcommand{\tilL}{\tilde{L}}
% m
\newcommand{\tilm}{\tilde{m}}
\newcommand{\tilM}{\tilde{M}}
% n
\newcommand{\tiln}{\tilde{n}}
\newcommand{\tilN}{\tilde{N}}
% o
\newcommand{\tilo}{\tilde{o}}
\newcommand{\tilO}{\tilde{O}}
% p
\newcommand{\tilp}{\tilde{p}}
\newcommand{\tilP}{\tilde{P}}
% q
\newcommand{\tilq}{\tilde{q}}
\newcommand{\tilQ}{\tilde{Q}}
\newcommand{\tilbq}{\tilde{\bq}}
% r
\newcommand{\tilr}{\tilde{r}}
\newcommand{\tilR}{\tilde{R}}
\newcommand{\tilcR}{\tilde{\cR}}
\newcommand{\tilbr}{\tilde{\br}}
% s
\newcommand{\tils}{\tilde{s}}
\newcommand{\tilS}{\tilde{S}}
% t
\newcommand{\tilt}{\tilde{t}}
\newcommand{\tilT}{\tilde{T}}
% u
\newcommand{\tilu}{\tilde{u}}
\newcommand{\tilU}{\tilde{U}}
% v
\newcommand{\tilv}{\tilde{v}}
\newcommand{\tilV}{\tilde{V}}
% w
\newcommand{\tilw}{\tilde{w}}
\newcommand{\tilW}{\tilde{W}}
% x
\newcommand{\tilx}{\tilde{x}}
\newcommand{\tilbx}{\tilde{\bx}}
\newcommand{\tilX}{\tilde{X}}
\newcommand{\tilbX}{\tilde{\bX}}
%y
\newcommand{\tily}{\tilde{y}}
\newcommand{\tilY}{\tilde{Y}}
\newcommand{\tilby}{\tilde{\by}}
\newcommand{\tilbY}{\tilde{\bY}}
%z
\newcommand{\tilz}{\tilde{z}}
\newcommand{\tilZ}{\tilde{Z}}
\newcommand{\tilbZ}{\tilde{\bZ}}


%%%%%%%%%%%%
%% 	Bold Grik	%
%%%%%%%%%%%%
%% alpha
\newcommand{\tilalpha}{\tilde{\alpha}}
%\newcommand{\sbalpha}{{\mbox{\scriptsize\boldmath$\alpha$}}}
%% beta
\newcommand{\tilbeta}{{\tilde{\beta}}}
%\newcommand{\sbbeta}{{\mbox{\scriptsize\boldmath$\beta$}}}
%% gamma
\newcommand{\tilgamma}{{\tilde{\gamma}}}
%\newcommand{\bGamma}{{\mbox{\boldmath$\Gamma$}}}
%\newcommand{\sbgamma}{{\mbox{\scriptsize\boldmath$\gamma$}}}
%% delta
%\newcommand{\bdelta}{{\mbox{\boldmath$\delta$}}}
%\newcommand{\bDelta}{{\mbox{\boldmath$\Delta$}}}
%\newcommand{\sbdelta}{{\mbox{\scriptsize\boldmath$\delta$}}}
%% epsilon
%\newcommand{\bepsilon}{{\mbox{\boldmath$\epsilon$}}}
%\newcommand{\bvarepsilon}{{\mbox{\boldmath$\varepsilon$}}}
%\newcommand{\sbvarepsilon}{{\mbox{\scriptsize\boldmath$\varepsilon$}}}
%% zeta
%\newcommand{\bzeta}{{\mbox{\boldmath$\zeta$}}}
%% eta
%\newcommand{\etab}{{\mbox{\boldmath$\eta$}}}
%% theta
%\newcommand{\btheta}{{\mbox{\boldmath$\theta$}}}
%\newcommand{\bTheta}{{\mbox{\boldmath$\Theta$}}}
%\newcommand{\bvartheta}{{\mbox{\boldmath$\vartheta$}}}
%% iota
%\newcommand{\biota}{{\mbox{\boldmath$\iota$}}}
%\newcommand{\sbiota}{{\mbox{\scriptsize\boldmath$\iota$}}}
%% kappa
%\newcommand{\bkappa}{{\mbox{\boldmath$\kappa$}}}
%% lambda
%\newcommand{\blambda}{{\mbox{\boldmath$\lambda$}}}
%\newcommand{\bLambda}{{\mbox{\boldmath$\Lambda$}}}
%% mu
%\newcommand{\bmu}{{\mbox{\boldmath$\mu$}}}
%\newcommand{\sbmu}{{\mbox{\scriptsize\boldmath$\mu$}}}
%% nu
%\newcommand{\bnu}{{\mbox{\boldmath$\nu$}}}
%\newcommand{\sbnu}{{\mbox{\scriptsize\boldmath$\nu$}}}
%% xi
%\newcommand{\bxi}{{\mbox{\boldmath$\xi$}}}
%\newcommand{\bXi}{{\mbox{\boldmath$\Xi$}}}
%% pi
%\newcommand{\bpi}{{\mbox{\boldmath$\pi$}}}
%\newcommand{\bPi}{{\mbox{\boldmath$\Pi$}}}
%\newcommand{\bvarpi}{{\mbox{\boldmath$\varpi$}}}
%% rho
%\newcommand{\brho}{{\mbox{\boldmath$\rho$}}}
%\newcommand{\bvarrho}{{\mbox{\boldmath$\varrho$}}}
%% sigma
%\newcommand{\bsigma}{{\mbox{\boldmath$\sigma$}}}
%\newcommand{\bSigma}{{\mbox{\boldmath$\Sigma$}}}
%\newcommand{\bvarsigma}{{\mbox{\boldmath$\varsigma$}}}
% tau
\newcommand{\tiltau}{{\mbox{$\tilde{\tau}$}}}
%% upsilon
%\newcommand{\bupsilon}{{\mbox{\boldmath$\upsilon$}}}
%\newcommand{\bUpsilon}{{\mbox{\boldmath$\Upsilon$}}}
%% phi
%\newcommand{\bphi}{{\mbox{\boldmath$\phi$}}}
%\newcommand{\bPhi}{\mathbf \Phi}
%\newcommand{\bvarphi}{{\mbox{\boldmath$\varphi$}}}
%% chi
%\newcommand{\bchi}{{\mbox{\boldmath$\chi$}}}
%% psi
%\newcommand{\bpsi}{{\mbox{\boldmath$\psi$}}}
%\newcommand{\bPsi}{\mathbf \Psi}
%% omega
%\newcommand{\bomega}{{\mbox{\boldmath$\omega$}}}
%\newcommand{\bOmega}{{\mbox{\boldmath$\Omega$}}}

%%%%%%%%%%%%%%
% 	Tilde English	%
%%%%%%%%%%%%%%
% a
\newcommand{\bara}{\bar{a}}
\newcommand{\barA}{\bar{A}}
% b
\newcommand{\barb}{\bar{b}}
\newcommand{\barB}{\bar{B}}
% c
\newcommand{\barc}{\bar{c}}
\newcommand{\barC}{\bar{C}}
% d
\newcommand{\bard}{\bar{d}}
\newcommand{\barD}{\bar{D}}
% e
\newcommand{\bare}{\bar{e}}
\newcommand{\barE}{\bar{E}}
% f
\newcommand{\barbf}{\bar{f}}
\newcommand{\barF}{\bar{F}}
% g
\newcommand{\barg}{\bar{g}}
\newcommand{\barG}{\bar{G}}
% h
\newcommand{\barh}{\bar{h}}
\newcommand{\barbh}{\bar\bh}
\newcommand{\barH}{\bar{H}}
% i
\newcommand{\bari}{\bar{i}}
\newcommand{\barI}{\bar{I}}
% i
\newcommand{\barj}{\bar{j}}
\newcommand{\barJ}{\bar{J}}
% k
\newcommand{\bark}{\bar{k}}
\newcommand{\barK}{\bar{K}}
% l
\newcommand{\barl}{\bar{l}}
\newcommand{\barL}{\bar{L}}
% m
\newcommand{\barm}{\bar{m}}
\newcommand{\barM}{\bar{M}}
% n
\newcommand{\barn}{\bar{n}}
\newcommand{\barN}{\bar{N}}
% o
\newcommand{\baro}{\bar{o}}
\newcommand{\barO}{\bar{O}}
% p
\newcommand{\barp}{\bar{p}}
\newcommand{\barP}{\bar{P}}
% q
\newcommand{\barq}{\bar{q}}
\newcommand{\barQ}{\bar{Q}}
% r
\newcommand{\barr}{\bar{r}}
\newcommand{\barR}{\bar{R}}
% s
\newcommand{\bars}{\bar{s}}
\newcommand{\barS}{\bar{S}}
% t
\newcommand{\bart}{\bar{t}}
\newcommand{\barT}{\bar{T}}
% u
\newcommand{\baru}{\bar{u}}
\newcommand{\barU}{\bar{U}}
% v
\newcommand{\barv}{\bar{v}}
\newcommand{\barV}{\bar{V}}
% w
\newcommand{\barw}{\bar{w}}
\newcommand{\barW}{\bar{W}}
\newcommand{\barbw}{\bar{\bw}}
% x
\newcommand{\barx}{\bar{x}}
\newcommand{\barX}{\bar{X}}
%y
\newcommand{\bary}{\bar{y}}
\newcommand{\barY}{\bar{Y}}
%z
\newcommand{\barz}{\bar{z}}
\newcommand{\barZ}{\bar{Z}}
%%%%%%%%%%%%%%
% 	Hat English	%
%%%%%%%%%%%%%%
% a
\newcommand{\hata}{\hat{a}}
\newcommand{\hatA}{\hat{A}}
% b
\newcommand{\hatb}{\hat{b}}
\newcommand{\hatB}{\hat{B}}
% c
\newcommand{\hatc}{\hat{c}}
\newcommand{\hatC}{\hat{C}}
\newcommand{\hatbc}{\hat{\bc}}
% d
\newcommand{\hatd}{\hat{d}}
\newcommand{\hatD}{\hat{D}}
% e
\newcommand{\hate}{\hat{e}}
\newcommand{\hatE}{\hat{E}}
% f
\newcommand{\hatbf}{\hat{f}}
\newcommand{\hatF}{\hat{F}}
% g
\newcommand{\hatg}{\hat{g}}
\newcommand{\hatG}{\hat{G}}
% h
\newcommand{\hath}{\hat{h}}
\newcommand{\hatbh}{\hat\bh}
\newcommand{\hatH}{\hat{H}}
% i
\newcommand{\hati}{\hat{i}}
\newcommand{\hatI}{\hat{I}}
% i
\newcommand{\hatj}{\hat{j}}
\newcommand{\hatJ}{\hat{J}}
% k
\newcommand{\hatk}{\hat{k}}
\newcommand{\hatK}{\hat{K}}
% l
\newcommand{\hatl}{\hat{l}}
\newcommand{\hatL}{\hat{L}}
% m
\newcommand{\hatm}{\hat{m}}
\newcommand{\hatM}{\hat{M}}
% n
\newcommand{\hatn}{\hat{n}}
\newcommand{\hatN}{\hat{N}}
% o
\newcommand{\hato}{\hat{o}}
\newcommand{\hatO}{\hat{O}}
% p
\newcommand{\hatp}{\hat{p}}
\newcommand{\hatP}{\hat{P}}
% q
\newcommand{\hatq}{\hat{q}}
\newcommand{\hatQ}{\hat{Q}}
% r
\newcommand{\hatr}{\hat{r}}
\newcommand{\hatR}{\hat{R}}
\newcommand{\hatbr}{\hat{\br}}
% s
\newcommand{\hats}{\hat{s}}
\newcommand{\hatS}{\hat{S}}
% t
\newcommand{\hatt}{\hat{t}}
\newcommand{\hatT}{\hat{T}}
% u
\newcommand{\hatu}{\hat{u}}
\newcommand{\hatU}{\hat{U}}
% v
\newcommand{\hatv}{\hat{v}}
\newcommand{\hatV}{\hat{V}}
% w
\newcommand{\hatw}{\hat{w}}
\newcommand{\hatW}{\hat{W}}
% x
\newcommand{\hatx}{\hat{x}}
\newcommand{\hatX}{\hat{X}}
\newcommand{\hatbX}{\hat{\bX}}
%y
\newcommand{\haty}{\hat{y}}
\newcommand{\hatY}{\hat{Y}}
%z
\newcommand{\hatz}{\hat{z}}
\newcommand{\hatZ}{\hat{Z}}
%%%%%%%%%%%%%%
% 	Tilde English	%
%%%%%%%%%%%%%%
% A
\newcommand{\mathall}{{\rm all}}
\newcommand{\mathand}{{\rm and}}
% B
\newcommand{\BER}{{\rm BER}}
% C
\newcommand{\Cov}{{\rm Cov}}
\newcommand{\conv}{{\rm conv}}
% D
\newcommand{\diver}{\mathrm{div~}}
% E
\newcommand{\eg}{\textit{e.g.}}
\newcommand{\etc}{\textit{etc.}}
\newcommand{\etal}{\textit{et al.}}
\newcommand{\iid}{\textit{i.i.d.}}
\newcommand{\ie}{\textit{i.e.}}
% F
\newcommand{\mathfor}{{\rm for}}
% G
\newcommand{\GF}{{\rm GF}}
\newcommand{\grad}{\mathrm{grad~}}
% I

\newcommand{\mathif}{{\rm if}}
% O
\newcommand{\opt}{{\rm opt}}
\newcommand{\mathor}{{\rm or}}
\newcommand{\mathotherwise}{{\rm otherwise}}
% R
\newcommand{\rot}{\mathrm{rot~}}
% S
\newcommand{\SER}{{\rm SER}}
\newcommand{\st}{~\mathrm{s.t.}~}
% T
%\newcommand{\th}{\mathrm{th}}
% V
\newcommand{\Var}{{\rm Var}}





% Environment
%%%%%%%%%%%%
% 	 Operations 		%
%%%%%%%%%%%%
\DeclareMathOperator{\diag}{diag}
\DeclareMathOperator*{\argmin}{arg~min}
\DeclareMathOperator*{\argmax}{arg~max}
\DeclareMathOperator*{\minmax}{min~max}
\DeclareMathOperator*{\maxmin}{max~min}
\DeclareMathOperator{\lcm}{lcm}
\DeclareMathOperator{\mtxop}{mtx}
\DeclareMathOperator{\vecop}{vec}
\DeclareMathOperator{\sinc}{sinc}
\DeclareMathOperator{\sgn}{sgn}
\DeclareMathOperator{\rank}{rank}
\DeclareMathOperator{\tr}{tr}
\DeclareMathOperator{\sig}{sig}
% New Environment
\newenvironment{MyDefinition}[1]
    {\begin{itembox}[l]{\bf #1}
    \begin{definition}}%
    {\end{definition}
    \end{itembox}}
    
\newenvironment{MyTheorem}
    {\begin{shadebox}
    \bigskip
    \begin{theorem}}%
    {\end{theorem}
    \smallskip
    \end{shadebox}}
 
\newenvironment{MyProof}
    {\begin{boxnote}
    	%\begin{quote}
    %\begin{proof}
    }
    {%\end{proof}
    %\end{quote}
\end{boxnote}}
 
 \newenvironment{MyExample}
 {\begin{shaded}
 		\begin{quote}
 		\begin{example}}%
 		{\end{example}
   \end{quote}
\end{shaded}}
  
\newenvironment{MyLemma}
    {\begin{doublebox}
    \begin{lemma}}%
    {\end{lemma}
    \end{doublebox}}
    

    
% References
%\newcommand{\IEEE_L_COM}{{IEEE} Commun. Lett.~}

\newcommand{\GCOM}{{IEEE} Global Telecommun. Conf.~}
\newcommand{\ICC}{{IEEE} Int. Conf. Commun.~}
\newcommand{\ISIT}{{IEEE} Int. Symp. Infor. Theory~}
\newcommand{\IWSDA}{{IEEE} Int. Workshop Signal Design and Its Applications in Commun.~}
\newcommand{\MILCOM}{{IEEE} Military Commun. Conf.~}
\newcommand{\PIMRC}{{IEEE} Int. Symp. Personal, Indoor and Mobile Radio Commun.~}
\newcommand{\VTC}{{IEEE} Veh. Tech. Conf.~}
\newcommand{\WCNC}{{IEEE} Wireless Commun. Networking Conf.~}
\bibliographystyle{IEEEtran}
% New Theorems
\theoremstyle{definition}
% Theorem style includes plain, definition, and remark
\newtheorem{theorem}{Theorem}
%\newtheorem{algorithm}{Algorithm}
\newtheorem{definition}{Definition}
\newtheorem{example}{Example}
\newtheorem{conjecture}{Conjecture}
\newtheorem{criterion}{Criterion}
\newtheorem{lemma}{Lemma}
\newtheorem{proposition}{Proposition}
\newtheorem{corollary}{Corollary}
\newtheorem{assumption}{Assumption}
\newtheorem{remark}{Remark}
\newtheorem{problem}{Problem}[section]
\usepackage[dvipdfmx]{color}
\usepackage[dvipdfmx]{graphicx}
\usepackage[sectionbib]{chapterbib}
%\renewcommand{\bibname}{参考情報}
\definecolor{shadecolor}{rgb}{0.9,0.9,0.9}
\makeatletter
\def\th@plain{\upshape}
\makeatother

\renewcommand{\theequation}{\arabic{chapter}-\arabic{equation}}
\renewcommand{\thefigure}{\arabic{chapter}-\arabic{figure}}
%\setCJKmainfont{SimSun}
\title{Memo of C.~C.~Pinter, ``A Book of Abstract Algebra"}
\author{Kwame Ackah Bohulu}
\date{\today}
\begin{document}

\maketitle
\section{Introduction to Groups}
\begin{MyDefinition}{Groups,~$< \cG, * >$}
A set $\cG$ is called a group if it satisfies the axioms
 \begin{enumerate}
 \item operation $*$ is associative  \ie~ $(a*b) * c = a* (b * c ))$
 \item $\exists e\in \cG$ such that $a * e = e * a =a$, $\forall a \in \cG$
 \item $\forall a\in \cG$, $\exists a^{-1}\in \cG$ suth that
 $a * a^{-1} = a^{-1} * a  = e$
\end{enumerate}
If the commutative law ($a * b = b * a$) holds in the group, it is known as an Abelian group.
\end{MyDefinition}

\begin{description}
  \item[$<\bbZ,+>$] additive group of the integers
  \item[$<\bbQ,+>$] additive group of the rational numbers
  \item[$<\bbR,+>$] additive group of the real numbers
  \item[$<\bbQ^*,\cdot>$] multiplication group of the nonzero rational numbers
  \item[$<\bbR^*,\cdot>$] multiplication group of the nonzero real numbers
  \item[$\bbZ_n$] group of integers modulo $n$
\end{description}
\section{Basic Properties of Groups}
\begin{MyTheorem}
	if $\exists a,b,c \in < \cG, * >$
	,then
	\begin{enumerate}
		\item $ab=ac$ $\Rightarrow$ $b=c$ and
		\item $ba=ca$ $\Rightarrow$ $b=c$
	\end{enumerate}
\end{MyTheorem}

\paragraph{}
\begin{MyTheorem}
	if $\exists a,b \in < \cG, * >$
	,then
	\begin{enumerate}
		\item $ab =e \Rightarrow a=b^{-1}$ and
		\item $ba =e \Rightarrow b=a^{-1}$
	\end{enumerate}
\end{MyTheorem}
\paragraph{}

\begin{MyTheorem}
	if $\exists a,b \in < \cG, * >$
	,then
	\begin{enumerate}
		\item $(ab)^{-1} = b^{-1}a^{-1}$
		\item $(a^{-1})^{-1} =a$
	\end{enumerate}
\end{MyTheorem}
\begin{description}
	\item[$|<\cG,*>|$] order (number of elements) of $<\cG,*>$
\end{description}

\begin{MyDefinition}{Subgroups,~$< \cS, * >$}
	Assuming  $\exists ~<\cG, *>$ and $cS \neq \{\},~\cS \subset \cG$. If $< \cS, * >$ 
	\begin{enumerate}
		\item is closed with respect 
		to operation $*$ and
		\item closed with respect to inverses 
	\end{enumerate}
	it is a subgroup of $~<\cG, *>$. Every subgroup is also a group on its own.
\end{MyDefinition}

\begin{description}
	\item[$<2\bbZ, +>$] group of all even integers is subgroup of $<\bbZ, +>$
	\item[$<\{e\}, *>$] smallest trivial group of $<\cG, *>$
	\item[$<\cG, *>$] largest trivial group of $<\cG, *>$
\end{description}

\begin{MyDefinition}{Cyclic (sub)Group,~$<a>$}
	if $<\cG, *>$ is generated by all possible combination of operations on $a$ and $a^{-1}$ it is a cyclic group. 
	
	If the element $a$ from $<\cG, *>$ is used to generate a subgroup $<\cS, *>$ it is called a cyclic subgroup.
\end{MyDefinition}

\begin{description}
	\item[$a$] Generator of cyclic group
	\item[Defining equation of $<\cG, *>$ ]  A set of equations involving only the generators and their inverses
\end{description}
Defining equation of $<\cG, *>$ must completely describe operation table

\section{Functions}
\begin{MyDefinition}{$y=f(x)$, $f:A \mapsto B$}
		Let $\cA$ and $\cB$ be sets. A function is a rule which assigns every element of $\cA$(the domain) to a unique element in $\cB$(the range)
	\end{MyDefinition}

\begin{description}
	\item[injective function] each element of the range is the image of no more than one element of domain
	\item[surjective function] each element of the range is the image of atleast one element of the domain
	\item[bijective function] injective and surjective function
\end{description}

\begin{MyDefinition}{Composition of functions, $f \circ g$}
	Let $f:\cA \mapsto \cB$ and $g:\cB \mapsto \cC$ be functions. 
	$[f \circ g](x) :=f(g(x)) ~\forall x \in \cA$
\end{MyDefinition}


\section{Groups of Permutations}

\begin{MyDefinition}{Permutation of sets, $f:\cA \rightarrow \cA$}
	Permutation of sets is a bijective function $f:\cA \rightarrow \cA$. It forms a group with respect to composition.
	
	\end{MyDefinition}
Every permutation can be broken down into cycles.

\begin{MyDefinition}{cycles}
	let $a_1,...a_n$ be distinct elements of $\{1,2,...,n\}$. A cycle  $(a_1a_2...a_s)$ is a permutation of $\{1,2,...,n\}$ which carries $a_1$ to $a_2$, $a_2$ to $a_3$,...,$a_{s-1}$ to $a_s$ and $a_s$ to $a_1$ while leaving all the remaining elements of $\{1,2,...,n\}$ fixed.
	
	\end{MyDefinition}
	
	\begin{MyTheorem}
	Every permutation is either the identity, a single cycle or a product of disjoint cycles.
	\end{MyTheorem}
	
	\begin{description}
	\item A cycle of length $2$ is called a transposition.
	
	\item Every cycle can be expressed as a product of transpositions and for a given permutation, the number of transpositions is either always odd or always even
	\end{description}
	
	\begin{MyTheorem}
	No matter how the identity permutation is written as a product of transpositions, the number of transpositions is even.
	\end{MyTheorem}
	
	\begin{MyTheorem}
	if $\Pi \in S_n$(group of permutations length n)  then $\Pi$ cannot be both an odd and even permutation
	\end{MyTheorem}
	
	\section{Isomorphism}
	for simplicity sake, we represent a group $<\cG, *>$ by $\cG$ unless otherwise stated.
	\begin{MyDefinition}{$\cG_1\cong \cG_2$}
	Let $\cG_1$ and $\cG_2$ be groups. A bijective function $f: \cG_1 \rightarrow \cG_2$ with the property that for any two elements $a,b \in G_1$
	$$f(ab) =f(a)f(b)$$ is called an isomorphism from $\cG_1$ to $\cG_2$. if an isomorphism from $\cG_1$ to $\cG_2$ exist, then $\cG_1$ is isomorphic $\cG_2$ ($\cG_1 \cong \cG_2$)
	\end{MyDefinition}
	
	\begin{MyTheorem}
	(Cayley's Theorem)\newline Every group is isomorphic to a group of permutations
	\end{MyTheorem}
	
	\section{Order of Group Elements}
	
	\begin{MyTheorem}
	(Law of exponents)\newline 
	if $\cG$ is a group and $a\in \cG$ then $\forall ~m,n \in \bbZ$
	\begin{enumerate}
	\item $a^ma^n =a^{m+n}$
	\item $(a^m)^n = a^{mn}$
	\item $a^{-n} = (a^{-1})^n = (a^n)^{-1}$
	\end{enumerate}
	\end{MyTheorem}
	
	\begin{MyTheorem}
	(Division Algorithm)\newline
	if $m,n \in \bbZ,~n>0$ there $\exists$ unique integers $q,r$ s.t. $$m=nq+r,~\text{and} ~0 \leq r < n$$
	\end{MyTheorem}
	
	\begin{MyDefinition}{}
	if $\exists ~m \in \bbZ$ s.t $a^m=e$ then the order of $a$ is the least positive integer $m$ s.t $a^m=e$. if no such $m$ exists, $a$ has order infinity
	\end{MyDefinition}
	
	\begin{MyTheorem}
	if the order of $a$ is $n$, then there are exactly $n$ powers of $a$ given by
	$$ a^0,a^1,...,a^{n-1}$$
	\end{MyTheorem}
	
	\begin{MyTheorem}
	if the order of $a$ is infinity, then all powers of $a$ are different, ie
	$$ a^r \neq a^s$$
	\end{MyTheorem}
	
	
	\begin{MyTheorem}
	if an element $a$ in group $\cG$ has order $n$. Then $a^t=e$ iff $t$ is a multiple of $n$
	\end{MyTheorem}
	
	\begin{description}
	\item[ord($a$)] order of element $a \in \cG$
	\end{description}
	
	\subsection{Cyclic Groups ($\cG =\{ a^n : n\in \bbZ \}$)}
	order of generator $a$ determines order of cyclic group $\cG$
	
	\begin{MyTheorem}{Isomorphism of Cyclic Groups}
	\newline 
	\begin{enumerate}
	\item $\forall n >0~$ every cyclic group of order $n$ is isomorphic to $\bbZ_n$
	
	\item every cyclic group of order $\infty$ is isomorphic to $\bbZ$
	\end{enumerate}
	\end{MyTheorem}
	
	\section{Partitions and Equivalence Relations}
	\begin{MyDefinition}{Partition of a Set $\cA$}
	a family $\{ \cA_i : i\in I\}$ of non-empty subsets of $\cA$ such that 
	\begin{enumerate}
	\item if any 2 classes $\cA_i, ~\cA_j$ have a common element $x$, then $\cA_i =\cA_j$
	
	\item Every element $x$ of $\cA$ lies in one of the classes
	\end{enumerate}
	\end{MyDefinition}
	
	\begin{description}
	\item[equivalence relation ] a relation $\sim $ which is
	\begin{enumerate}
	\item reflexive : if $x \sim x \forall~ x\in \cA$
	\item symmetric : if $x \sim y$ then $y \sim x$
	\item reflexive : if $x \sim y$ and $y \sim z$ then $x \sim z$
	\end{enumerate}
	\item[equivalence of elements] means two elements are members of the same class
	
	\item[equivalence class of $x$] $[x] =\{ y\in A y \sim x \}$
	\end{description}
	
	\paragraph{Lemma:}
	if $x \sim y ~\text{then}~ [x] = [y]$
	
	\begin{MyTheorem}
	if $\sim$ is an equivalence relation on $\cA$ the family of all the equivalence classes is a partition of A
	\end{MyTheorem}
	
	\section{Counting Cosets}
	\begin{description}
	\item[$\cG$] represents a group
	
	\item[$\cH$] represents a subgroup of $\cG$
	\end{description}
	\begin{MyDefinition}{Cosets}
	 For any element $a\in\cG$, the symbol $a\cH$ denotes the set of all products $ah$ as $a$ remains constant and $h$ ranges over $\cH$ and $a\cH$ is called the \textit{left coset}. The right coset may be defined in similar fashion.
	\end{MyDefinition}
	
	\begin{MyTheorem}
	The family of all cosets $\cH a$ as a range over $\cG$ is a partition of $\cG$
	\end{MyTheorem}
	
	\begin{MyTheorem}
	if $\cH a$ is any coset of $\cH$, there is a one-to-one correspondence from $\cH$ to $\cH a$
	\end{MyTheorem}
	
	\begin{MyTheorem}
	Assume that $\cG$ is a finite group.  then $\text{ord}(\cG)=k\text{ord}(\cH)~k \in \bbZ$. This is known as Lagrange's theorem
	\end{MyTheorem}
	
	\begin{MyTheorem}
	if $\text{ord}(\cG)$ is prime, then $\cG$ is a cyclic group and all $a \in \cG, ~ a \neq e$ is a generator of the group.
	\end{MyTheorem}
	
	\begin{MyTheorem}
	The order of every element of a finite group divides the order of the group
	\end{MyTheorem}
	
	\begin{description}
	\item index of $\cH$ in $\cG$ ($\cH : \cG$) is the number of cosets of $\cH ~\text{in} ~\cG$
	\end{description}
	
	\section{Homomorphism}
	\begin{description}
	\item $\cG$ and $\cH$ be groups. 
	\item $xax^{-1}$ is a conjugate
	\end{description}
	
	\begin{MyDefinition}{}
	A homomorphism from $\cG$ to $\cH$ is a function $f: \cG \rightarrow \cH$ s.t. for any 2 elements $a,b \in \cG$ $$f(ab) = f(a)f(b)$$ The operations are preserved by the homomorphism
	\end{MyDefinition}
	
	\begin{MyTheorem}
	if a homomorphism exist between $\cG$ and $\cH$, then $\forall a \in \cG$
	\begin{enumerate}
	\item $f(e) = e$
	\item $f(a^{-1}) = [f(a)]^{-1}$
	\end{enumerate}
	\end{MyTheorem}

\begin{MyDefinition}{Normal Subgroup}
let $\cH$ be a subgroup of $\cG$. $\cH$ is called a normal subgroup of $\cG$ if it is closed with respect to conjugates, ie $$ \forall ~a\in \cH, ~x\in \cG~ xax^{-1} \in \cH$$
\end{MyDefinition}

\begin{MyDefinition}{Kernel}
let $f:\cG \rightarrow \cH$ be a homomorphism. The kernel of $f$ is the set $\cK$ of all elements of $\cG$ which are carried by $f$ onto the neutral element of $H$ ie 
$$ \cK = \{x\in \cG : f(x) = e  \}$$
\end{MyDefinition}

\begin{MyTheorem}
let $f:\cG \rightarrow \cH$ be a homomorphism. 
\begin{enumerate}
\item The kernel of $f$ is a normal subgroup of $\cG$

\item the range of $f$ is a subgroup of $\cH$
\end{enumerate}
\end{MyTheorem}

\section{Quotient Groups}
\begin{description}
\item let $\cG$ be a group and $\cH$ be a normal subgroup of $\cG$
\item 
\end{description}

\begin{MyTheorem}
$a\cH = \cH a ,~ \forall a \in \cG$
\end{MyTheorem}

\begin{MyTheorem}
if $\cH a= \cH c ~\text{and}~ \cH b = \cH d$. then  $\cH(ab) = \cH(cd)$ (Coset Multiplication)
\end{MyTheorem}

\begin{description}
\item[$\cG / \cH$] : set of all cosets of $\cH$
\end{description}

\begin{MyTheorem}
$\cG / \cH$ with coset multiplication is a group. such a group is known as a quotient/factor group of $\cG$ by $\cH$
\end{MyTheorem}

\begin{MyTheorem}
$\cG / \cH$ is a homomorphic group of $\cG$ . 
\end{MyTheorem}

\begin{MyTheorem}
if $\cG$ is a group and $\cH$ is its subgroup, then
\begin{enumerate}
\item $\cH a = \cH b$ iff $ab^{-1} \in \cH$
\item $\cH a = \cH $ iff $a \in \cH$
\end{enumerate}
\end{MyTheorem}
\section{Fundemental Theorem of Homomorphism}

\begin{MyTheorem}
let $f: \cG \rightarrow \cH$ be a homomorphism with kernel $\cK$. Then
$$ f(a)=f(b) ~\text{iff}~ \cK a =\cK b$$
\end{MyTheorem}


\begin{MyTheorem}
let $f: \cG \rightarrow \cH$ be a homomorphism with kernel $\cK$. Then
$$ \cH \tilde = \cG / \cK$$ .ie $\cH$ is isomorphic image of $\cG / \cK$
\end{MyTheorem}

\section{Rings}

\begin{MyDefinition}{Rings}
A ring is a set $\cA$ with two operations $(+, \cdot)$ which satisfy the following axioms
\begin{enumerate}
\item $\cA$ with $+$ alone is an abelian group
\item $\cdot$ is associative 
\item $\cdot$ is distributive over $+$
\end{enumerate}
\end{MyDefinition}

\begin{description}
\item $\bbZ, \bbQ,\bbC, \bbR$ are examples of rings
\end{description}

\begin{MyTheorem}
let $a,b$ be elemets of ring $\cA$. then
 \begin{enumerate}
\item $0a =a0 =0$ 
\item $a(-b)=(-a)b =-(ab)$ 
\item $(-a)(-b)=ab$
\end{enumerate}
\end{MyTheorem}

\paragraph{optional properties of rings}
\begin{enumerate}
\item if $\cdot$ is commutative in a ring it is known as a \textit{commutative ring}

\item if a multiplicative identity element exists in a ring, it is known as a \text{ring with unity}

\item if a ring $\cA$ with unity has elements with multiplicative inverse we call such elements invertible

\item if $\cA$ is a commutative ring with unity in which every nonzero element is invertible $\cA$ is called a \textit{Field}

\item in any ring, a nonzero element $a$ is called a \textit{divisor of zero} if there is a nonzero element $b$ in the ring s.t. $ba=ab=0$

\item A ring has a cancellation property if for any$a,b,c \in \cA~,a\neq 0,~ab=ac \text{or}~ba=ca \implies b=c$
\end{enumerate}

\begin{MyTheorem}
A ring has cancellation property iff it has no divisors of zero
\end{MyTheorem}

\begin{MyDefinition}{Integral Domain}
An integral domain is a commutative ring with unity which has the cancellation property
\end{MyDefinition}

\section{Ideals and Homomorphisms}
\begin{MyDefinition}{Subring}
$\cB$ is a subring of $\cA$ if it is closed with respect to addition multiplication and 
negatives
\end{MyDefinition}

\begin{description}
\item $\cB$ absorbs products of $\cA$ if $\forall b \in \cB ~\text{and}~ x \in \cA ,~xb \in \cB~\text{and}~ bx \in \cB $
\end{description}

\begin{MyDefinition}{Ideal}
A nonempty subset$\cB$ of a ring  $\cA$ which is closed with respect to addition multiplication and absorbs products in $\cA$
negatives
\end{MyDefinition}

\begin{description}
\item A homomorphism from ring $\cA$to ring $\cB$ is  a function $f: \cA \rightarrow \cB$ such that if $f(x_1)=y_1,~f(x_2)=y_2$ then 
\begin{enumerate}
\item $f(x_1 +x_2) =y_1+y_2$

\item $f(x_1x_2) =y_1y_2$
\end{enumerate}

\item if there exists a homomorphism from ring $\cA$ to ring $\cB$ then the kernel $\cK$ is given by $\cK=\{ x\in \cA : f(x) =0 \}$ and is an ideal of $\cA$
\end{description}

\section{Quotient Rings}
\begin{description}
\item $\cA,~\cB$ is a ring
\item $\cJ$ is an ideal of $\cA$
\end{description}

\begin{MyDefinition}{Coset~$\cJ + a$}
For any element $a\in \cA,~\cJ + a$ (coset) is the set of all sums $j + a$ as $a$ remains constant and $j$ ranges over $\cJ$, ie $\cJ + a = \{ j+a : j \in \cJ \}$ 
\end{MyDefinition}

\begin{description}
\item[Coset Addition] $(\cJ +a)+ (\cJ + b) = \cJ+(a+b)$ 

\item[Coset Multiplication] $(\cJ +a) (\cJ + b) = \cJ+(ab)$ 
\end{description}

\begin{MyTheorem}
if $\cJ+a =\cJ + c$ and $\cJ+b = \cJ+d$ then
\begin{enumerate}
\item $\cJ+(a+b)= \cJ+(c+d)$

\item $\cJ+(ab)= \cJ+(cd)$
\end{enumerate}
\end{MyTheorem}

\begin{description}
\item[\cA/\cJ] set of all cosets of $\cJ$ in $\cA$
\end{description}

\begin{MyTheorem}
$\cA/\cJ$ with coset addition and multiplication is a ring 
\end{MyTheorem}

\begin{MyTheorem}
$\cA/\cJ$ is a homomorphic image of $\cA$
\end{MyTheorem}

\begin{MyTheorem}
$\cB \equiv \cA/\cK$ ie $\cB$ is a homomorphic image of $\cA/\cK$ 
\end{MyTheorem}

\begin{description}
\item An ideal $\cJ$ of a commutative ring $\cA$ is said to be \textit{prime ideal} if for any two elements $a,b$ in the ring , if $ab \in \cJ$ then $a\in \cJ$ or $b\in \cJ$

\item Whenever $\cJ$ is a prime ideal of a  commutative ring with unity $\cA$, the quotient ring $\cA/\cJ$ is an \textit{ideal integral domain}

\item a proper ideal of a ring is not equal to the whole ring

\item a proper ideal is called \textit{maximal ideal} if there exists no proper ideal $\cK$ of $\cA$ such that $\cJ \subset \cK,~ \cJ \neq \cK$

\item if $\cA$ is a commutative ring with unity, then $\cJ$ is a maximal ideal of $\cA$ if $\cA/\cJ$ is a field
\end{description}

\section{Integral Multiples}
\begin{MyDefinition}{Integral Domain}
An integral domain is a commutative ring with the cancellation property(no divisors of zero)
\end{MyDefinition}

\begin{MyDefinition}{Characteristic of a Ring}
The characteristic of a ring $\cA$ is the least positive integer $n$ s.t. 
$n\cdot 1 = 0$. Else, $\cA$ has characteristic $0$
\end{MyDefinition}

\begin{MyTheorem}
all nonzero elements in an integral domain hhave the same additive order, where the additive order is the least positive  integer $n$ s.t $n \cdot a = 0$. 
\end{MyTheorem}

\begin{MyTheorem}
in an integral domain with non-zero characteristic, the characteristic is a prime number $p$
\end{MyTheorem}

\begin{MyTheorem}
in any integral domain $\cA$ with characteristic $p,~(a+b)^p=a^p + b^p \forall a,b in \cA$
\end{MyTheorem}

\begin{MyTheorem}
every finite integral domain is a field
\end{MyTheorem}

\section{The Integers}
\begin{MyDefinition}{Ordered Integral Domain}
An integral domain $\cA$ with a relation symbolized by $<$ with the following properties
\begin{enumerate}
\item for any $a,b \in \cA$ exactly one of the ff is true
$$a=b,~a<b,~b<a$$. Furthermore, for any $a,b,c \in \cA$
\item if $a <b $ and $b < c$ then $a < c$
\item if $a < b$, then $a+c < b+c$
\item if $a<b$, then $ac <bc$ if $0<c$
\end{enumerate}
\end{MyDefinition}

\begin{MyDefinition}{Integral System}
An ordered integral domain $\cA$ is an integral system if every nonempty subset of $\cA^+$ has a least element. \newline Every element of the integral system is a multiple of 1 and the integral system is isomorphic to $\bbZ$
\end{MyDefinition}

\begin{MyTheorem}
Let $\cK$ represent a set of positive integers. Consider the following two conditions
\begin{enumerate}
\item $1 \in \cK$

\item For any positive integer $k$ if $k \in \cK$, then also $k + 1 \in \cK$
\end{enumerate}

if $\cK$ is any set of positive integers satisfying these two conditions, then $\cK$ consists of all positive integers
\end{MyTheorem}

\begin{MyTheorem}
Principle of Mathematical induction. \newline Consider the following conditions
\begin{enumerate}
\item $S_1$ is true
\item For any positive integer $k$ if $S_{k}$ is true, then $S_{k+1}$ is true
\end{enumerate}
if both of the above conditions are satisfied then $S_n$ is true for every positive integer $n$
\end{MyTheorem}
\begin{description}
\item $S_n$ reperesents a statement about the positive integer $n$
\end{description}

\begin{MyTheorem}
if $m,n \in \bbZ,~0 < n, \exists q,r $ such that
$$m =nq +r,~0 \leq r < n$$
$q,r$ are the  quotient and remainder respectively and they are both unique
\end{MyTheorem}
\section{Factoring into primes} 

\begin{MyTheorem}
Every ideal of $\bbZ$ is principal
\end{MyTheorem}

\begin{MyTheorem}
The only invertible elements of $\bbZ$ are $1 ~\text{and}~ -1$
\end{MyTheorem}

\begin{MyTheorem}
Any 2 nonzero integers $r,s$ have a greatest common divisor(gcd) $t$. Also
$$ t = kr + ls ~ k,l \in \bbZ$$ 
\end{MyTheorem}


\begin{lemma}[Composite Number Lemma]
if a positive number $m$ is composite, then $m=rs$ where
$$ 1 < r <m ~\text{and} ~ 1 < s < m$$
\end{lemma}

\begin{lemma}[Euclids Lemma]
let $m,n \in \bbZ$ and $p$ be a prime number. if $p |(mn)$,then either $p|m$ or $p|n$
\end{lemma}

\begin{MyTheorem}[Factorization into prime]
Every $n \in \bbZ, n>1$ can be expressed as a product of positive primes.$$ n= p_1p_2...p_r$$
\end{MyTheorem}

\begin{MyTheorem}[Uniqe Factorization]
Suppose $n$ can be factorized into positive primes in two ways, namely $n= p_1p_2...p_r =  q_1q_2...q_t$. Then $r=t$ and $p_i,q_i$ are the same numbers except for the order in which they appear
\end{MyTheorem}

\section{Ring of Polynomials}
\begin{MyDefinition}{a(x)}
Let $\cA$ be a commutative ring with unity  and $x$ an arbitrary symbol. Every expression of the form $a_0 +a_1x+....+a_nx^n$ is called \textit{a polynomial in $x$ with coefficients in $\cA$}
\end{MyDefinition}

\begin{description}
\item[$a_kx^k$] terms of the polynomial, $k\in\{0,1,...,n \}$

\item[polynomial degree (deg $a(x)$)] the greatest n such that $a_n \neq 0$

\item[compact form of $a(x)$]  $a(x)= \sum_{k=0}^{n}a_kx^k$
\end{description}

\begin{MyTheorem}
Let $\cA$ be a commutative ring with unity. Then $\cA[x]$ is a commutative ring where $\cA[x]$ is the set of polynomials in $x$ with coefficients in $\cA$
\end{MyTheorem}

\begin{MyTheorem}
if $\cA$ is an integral domain, then $\cA[x]$ is an integral domain and it is called \textit{a domain of polynomials}
\end{MyTheorem}

\begin{MyTheorem}[Division algorithm for polynomials]
If $a(x), b(x)$ are polynomials over a finite field $\cF, b(x) \neq 0$ there exists polynomials $q(x),r(x)$ over $\cF$ s.t $$ a(x) =b(x)q(x) + r(x)$$ $r(x)=0 ~\text{or}~ \text{deg} ~r(x) < \text{deg} ~b(x) $
\end{MyTheorem}

\section{Factoring Polynomials}

\begin{MyTheorem}
Every ideal of $\cF[x]$ is principal
\end{MyTheorem}

\begin{description}
\item $a(x) ,b(x)$ are associates if they are constant multiples of each other

\item $d(x)$ is gcd of $a(x),b(x)$ if $d(x)|a(x),~,d(x)|b(x)$

\item for any $u(x) \in \cF[x]$ if $u(x) | a(x), u(x) | b(x)$ then $u(x)|d(x)$
\end{description}

\begin{MyTheorem}
Any 2 polynomials $a(x), b(x) \neq 0,~a(x),b(x) \in \cF[x]$ have a gcd $d(x)$ which can be expressed as $$ d(x) = r(x)a(x) + s(x)b(x)$$
\end{MyTheorem}

\begin{MyDefinition}{Reducible Polynomial}
A polynomial $a(x)$ with positive degree is said to be reducibe over $\cF$ if there are polynomials $b(x),c(x) \in \cF[x]$ such that $a(x) =b(x)c(x),~ \text{deg} ~b(x),\text{deg} ~c(x)>0$. otherwise $a(x)$ is irreducible over field $\cF$
\end{MyDefinition}

\begin{lemma}[Euclids Lemma for Polynomials]
let $p(x)$ be irreducible if $p(x)|a(x)b(x)$, then $p(x)|a(x) \text{and} p(x)|b(x)$
\end{lemma}

\begin{corollary}
Let $p(x)$ be irreducible. if $p(x)|a_1(x)a_2(x)...a_n(x)$, then $p(x)|a_i(x)$ for one of the factors $a_i(x)$ among $a_1(x),...,a_n(x)$
\end{corollary}

\begin{corollary}
Let $q_1(x),...q_r(x)$ and $p(x)$ be a monic irreducible polynomials. if $p(x)|q_1(x).....q_r(x)$, then $p(x)$ is equal to one of the factors $q_1(x),...,q_r(x)$ 
\end{corollary}

\begin{MyTheorem}[Factorization into irreducible polynomials]
Every polynomial $a(x)$ of positive degree in $f(x)$ can be written as a product $$ a(x) = kp_1(x)...p_r(x)$$ where $k$ is a constant in $\cF$ and $p_1(x),...,p_r(x)$ are monic irreducible polynomials of $\cF[x]$
\end{MyTheorem}

\begin{MyTheorem}[Unique Factorization]
if $a(x)$ can be written in two ways as a product of irreducibles, say $a(x)=kp_1(x)...p_r(x) =lq_1(x)...q_s(x)$ then $k=l,~r=s$ and each $p_r(x)=q_s(x)$
\end{MyTheorem}

\section{Substitution in Polynomials}
Let $a(x)=a_0 +a_1x+....+a_nx^n$. if $c \in \cF$ then $a(c)=a_0 +a_1c+....+a_nc^n$ is an element in $\cF$ obtained by substituting $c$ for $x$ in $a(x)$ and $a(x)$ is a polynomial function

if $a(x)$ is a polynomial with coefficients in $\cF$ and $c \in \cF$ such that $a(c)=0$, then $c$ is a root of $a(x)$

\begin{MyTheorem}
$c$ is a root of $a(x)$ iff $x-c$ is a factor of $a(x)$
\end{MyTheorem}

\begin{MyTheorem}
$a(x)$ has distinct roots $c_1,c_2,...,c_m \in \cF$, then $(x-c_1),..,(x-c_n)$ is a factor of $a(x)$
\end{MyTheorem}

\begin{MyTheorem}
if $a(x)$ has degree $n$, it has at most $n$ roots.
\end{MyTheorem}

\begin{MyTheorem}
if $s/t$ is a root of $a(x)$, then $s|a_0,~\text{and}~t|a_n$
\end{MyTheorem}

\begin{Lemma}
Let $a(x)=b(x)c(x),~\text{coefficient of }~a(x),b(x),c(x) \in \bbZ$
if a prime number $p$ divides every coefficient of $a(x)$, it either divides every coefficient of $b(x)$ or every coefficient of $c(x)$
\end{Lemma}

\begin{MyTheorem}
Let $a(x) \in \bbZ[x]$. Suppose $a(x)=b(x)c(x)$ where $b(x),c(x)$ have rational coefficients. Then there are polynomials $B(x),C(x)$ with integer coefficients which are integer coefficients of $b(x),c(x)$ such that $a(x)=B(x)C(x)$
\end{MyTheorem}

\begin{MyTheorem}[Einstein's irreducibility criterion]
Let $a(x)$ be a polynomial with integer coefficients. Suppose there is a prime number $p$ which divides every coefficient of $a(x)$ except the leading coefficient $a_n$, suppose $p$ doesnt divide $a_n$ and  $p^2$ doesnt divide $a_n$. Then $a(x)$ is irreducible over $\bbQ$
\end{MyTheorem}


\end{document}
