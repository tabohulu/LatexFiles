\documentclass[24 pts]{article}
\usepackage{xeCJK}
\usepackage{amssymb}
\usepackage{amsmath}
\usepackage{amsthm}
\usepackage{graphicx}
\graphicspath{ {images/} }
\usepackage{relsize}

\newcommand{\me}{\mathrm{e}}
\setCJKmainfont[BoldFont= Yu Mincho Demibold]{MS Mincho}
\title{国際社会の政治・経済レポート	 }
\date{04-08-2016}
\author{Kwame Ackah Bohulu 1631133}
\begin{document}
\maketitle
\section{WHY THE UNITED STATES GOVERNMENT WOULD NOT RETREAT THE MARINE CORPS}
For years, the relocation of the U.S. marine corps base has been a major political issue for Okinawa, Japan and the US military and diplomacy in Asia. The following may be why the United States government is yet to retreat the Marine corps from their Okinawa base.
\subsection{Strategic Location}
The location of the U.S. marine corps base in Okinawa has been said to a “superior” strategic location, that is it enables the USA to monitor the countries that pose a threat to the security of its allies in the region and deploy the necessary contingencies in time to help its allies if need be. For example, should North Korea decide to invade South Korea, the U.S. Marines on Okinawa would play a critical role in Operations Plan 5027, which is the joint U.S.–South Korean war plan for responding to a North Korean invasion. This was illustrated by the North Korean attack on Yeonpyeong Island in November 2010. As a result, Seoul augmented its own 27,000-member Marine Corps by 2,000, thereby bolstering its ability to defend the five islands in the West Sea. 
Aside situations of war, the USA and South Korea have also developed Concept Plan 5029 to respond to crisis contingencies which would include limited amphibious raids and full-scale amphibious assaults, airfield and port seizure operations, maritime interdiction operations, amphibious advanced force operations, stability operations, and tactical air support. 
Also the presence of the U.S. marine corps helps the U.S. conduct humanitarian support. The Okinawa Marines have routinely been the primary responders to major natural disasters in Asia, such as the 2004 Asian tsunami, mudslides in the Philippines, and the typhoon in Taiwan and the March 2011 natural disasters in Japan. The Marines have led or participated in 12 significant humanitarian assistance–disaster relief (HADR) missions during the past five years alone, helping to save hundreds of thousands of lives in the region.
\subsection{Cheaper Operation Cost}
It is much cheaper for the U.S. government to maintain the operations of its base in Okinawa than on U.S. soil. According to a 2004 report by the U.S. Department of Defense, Japan contributed direct financial support worth \$3.23 billion and indirect support worth \$1.18 billion in fiscal 2002, which offset as much as 74.5 percent of the total costs for the U.S. to station its forces in Japan. Japan's direct financial support includes paying the salaries of some 25,000 non-military workers at U.S. military facilities in Japan. Japan also pays for the electricity, gas, water and sewage as well as for the cooking and heating fuels at U.S. military housing facilities. This is probably an incentive for the U.S. to maintain the Military Corps in Okinawa.
\end{document}