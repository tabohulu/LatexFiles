\documentclass[20 pts]{article}
\usepackage{xeCJK}
\usepackage{amsfonts}
\usepackage{amssymb}
\usepackage{amsmath}
\usepackage{bm}
\setCJKmainfont{SimSun}
\title{プログラミング言語基礎論第1回レポート} 
\author{Kwame Ackah Bohulu, 1631133}
\date{22-06-2017}
\begin{document}
\maketitle

\newpage
\paragraph{[1]}

primes :: Int ->[Int]\newline
primes n =seive [2..]\newline
       where \newline
       seive (p:xs)= p:seive (take n [x|x <- xs, mod x p > 0])\newline
       seive[]=[]\newline
       
    \paragraph{[2]}
       poly :: (Num a) => [a] -> a -> a\newline
poly [] v=0\newline
poly (x:xs) v = x*v + poly xs v\newline
       
        \paragraph{[3]}
       
       fib2 :: Int -> Int\newline
fib2 n = fibsub n (1, 1)\newline
fibsub :: Int -> (Int, Int) -> Int\newline
fibsub n (x,y) | n== 0 =x\newline
               | otherwise = fibsub (n-1) (y,x+y)\newline
               
               fib2関数は0.00秒で値を計算して、メモリ量は102984 bytesなので、fib2関数は効率的である。
               
               \paragraph{[5]}
               (1). data BETree a = BLeaf a | BNode a (BETree a) (BETree a)\newline
               deriving (Eq, Show)\paragraph{}
               (2). sumBETree :: (Num a) => a -> BETree a -> a\\
sumBETree u (BLeaf x) = u + x\\
sumBETree u (BNode x lt rt) =  sumBETree (x + sumBETree u lt) rt\\

\paragraph{}
depthBETree :: BETree a -> Int\\
depthBETree (BLeaf x) = 0 \\
depthBETree (BNode x lt rt) = 1 + max (depthBETree lt) (depthBETree rt)\\
\paragraph{}

upAccBETree :: (Num a) => a -> BETree a -> BETree a\\
upAccBETree u (BLeaf x) = BLeaf (u+x)\\
upAccBETree u (BNode x lt rt) = BNode(sumBETree (x + sumBETree u lt) rt) ((upAccBETree u lt)) (upAccBETree u rt)\\
\newpage
\paragraph{[6]}
               
               
               f :: Int -> [Int]\\
f a = if a <= 0 then [0] else a : f (a-1)\\

g :: (Num a) => ([a],[a]) -> [a]\\
g ([],ys)=[]\\
g (xs,[])=[]\\
g ((x:xs), (y:ys)) = (x+y) : g(xs,ys)\\

\end{document}