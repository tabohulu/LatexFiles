\documentclass[20 pts]{article}
\usepackage{xeCJK}
\usepackage{amsfonts}
\usepackage{amssymb}
\usepackage{amsmath}
\usepackage{bm}
\setCJKmainfont{SimSun}
\title{IT最前線レポート1} 
\author{Bohulu Kwame Ackah, 1631133}
\date{2017/10/24}
\begin{document}
\maketitle

\newpage
\paragraph{【1】}
\textbf{第5 世代(5G)移動通信ネットワークについて、第4 世代\\ネットワークと
比較して可能となる仕様条件を示してください。}

\paragraph{}
可能となる仕様条件は以下で示される。

\paragraph{[a]}大容量化\\
今まで使用する4Gのアンテナの中の容量が5Gと比べて、$1Km^2$当たりの中かの容量が約1000倍となります。
\paragraph{[b]}高速通信\\
5Gのユーザ体感スループットが最大で基本今より100倍(100Gbps)となります。

\paragraph{[c]}艇遅延化\\
今使用する4Gの無線空間の遅延が数100msなんですが、5Gの無線空間の遅延が1ms以下のなりる。

\paragraph{[d]}超多数端末の同時接続\\
ものをつなごうとすると、(IoTなど)今までの同時端末数より100倍となる。
\paragraph{[e]}低コスト化・省電力化\\
また4Gと比べると、通信量当たりのネットワークコストが減らし、IoTなどのサポートがうまくいけるようになる。

\newpage
\paragraph{【2】}コネクテッド・カーを実現する上で標準化が必要になる理由について、具体
的な事例を挙げて説明してください。\\
\paragraph{}
コネクテッド・カーを実現するのに必要なところが、車会社関わらず車と車が情報交換ができることです。自動運転できるToyotaとMazdaの車があるとします。
それぞれの車に使用する送信方法、受信方法、ソフトウェアなどが違う。
通信と受信方法が違う場合、もしToyotaの車がMazdaの車に必要な情報あっても交換できないです。
\paragraph{}そしてそれぞれの車に採用されたソフトウェアの判断アルゴリズムなどが違ったら、事故を起こすことが可能です。
こういった問題を防止するため、送信方法、受信方法、ソフトウェアに使用する判断アルゴリズムなどの標準化が必要です。


\newpage
\paragraph{【3】}あなた自身のキャリアパスを考える上で参考になった点を書いてください。\\
自分のキャリアパスを考えたうえで参考になった点は標準化の重要性です。標準化がないと、目的が達成できないです。
\paragraph{【4】}本講義についてのコメントを書いてください\\
情報が多かったですが、非常に面白かったです。勉強になりました。
\end{document}