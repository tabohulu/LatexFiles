\documentclass[20 pts]{article}
\usepackage{xeCJK}
\usepackage{amsfonts}
\usepackage{amssymb}
\usepackage{amsmath}
\usepackage{bm}
\setCJKmainfont{SimSun}
\title{IT最前線レポート10} 
\author{Bohulu Kwame Ackah, 1631133}
\date{2018/01/09}
\begin{document}
\maketitle

\newpage
\paragraph{【1】制御システムと、安全システムを比較し、共通している部分、異なる部分を
説明してください。 }
\paragraph{}
制御システムと、安全システムの共通している部分は、操作監視機能がついています。

異なる部分は、安全システムの場合、操作監視機能は動作の定義をしてから作られている。ほかの安全システムに使うのはできないです。
制御システムの場合、機械の監視によく使われているので、徳化する必要はない。

そして、安全システムは寝ずに監視していて、危険なら機械を確実に停止する。制御システムは調節やシーケンスを実行して、プラントの運転を制御する。

\paragraph{【2】安全システムのソフトウェアの Systematic failure の対策として重要な取
り組みを 2 つ上げて、それぞれ説明してください。 }
\paragraph{}




\newpage
\paragraph{【3】}あなた自身のキャリアパスを考える上で参考になった点を書いてください。\\
参考になった点は、安全システムの安全性のじゅうようさです。もしウイルスが操作監視機能に入力したら、危険なら機械を確実に停止できなくなったら、さいがいになるかもしれない。



\paragraph{【4】}本講義についてのコメントを書いてください\\
面白かったです。説明も良かったです。


\end{document}