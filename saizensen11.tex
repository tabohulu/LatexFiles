\documentclass[20 pts]{article}
\usepackage{xeCJK}
\usepackage{amsfonts}
\usepackage{amssymb}
\usepackage{amsmath}
\usepackage{bm}
\setCJKmainfont{SimSun}
\title{IT最前線レポート11} 
\author{Bohulu Kwame Ackah, 1631133}
\date{2018/01/016}
\begin{document}
\maketitle

\newpage
\paragraph{【1】}開発手法の種類と、その特徴を示してください。 \\
\paragraph{}
開発手法は、ウオーターフォール開発とアジャイル開発である。ウオーターフォール開発とはプロジェクト
の各工程を一気に実施する手法である。利点としては、全体の工程管理がしやすい。そして、開発の期間が短い。
弱点としては、仕様全て決めないと進めないような開発手法である。そして、開発の最終段階まで、認識のズレの
確認が困難である。最後に、仕様の変更に対応することが困難である。\\

アジャイル開発とは、長い開発期間単位の開発を繰り返す手法である。利点としては、要件委変更が発生しても影響範囲が小さいである。
そして、重要顧客が早い段階で動くものを確認することができ、認識合わせが可能である。最後に、優先度高い重要な機能から順次開発が可能である。

\paragraph{【2】}アジャイル開発において『順次開発できる・柔軟であるという意味をはき違
えないこと』とはどういうことか示してください。\\
\paragraph{}
『順次開発できる』と聞くと、顧客に見せるものができて、開発する方も不確定要素がが少ない段階で開発は進められるという履き違い
ありますが、本当の意味は
必要なものを準備できて、ちゃんとう理解して、顧客と認識を合わせてから開発をする。\\

『柔軟である』と聞くと、いつでも新しいものが入れられるので最初から決めなくてもいいという履き違いがありますが、
本当の意味は、開発する中入れようとしたものが入れられなかったら、新しいものにするのが可能である。


\newpage
\paragraph{【3】}あなた自身のキャリアパスを考える上で参考になった点を書いてください。\\
参考になった点は、利用する開発手法によって作る出す物の品質の関係である。



\paragraph{【4】}本講義についてのコメントを書いてください\\
勉強になりました。説明も良かったです。


\end{document}