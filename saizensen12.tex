\documentclass[20 pts]{article}
\usepackage{xeCJK}
\usepackage{amsfonts}
\usepackage{amssymb}
\usepackage{amsmath}
\usepackage{bm}
\setCJKmainfont{SimSun}
\title{IT最前線レポート12} 
\author{Bohulu Kwame Ackah, 1631133}
\date{\today}
\begin{document}
\maketitle

\newpage
\paragraph{【1】}開発手法の種類と、その特徴を示して企業がオープンソースを戦略的に活用
しようと考えた場合に、ソースコード公開以外にどのような活動を行ってい
く必要があるか説明してください。 \\
\paragraph{期限までに完成なければならない}:
クローズドソースソフトウェア企業は、独自の時間にソフトウェアを設計し、リリースします。
 しかし、オープンソースに変えることで、
顧客を満足させるために時間通りに製品を完成させ、リリースする必要があります。
\vspace{8mm}
\paragraph{商品として品質保証があることが当然期待される}:
オープンソースソフトウェアは多くの人々によって使用され、製品が使用されるためには、
いくつかの品質要件を満たす必要があります。そうでなければ、会社はお金を失います。
\vspace{8mm}          
\paragraph{開発規模がけた違いに大きい}:
オープンソースソフトウェアでは、多くのプラットフォームで実装を検討する必要があります。 
これにより、必要なソフトウェアを作成するために必要なチームのサイズが大きくなります。



\vspace{8mm}
\paragraph{【2】}破壊イノベーションを企図したオープンソース戦略は日本企業にどのような
インパクトを与え、どのような対策が必要か説明してください。\\

採用されているオープンソース戦略は、コミュニティ開発です。 この戦略は、
企業がソフトウェアの最初のリリースに存在するバグを見つけて修正しやすくし、
開発時間を短縮し、より良いイノベーションをもたらすようにする。




\newpage
\paragraph{【3】}あなた自身のキャリアパスを考える上で参考になった点を書いてください。\\
参考になった点は、ソフトウェア企業に対するOSS の影響ということです。



\paragraph{【4】}本講義についてのコメントを書いてください\\
勉強になりました。説明も良かったです。


\end{document}