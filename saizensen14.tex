\documentclass[20 pts]{article}
\usepackage{xeCJK}
\usepackage{amsfonts}
\usepackage{amssymb}
\usepackage{amsmath}
\usepackage{bm}
\setCJKmainfont{SimSun}
\title{IT最前線レポート14} 
\author{Bohulu Kwame Ackah, 1631133}
\date{\today}
\begin{document}
\maketitle

\newpage
\paragraph{【1】}ポスト「京」に向けて、2段階のステップを踏んだ開発を行っていますが、
第一段階でエクサを見据えてチャレンジした4つの技術について述べてくだ
さい。また、何故、それらが重要なのか、簡潔に述べてください。 \\
\paragraph{}
エクサを見据えてチャレンジした4つの技術は以下で述べている。
%\vsapce{8mm}
\paragraph{最新プロセステクノロジ}:
CPUの大きさを45nmから20nmにすることチャレンジである。
最新テクノロジを使うことによってCPUの微細化が可能になる。
そして競合力強化も上がってくる。
%\vsapce{8mm}
\paragraph{積層DRAM}:
京の場合DDRを使ったが、エクサのメモリ、DDRからHMCに変更するチャレンジもある。
サーバ用メモリとして世界初で、HMCを使うことによって圧倒的メモリバンド幅を提供することが可能
です。
%\vsapce{8mm}
\paragraph{SoC}:
エクサでチャレンジしようとするっことは、1システムを1チップに凝縮することです。
それをすることで、一つのサーバにたくさんのシステムを入れるのが可能になる。
%\vsapce{8mm}
\paragraph{光伝送}:
最後に、エクサでは電気ケーブルから光ケーブルを変更するチャレンジあります。
最先端光テクノロジで、ノード間データ転送性能が 12.5 Gbps x 2(双方向)/リンクのスペードが得られる。
%\vsapce{8mm}

\paragraph{【2】}「京」の世界一達成においては、開発面では4つの技術がキーとなりました
が、その内の1つ「システム技術」では水冷を使用しました。何故、水冷を
使用したのか、その理由を簡潔に3つ述べてください。\\
\paragraph{}
水冷を使用した理由,以下で述べてある。
\paragraph{スペースを増やすため :}
水冷を使用することで、たくさんの扇風機を入れる必要がなくなってメモリなどを入れられるスペースが
増やしました。

\paragraph{サーバからの雑音を小さくする :}
たくさんの扇風機を入れる必要がなくなったら、サーバからの雑音がかなりすくなくなる。

\paragraph{重要な部分を冷却 :}
水冷を使用することで、CPUなどの重要な部分だけを冷却することが可能です。

\newpage
\paragraph{【3】}あなた自身のキャリアパスを考える上で参考になった点を書いてください。\\
参考になった点は、いい開発をするために他の人の意見を聞く必要があるということでした。



\paragraph{【4】}本講義についてのコメントを書いてください\\
勉強になりました。説明も良かったです。


\end{document}