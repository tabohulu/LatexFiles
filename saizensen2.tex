\documentclass[20 pts]{article}
\usepackage{xeCJK}
\usepackage{amsfonts}
\usepackage{amssymb}
\usepackage{amsmath}
\usepackage{bm}
\setCJKmainfont{SimSun}
\title{IT最前線レポート2} 
\author{Bohulu Kwame Ackah, 1631133}
\date{2017/10/31}
\begin{document}
\maketitle

\newpage
\paragraph{【1】}FinTech とは何の略で、どのようなサービスのことを意味しているか説明し
てください。
\paragraph{}
FinTech とは英語のFinance + Technologyの略です。
スマホアプリを使用して、銀行振込、銀行口座残高の確認ができるサービス(API系)、スマホやタブレットに簡易デバイスを追加しカード対応などのサービス(決済系)、
分散台帳で取引の記録を多数で保証できるサービス(取引系)などがFinTechを意味する。

\paragraph{【2】}銀行などでユーザーID を認証し、他のFinTech 企業などに自社をアクセスす
る許可を与える認証認可方式の標準は何ですか。また、その標準を使うと従
来と何が変わるのかを説明してください。\\

\paragraph{}
銀行などでユーザーID を認証し、他のFinTech 企業などに自社をアクセスす
る許可を与える認証認可方式の標準はAPIと言います。
従来の場合、FinTech 企業が各ユーザーのパスワードとIDをフォルダに保存する。次にユーザーの銀行を調べて、銀行の画面に合わせて(画面シミュレーションを行う)パスワードとIDを入力する。最後に情報をもらってお客様が
使用するアプリに移す。
\paragraph{}
現在の標準と従来の大違いはFinTech 企業が各ユーザーのパスワードとIDがわからないということです。ユーザーが銀行の画面にパスワードとIDを記入(画面シミュレーションを行わない)。パスワードとIDが正しければ、銀行がFinTech 企業tokenを渡す。
そのtokenを使用して、必要な情報をお客様に見せる。



\newpage
\paragraph{【3】}あなた自身のキャリアパスを考える上で参考になった点を書いてください。\\
FinTechAPIについて参考に点は、どんな業務でも現在で使用する方法は改善できる。


\paragraph{【4】}本講義についてのコメントを書いてください\\
面白かったです。説明も良かったです。


\end{document}