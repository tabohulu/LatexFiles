\documentclass[20 pts]{article}
\usepackage{xeCJK}
\usepackage{amsfonts}
\usepackage{amssymb}
\usepackage{amsmath}
\usepackage{bm}
\setCJKmainfont{SimSun}
\title{IT最前線レポート3} 
\author{Bohulu Kwame Ackah, 1631133}
\date{2017/11/14}
\begin{document}
\maketitle

\newpage
\paragraph{【1】コンピュータ上で三次元データを扱うための代表的な方法 2 つについて、それぞれどのような表現方法であるか、それぞれの表現方法の特徴は何か答えてください。}
\paragraph{}
コンピュータ上で三次元データを扱うための代表的な方法 はB-rep表現とポリゴン表現です。

B-rep表現方法は方程式表される面とその境界および隣接関係による表現。とポリゴン表現方法は三角形の集まりで形状の表現である。

B-rep表現方法の特徴はデータサイズが小さい。そして、構成的にデータを構築する必要がある。

ポリゴン表現方法はデータサイズが大きい。そして、座標値の集まり(点群)から比較的容易に作成できる。

\paragraph{【2】}3D レーザースキャナ等により広範囲の空間情報が取得できるようになるとどのようなことが実現できそうと考えられるか、具体的に答えてください。\\





\newpage
\paragraph{【3】}あなた自身のキャリアパスを考える上で参考になった点を書いてください。\\
自分のキャリアパスは情報通信系の研究、特に符号です。コンピュータによる三次元形状データが使えば、通信ネットワークのモデルが作られて、元詳しい研究
ができると思います。


\paragraph{【4】}本講義についてのコメントを書いてください\\
面白かったです。元々興味にあった話なので期待通りになった。説明も良かったです。


\end{document}