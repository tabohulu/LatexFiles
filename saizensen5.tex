\documentclass[20 pts]{article}
\usepackage{xeCJK}
\usepackage{amsfonts}
\usepackage{amssymb}
\usepackage{amsmath}
\usepackage{bm}
\setCJKmainfont{SimSun}
\title{IT最前線レポート5} 
\author{Bohulu Kwame Ackah, 1631133}
\date{\today}
\begin{document}
\maketitle

\newpage
\paragraph{【1】}反復に基づく開発プロセスが必要となった理由・背景について述べ、ウォー
ターフォール型開発プロセスと、反復に基づく開発プロセスの各々について
利点と欠点(難しい点)について述べてください。 \\
\paragraph{}
反復に基づく開発プロセスが必要となった理由・背景は最初にソフトウェア工学真の要求は一回の
要求定義では定まらないという共通認識です。
\paragraph{}
そして、実現可能性や性能の面も実装してテスト
してみないとリスクは低減されないである。
\paragraph{}
最後に、ユーザインタフェースの操作性は実際に働くシステムで試されなければ改善されないである。

\paragraph{}
ウォーターフォール型開発プロセスの利点と欠点以下で述べてある。

\paragraph{利点}:
利点としては開発方法として定着しており、開発要員の教育がしやすいです。そして、各段階
ごとの策業分担がしやすい。最後に、大規模システムの開発の基づいている。
\paragraph{欠点}:
欠点としては要求定義
の結果がコンピューター上の動作で確認されるまで長期間を要する。最後に、作業中のある階段で
遅れが生じるとそれ以降の階段に遅れが普及し次々に遅れが生じる。

\paragraph{}
反復に基づく開発プロセスの利点と欠点以下で述べてある。

\paragraph{利点}:
利点としては、文書化に要する時間が短く、設計に多くの時間がかかる。そして、
信頼できるユーザーのフィードバックを得ることができる。最後に、
反復モデルでは、段階的に製品を構築し改善しているので 、
初期段階で欠陥を追跡することができる。
\paragraph{欠点}:
欠点としてはすべての要件がライフサイクル全体にわたって前もって集められているわけではないため、
高価なシステムアーキテクチャや設計上の問題が発生する可能性がある。


\paragraph{【2】}海外と日本の拠点で一つのソフトウェアを共同開発する際に生じる困難につ
いて説明してください。\\


\paragraph{言葉違い:}
日本と海外で使ってる言葉が違うであれば、聞き間違いが起こりやすいです。

\paragraph{時間違い:}
時間違いのせいでもし急な相談あれば、どうしても相手の働いている時間しか相談できないから
プロジェクトの遅れが生じやすいです。


\newpage
\paragraph{【3】}あなた自身のキャリアパスを考える上で参考になった点を書いてください。\\
参考になった点は、プロジェクトに対して開発プロセスの選択の重要さということでした。



\paragraph{【4】}本講義についてのコメントを書いてください\\
勉強になりました。説明も良かったです。


\end{document}