\documentclass[20 pts]{article}
\usepackage{xeCJK}
\usepackage{amsfonts}
\usepackage{amssymb}
\usepackage{amsmath}
\usepackage{bm}
\setCJKmainfont{SimSun}
\title{IT最前線レポート6} 
\author{Bohulu Kwame Ackah, 1631133}
\date{\today}
\begin{document}
\maketitle

\newpage
\paragraph{【1】}本人の確認手段にはどのような種類があるか。またそれぞれの利点・欠点を
示してください \\
\paragraph{}
本人の確認手段にはどのような種類は身体特徴と行動特徴です。
\vspace{8mm}
\paragraph{身体特徴}
利点としては、忘れたりなくしたりすることは不可能です。そして、行動特徴と比べるとセキュリティーが
高いです。欠点としては、こういうシステムを作るのに追加のハードウェアが必要。そして、
身体特徴に基づくシステムは100%正確ではない。
\vspace{8mm}          
\paragraph{行動特徴}
利点としては、行動特徴に基づくシステムを作るのが簡単です。欠点としては、偽造しやすい。そして、
身体特徴に基づくシステムは100%正確ではない。



\vspace{8mm}
\paragraph{【2】}顔による防犯カメラシステムが広がっていくために必要な技術的要件を述べ
るとともに、実社会で本当に広まる可能性があるかを論じてください。\\

顔による防犯カメラシステムが広がっていくために必要な技術的要件は
リアルタイムで顔認証できるアルゴリズムと顔認証のデータベースが必要になる。
実社会で広まる可能性があると思います。現在でも多くの駅に防犯カメラはちゃんとついている。問題
点、プライバシーでに対するルールがちゃんと作れば本当に広まると思う。





\newpage
\paragraph{【3】}あなた自身のキャリアパスを考える上で参考になった点を書いてください。\\
参考になった点は、顔認証に対する問題です。



\paragraph{【4】}本講義についてのコメントを書いてください\\
勉強になりました。説明も良かったです。


\end{document}