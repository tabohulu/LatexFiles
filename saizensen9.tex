\documentclass[20 pts]{article}
\usepackage{xeCJK}
\usepackage{amsfonts}
\usepackage{amssymb}
\usepackage{amsmath}
\usepackage{bm}
\setCJKmainfont{SimSun}
\title{IT最前線レポート9} 
\author{Bohulu Kwame Ackah, 1631133}
\date{\today}
\begin{document}
\maketitle

\newpage
\paragraph{【1】}現在、企業において活用されている主なビッグデータ分析や活用の例として、
講義では3つ述べました(①マーケットバスケット分析、②ソーシャルメデ
ィア分析、③チャットボット)。この3つの活用事例について、その概要と、
内部的に利用されている主なIT 技術について説明してください。 \\
\paragraph{マーケットバスケット分析}:
マーケットバスケット分析では顧客は、予め定義されたパラメータに従ってグループに分けられ、
そのショッピングデータが記録され、予め定義されたグループ内の他の顧客にとって何
が魅力的であるかに関する予測を行うために使用される。使用するIT 技術はデータマイニングです。
\vspace{8mm}
\paragraph{ソーシャルメディア分析}:
ツイッターなどのを使用して、ある会社の製品について顧客が
何を考えているのかを知ることができ、将来の製品を改善するために使用することができる。
使用するIT 技術はデータマイニングです。
\vspace{8mm}          
\paragraph{チャットボット}:
顧客サービス担当者に連絡する代わりに、チャットボットを使用して、
企業が提供する製品またはサービスに関連する一般的な問題を解決する。使用するIT 技術は
人工知能です。



\vspace{8mm}
\paragraph{【2】}今後、企業や組織において人工知能による業務の支援が広まっていくと考え
られます。企業や組織においては、様々な人工知能や機械学習、深層学習を
活用したIT システムが複数、構築し運用されていくことになりますが、この
際に重要になる概念について述べて、その概要について説明してください。\\

最小の遅延ですべてのデータ交換を処理するのに高速通信が必要です。
このようなネットワークがなければ、IOTの実装は不可能になる。 
また、さまざまなデバイスすべてをインターフェースする方法を見つける必要がある。 
そのようなインターフェースがなければ、他のすべてのデバイスが情報をやりとりして交換することは難しいです。





\newpage
\paragraph{【3】}あなた自身のキャリアパスを考える上で参考になった点を書いてください。\\
参考になった点は、色んな分野でIT技術が流行ってるということです。



\paragraph{【4】}本講義についてのコメントを書いてください\\
勉強になりました。説明も良かったです。


\end{document}