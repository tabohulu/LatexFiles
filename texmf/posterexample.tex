\documentclass[dvipdfmx]{beamer}
\newcommand{\folder}{/usr/local/share/texmf}
%%%%%%%%%%%%
% 	 UsePackege	%
%%%%%%%%%%%%
\usepackage{amssymb}
\usepackage{amsmath}
\usepackage{amsthm}
\usepackage{ascmac}
\usepackage{lscape}
\usepackage{pifont}
\usepackage{cite}
\usepackage{ifthen}
\usepackage{framed}
\usepackage{mathrsfs}
\usepackage{booktabs}
\usepackage{algorithm}
\usepackage{algorithmic}
\usepackage{comment}
\usepackage{longtable}
% Font
%%%%%%%%%%%%%%
% 	Bold Number	%
%%%%%%%%%%%%%%
\newcommand{\bzero}{{\bf 0}}
\newcommand{\bone}{{\bf 1}}
%%%%%%%%%%%%%%
% 	Bold English	%
%%%%%%%%%%%%%%
% a
\newcommand{\ba}{{\mbox{\boldmath$a$}}}
\newcommand{\bA}{{\mbox{\boldmath$A$}}}
\newcommand{\sba}{{\mbox{\scriptsize\boldmath $a$}}}
% b
\newcommand{\bb}{{\mbox{\boldmath$b$}}}
\newcommand{\bB}{{\mbox{\boldmath$B$}}}
\newcommand{\sbb}{{\mbox{\scriptsize\boldmath $b$}}}
% c
\newcommand{\bc}{{\mbox{\boldmath$c$}}}
\newcommand{\bC}{{\mbox{\boldmath$C$}}}
\newcommand{\sbC}{{\mbox{\scriptsize\boldmath $C$}}}
\newcommand{\sbc}{{\mbox{\scriptsize\boldmath $c$}}}
% d
\newcommand{\bd}{{\mbox{\boldmath$d$}}}
\newcommand{\bD}{{\mbox{\boldmath$D$}}}
\newcommand{\sbd}{{\mbox{\scriptsize\boldmath $d$}}}
% e
\newcommand{\be}{{\mbox{\boldmath$e$}}}
\newcommand{\bE}{{\mbox{\boldmath$E$}}}
\newcommand{\sbe}{{\mbox{\scriptsize\boldmath $e$}}}
% f
\newcommand{\bbf}{{\mbox{\boldmath$f$}}}
\newcommand{\bF}{{\mbox{\boldmath$F$}}}
% g
\newcommand{\bg}{{\mbox{\boldmath$g$}}}
\newcommand{\bG}{{\mbox{\boldmath$G$}}}
% h
\newcommand{\bh}{{\mbox{\boldmath$h$}}}
\newcommand{\bH}{{\mbox{\boldmath$H$}}}
\newcommand{\sbH}{{\mbox{\scriptsize\boldmath$H$}}}
\newcommand{\sbh}{{\mbox{\scriptsize\boldmath$h$}}}
% i
\newcommand{\bi}{{\mbox{\boldmath$i$}}}
\newcommand{\sbi}{{\mbox{\scriptsize \boldmath$i$}}}
\newcommand{\bI}{{\mbox{\boldmath$I$}}}
% i
\newcommand{\bj}{{\mbox{\boldmath$j$}}}
\newcommand{\sbj}{{\mbox{\scriptsize \boldmath$j$}}}
\newcommand{\bJ}{{\mbox{\boldmath$J$}}}
% k
\newcommand{\bk}{{\mbox{\boldmath$k$}}}
\newcommand{\bK}{{\mbox{\boldmath$K$}}}
% l
%\newcommand{\ell}{{\mbox{\boldmath $l$}}}
\newcommand{\bl}{{\mbox{\boldmath$l$}}}
\newcommand{\bL}{{\mbox{\boldmath$L$}}}
% m
\newcommand{\bm}{{\mbox{\boldmath$m$}}}
\newcommand{\sbm}{{\mbox{\scriptsize \boldmath$m$}}}
\newcommand{\bM}{{\mbox{\boldmath$M$}}}
% n
\newcommand{\bn}{{\mbox{\boldmath$n$}}}
\newcommand{\bN}{{\mbox{\boldmath$N$}}}
\newcommand{\sbn}{{\mbox{\scriptsize \boldmath$n$}}}
\newcommand{\sbN}{{\mbox{\scriptsize\boldmath $N$}}}
% o
\newcommand{\bo}{{\mbox{\boldmath$o$}}}
\newcommand{\bO}{{\mbox{\boldmath$O$}}}
% p
\newcommand{\bp}{{\mbox{\boldmath$p$}}}
\newcommand{\bP}{{\mbox{\boldmath$P$}}}
\newcommand{\sbp}{{\mbox{\scriptsize \boldmath$p$}}}
% q
\newcommand{\bq}{{\mbox{\boldmath$q$}}}
\newcommand{\bQ}{{\mbox{\boldmath$Q$}}}
% r
\newcommand{\br}{{\mbox{\boldmath$r$}}}
\newcommand{\bR}{{\mbox{\boldmath$R$}}}
\newcommand{\sbr}{{\mbox{\scriptsize\boldmath$r$}}}
% s
\newcommand{\bs}{{\mbox{\boldmath$s$}}}
\newcommand{\sbs}{{\mbox{\scriptsize \boldmath$s$}}}
\newcommand{\bS}{{\mbox{\boldmath$S$}}}
% t
\newcommand{\bt}{{\mbox{\boldmath$t$}}}
\newcommand{\bT}{{\mbox{\boldmath$T$}}}
% u
\newcommand{\bu}{{\mbox{\boldmath$u$}}}
\newcommand{\bU}{{\mbox{\boldmath$U$}}}
\newcommand{\sbu}{{\mbox{\scriptsize\boldmath $u$}}}
% v
\newcommand{\bv}{{\mbox{\boldmath$v$}}}
\newcommand{\bV}{{\mbox{\boldmath$V$}}}
% w
\newcommand{\bw}{{\mbox{\boldmath$w$}}}
\newcommand{\bW}{{\mbox{\boldmath$W$}}}
\newcommand{\sbW}{{\mbox{\scriptsize\boldmath $W$}}}
\newcommand{\sbw}{{\mbox{\scriptsize\boldmath $w$}}}
% x
\newcommand{\bx}{{\mbox{\boldmath$x$}}}
\newcommand{\bX}{{\mbox{\boldmath$X$}}}
\newcommand{\sbx}{{\mbox{\scriptsize\boldmath $x$}}}
\newcommand{\sbX}{{\mbox{\scriptsize\boldmath $X$}}}
%y
\newcommand{\by}{{\mbox{\boldmath$y$}}}
\newcommand{\bY}{{\mbox{\boldmath$Y$}}}
\newcommand{\sby}{{\mbox{\scriptsize\boldmath $y$}}}
\newcommand{\sbY}{{\mbox{\scriptsize\boldmath $Y$}}}
%z
\newcommand{\bz}{{\mbox{\boldmath$z$}}}
\newcommand{\bZ}{{\mbox{\boldmath$Z$}}}
\newcommand{\sbz}{{\mbox{\scriptsize\boldmath$z$}}}
\newcommand{\sbZ}{{\mbox{\scriptsize\boldmath$Z$}}}


%%%%%%%%%%%
% 	Bold Grik	%
%%%%%%%%%%%
% alpha
\newcommand{\balpha}{{\mbox{\boldmath$\alpha$}}}
\newcommand{\sbalpha}{{\mbox{\scriptsize\boldmath$\alpha$}}}
% beta
\newcommand{\bbeta}{{\mbox{\boldmath$\beta$}}}
\newcommand{\sbbeta}{{\mbox{\scriptsize\boldmath$\beta$}}}
% gamma
\newcommand{\bgamma}{{\mbox{\boldmath$\gamma$}}}
\newcommand{\bGamma}{{\mbox{\boldmath$\Gamma$}}}
\newcommand{\sbgamma}{{\mbox{\scriptsize\boldmath$\gamma$}}}
% delta
\newcommand{\bdelta}{{\mbox{\boldmath$\delta$}}}
\newcommand{\bDelta}{{\mbox{\boldmath$\Delta$}}}
\newcommand{\sbdelta}{{\mbox{\scriptsize\boldmath$\delta$}}}
% epsilon
\newcommand{\bepsilon}{{\mbox{\boldmath$\epsilon$}}}
\newcommand{\bvarepsilon}{{\mbox{\boldmath$\varepsilon$}}}
\newcommand{\sbvarepsilon}{{\mbox{\scriptsize\boldmath$\varepsilon$}}}
% zeta
\newcommand{\bzeta}{{\mbox{\boldmath$\zeta$}}}
% eta
\newcommand{\etab}{{\mbox{\boldmath$\eta$}}}
% theta
\newcommand{\btheta}{{\mbox{\boldmath$\theta$}}}
\newcommand{\bTheta}{{\mbox{\boldmath$\Theta$}}}
\newcommand{\bvartheta}{{\mbox{\boldmath$\vartheta$}}}
% iota
\newcommand{\biota}{{\mbox{\boldmath$\iota$}}}
\newcommand{\sbiota}{{\mbox{\scriptsize\boldmath$\iota$}}}
% kappa
\newcommand{\bkappa}{{\mbox{\boldmath$\kappa$}}}
% lambda
\newcommand{\blambda}{{\mbox{\boldmath$\lambda$}}}
\newcommand{\bLambda}{{\mbox{\boldmath$\Lambda$}}}
% mu
\newcommand{\bmu}{{\mbox{\boldmath$\mu$}}}
\newcommand{\sbmu}{{\mbox{\scriptsize\boldmath$\mu$}}}
% nu
\newcommand{\bnu}{{\mbox{\boldmath$\nu$}}}
\newcommand{\sbnu}{{\mbox{\scriptsize\boldmath$\nu$}}}
% xi
\newcommand{\bxi}{{\mbox{\boldmath$\xi$}}}
\newcommand{\bXi}{{\mbox{\boldmath$\Xi$}}}
% pi
\newcommand{\bpi}{{\mbox{\boldmath$\pi$}}}
\newcommand{\bPi}{{\mbox{\boldmath$\Pi$}}}
\newcommand{\bvarpi}{{\mbox{\boldmath$\varpi$}}}
% rho
\newcommand{\brho}{{\mbox{\boldmath$\rho$}}}
\newcommand{\bvarrho}{{\mbox{\boldmath$\varrho$}}}
% sigma
\newcommand{\bsigma}{{\mbox{\boldmath$\sigma$}}}
\newcommand{\bSigma}{{\mbox{\boldmath$\Sigma$}}}
\newcommand{\bvarsigma}{{\mbox{\boldmath$\varsigma$}}}
% tau
\newcommand{\btau}{{\mbox{\boldmath$\tau$}}}
% upsilon
\newcommand{\bupsilon}{{\mbox{\boldmath$\upsilon$}}}
\newcommand{\bUpsilon}{{\mbox{\boldmath$\Upsilon$}}}
% phi
\newcommand{\bphi}{{\mbox{\boldmath$\phi$}}}
\newcommand{\bPhi}{\mathbf \Phi}
\newcommand{\bvarphi}{{\mbox{\boldmath$\varphi$}}}
% chi
\newcommand{\bchi}{{\mbox{\boldmath$\chi$}}}
% psi
\newcommand{\bpsi}{{\mbox{\boldmath$\psi$}}}
\newcommand{\bPsi}{\mathbf \Psi}
% omega
\newcommand{\bomega}{{\mbox{\boldmath$\omega$}}}
\newcommand{\bOmega}{{\mbox{\boldmath$\Omega$}}}
% nabla
\newcommand{\bnabla}{{\mbox{\boldmath$\nabla$}}}

% calligraphic letters
\newcommand{\cA}{{\cal A}}
\newcommand{\cB}{{\cal B}}
\newcommand{\cC}{{\cal C}}
\newcommand{\cD}{{\cal D}}
\newcommand{\cE}{{\cal E}}
\newcommand{\cF}{{\cal F}}
\newcommand{\cG}{{\cal G}}
\newcommand{\cH}{{\cal H}}
\newcommand{\cI}{{\cal I}}
\newcommand{\cJ}{{\cal J}}
\newcommand{\cK}{{\cal K}}
\newcommand{\cL}{{\cal L}}
\newcommand{\cM}{{\cal M}}
\newcommand{\cN}{{\cal N}}
\newcommand{\cO}{{\cal O}}
\newcommand{\cP}{{\cal P}}
\newcommand{\cQ}{{\cal Q}}
\newcommand{\cR}{{\cal R}}
\newcommand{\cS}{{\cal S}}
\newcommand{\cT}{{\cal T}}
\newcommand{\cU}{{\cal U}}
\newcommand{\cV}{{\cal V}}
\newcommand{\cW}{{\cal W}}
\newcommand{\cX}{{\cal X}}
\newcommand{\cY}{{\cal Y}}
\newcommand{\cZ}{{\cal Z}}
% calligraphic letters
\newcommand{\mA}{{\mathscr A}}
\newcommand{\mB}{{\mathscr B}}
\newcommand{\mC}{{\mathscr C}}
\newcommand{\mD}{{\mathscr D}}
\newcommand{\mE}{{\mathscr E}}
\newcommand{\mF}{{\mathscr F}}
\newcommand{\mG}{{\mathscr G}}
\newcommand{\mH}{{\mathscr H}}
\newcommand{\mI}{{\mathscr I}}
\newcommand{\mJ}{{\mathscr J}}
\newcommand{\mK}{{\mathscr K}}
\newcommand{\mL}{{\mathscr L}}
\newcommand{\mM}{{\mathscr M}}
\newcommand{\mN}{{\mathscr N}}
\newcommand{\mO}{{\mathscr O}}
\newcommand{\mP}{{\mathscr P}}
\newcommand{\mQ}{{\mathscr Q}}
\newcommand{\mR}{{\mathscr R}}
\newcommand{\mS}{{\mathscr S}}
\newcommand{\mT}{{\mathscr T}}
\newcommand{\mU}{{\mathscr U}}
\newcommand{\mV}{{\mathscr V}}
\newcommand{\mW}{{\mathscr W}}
\newcommand{\mX}{{\mathscr X}}
\newcommand{\mY}{{\mathscr Y}}
\newcommand{\mZ}{{\mathscr Z}}
% calligraphic letters
\newcommand{\bbA}{{\mathbb A}}
\newcommand{\bbB}{{\mathbb B}}
\newcommand{\bbC}{\mathbb{C}}
\newcommand{\bbD}{\mathbb{D}}
\newcommand{\bbE}{{\mathbb E}}
\newcommand{\bbF}{{\mathbb F}}
\newcommand{\bbG}{{\mathbb G}}
\newcommand{\bbH}{{\mathbb H}}
\newcommand{\bbI}{{\mathbb I}}
\newcommand{\bbJ}{{\mathbb J}}
\newcommand{\bbK}{{\mathbb K}}
\newcommand{\bbL}{\mathbb{L}}
\newcommand{\bbM}{{\mathbb M}}
\newcommand{\bbN}{\mathbb{N}}
\newcommand{\bbO}{{\mathbb O}}
\newcommand{\bbP}{\mathbb{P}}
\newcommand{\bbQ}{{\mathbb Q}}
\newcommand{\bbR}{\mathbb{R}}
\newcommand{\bbS}{\mathbb{S}}
\newcommand{\bbT}{{\mathbb T}}
\newcommand{\bbU}{{\mathbb U}}
\newcommand{\bbV}{\mathbb{V}}
\newcommand{\bbW}{{\mathbb W}}
\newcommand{\bbX}{\mathbb{X}}
\newcommand{\bbY}{\mathbb{Y}}
\newcommand{\bbZ}{{\mathbb Z}}
%%%%%%%%%%%%%%
% 	Tilde English	%
%%%%%%%%%%%%%%
% a
\newcommand{\tila}{\tilde{a}}
\newcommand{\tilA}{\tilde{A}}
% b
\newcommand{\tilb}{\tilde{b}}
\newcommand{\tilB}{\tilde{B}}
% c
\newcommand{\tilc}{\tilde{c}}
\newcommand{\tilC}{\tilde{C}}
% d
\newcommand{\tild}{\tilde{d}}
\newcommand{\tilD}{\tilde{D}}
% e
\newcommand{\tile}{\tilde{e}}
\newcommand{\tilE}{\tilde{E}}
% f
\newcommand{\tilf}{\tilde{f}}
\newcommand{\tilF}{\tilde{F}}
% g
\newcommand{\tilg}{\tilde{g}}
\newcommand{\tilG}{\tilde{G}}
% h
\newcommand{\tilh}{\tilde{h}}
\newcommand{\tilbh}{\tilde\bh}
\newcommand{\tilH}{\tilde{H}}
% i
\newcommand{\tili}{\tilde{i}}
\newcommand{\tilI}{\tilde{I}}
% i
\newcommand{\tilj}{\tilde{j}}
\newcommand{\tilJ}{\tilde{J}}
% k
\newcommand{\tilk}{\tilde{k}}
\newcommand{\tilK}{\tilde{K}}
% l
\newcommand{\till}{\tilde{l}}
\newcommand{\tilL}{\tilde{L}}
% m
\newcommand{\tilm}{\tilde{m}}
\newcommand{\tilM}{\tilde{M}}
% n
\newcommand{\tiln}{\tilde{n}}
\newcommand{\tilN}{\tilde{N}}
% o
\newcommand{\tilo}{\tilde{o}}
\newcommand{\tilO}{\tilde{O}}
% p
\newcommand{\tilp}{\tilde{p}}
\newcommand{\tilP}{\tilde{P}}
% q
\newcommand{\tilq}{\tilde{q}}
\newcommand{\tilQ}{\tilde{Q}}
\newcommand{\tilbq}{\tilde{\bq}}
% r
\newcommand{\tilr}{\tilde{r}}
\newcommand{\tilR}{\tilde{R}}
\newcommand{\tilcR}{\tilde{\cR}}
\newcommand{\tilbr}{\tilde{\br}}
% s
\newcommand{\tils}{\tilde{s}}
\newcommand{\tilS}{\tilde{S}}
% t
\newcommand{\tilt}{\tilde{t}}
\newcommand{\tilT}{\tilde{T}}
% u
\newcommand{\tilu}{\tilde{u}}
\newcommand{\tilU}{\tilde{U}}
% v
\newcommand{\tilv}{\tilde{v}}
\newcommand{\tilV}{\tilde{V}}
% w
\newcommand{\tilw}{\tilde{w}}
\newcommand{\tilW}{\tilde{W}}
% x
\newcommand{\tilx}{\tilde{x}}
\newcommand{\tilbx}{\tilde{\bx}}
\newcommand{\tilX}{\tilde{X}}
\newcommand{\tilbX}{\tilde{\bX}}
%y
\newcommand{\tily}{\tilde{y}}
\newcommand{\tilY}{\tilde{Y}}
\newcommand{\tilby}{\tilde{\by}}
\newcommand{\tilbY}{\tilde{\bY}}
%z
\newcommand{\tilz}{\tilde{z}}
\newcommand{\tilZ}{\tilde{Z}}
\newcommand{\tilbZ}{\tilde{\bZ}}


%%%%%%%%%%%%
%% 	Bold Grik	%
%%%%%%%%%%%%
%% alpha
\newcommand{\tilalpha}{\tilde{\alpha}}
%\newcommand{\sbalpha}{{\mbox{\scriptsize\boldmath$\alpha$}}}
%% beta
\newcommand{\tilbeta}{{\tilde{\beta}}}
%\newcommand{\sbbeta}{{\mbox{\scriptsize\boldmath$\beta$}}}
%% gamma
\newcommand{\tilgamma}{{\tilde{\gamma}}}
%\newcommand{\bGamma}{{\mbox{\boldmath$\Gamma$}}}
%\newcommand{\sbgamma}{{\mbox{\scriptsize\boldmath$\gamma$}}}
%% delta
%\newcommand{\bdelta}{{\mbox{\boldmath$\delta$}}}
%\newcommand{\bDelta}{{\mbox{\boldmath$\Delta$}}}
%\newcommand{\sbdelta}{{\mbox{\scriptsize\boldmath$\delta$}}}
%% epsilon
%\newcommand{\bepsilon}{{\mbox{\boldmath$\epsilon$}}}
%\newcommand{\bvarepsilon}{{\mbox{\boldmath$\varepsilon$}}}
%\newcommand{\sbvarepsilon}{{\mbox{\scriptsize\boldmath$\varepsilon$}}}
%% zeta
%\newcommand{\bzeta}{{\mbox{\boldmath$\zeta$}}}
%% eta
%\newcommand{\etab}{{\mbox{\boldmath$\eta$}}}
%% theta
%\newcommand{\btheta}{{\mbox{\boldmath$\theta$}}}
%\newcommand{\bTheta}{{\mbox{\boldmath$\Theta$}}}
%\newcommand{\bvartheta}{{\mbox{\boldmath$\vartheta$}}}
%% iota
%\newcommand{\biota}{{\mbox{\boldmath$\iota$}}}
%\newcommand{\sbiota}{{\mbox{\scriptsize\boldmath$\iota$}}}
%% kappa
%\newcommand{\bkappa}{{\mbox{\boldmath$\kappa$}}}
%% lambda
%\newcommand{\blambda}{{\mbox{\boldmath$\lambda$}}}
%\newcommand{\bLambda}{{\mbox{\boldmath$\Lambda$}}}
%% mu
%\newcommand{\bmu}{{\mbox{\boldmath$\mu$}}}
%\newcommand{\sbmu}{{\mbox{\scriptsize\boldmath$\mu$}}}
%% nu
%\newcommand{\bnu}{{\mbox{\boldmath$\nu$}}}
%\newcommand{\sbnu}{{\mbox{\scriptsize\boldmath$\nu$}}}
%% xi
%\newcommand{\bxi}{{\mbox{\boldmath$\xi$}}}
%\newcommand{\bXi}{{\mbox{\boldmath$\Xi$}}}
%% pi
%\newcommand{\bpi}{{\mbox{\boldmath$\pi$}}}
%\newcommand{\bPi}{{\mbox{\boldmath$\Pi$}}}
%\newcommand{\bvarpi}{{\mbox{\boldmath$\varpi$}}}
%% rho
%\newcommand{\brho}{{\mbox{\boldmath$\rho$}}}
%\newcommand{\bvarrho}{{\mbox{\boldmath$\varrho$}}}
%% sigma
%\newcommand{\bsigma}{{\mbox{\boldmath$\sigma$}}}
%\newcommand{\bSigma}{{\mbox{\boldmath$\Sigma$}}}
%\newcommand{\bvarsigma}{{\mbox{\boldmath$\varsigma$}}}
% tau
\newcommand{\tiltau}{{\mbox{$\tilde{\tau}$}}}
%% upsilon
%\newcommand{\bupsilon}{{\mbox{\boldmath$\upsilon$}}}
%\newcommand{\bUpsilon}{{\mbox{\boldmath$\Upsilon$}}}
%% phi
%\newcommand{\bphi}{{\mbox{\boldmath$\phi$}}}
%\newcommand{\bPhi}{\mathbf \Phi}
%\newcommand{\bvarphi}{{\mbox{\boldmath$\varphi$}}}
%% chi
%\newcommand{\bchi}{{\mbox{\boldmath$\chi$}}}
%% psi
%\newcommand{\bpsi}{{\mbox{\boldmath$\psi$}}}
%\newcommand{\bPsi}{\mathbf \Psi}
%% omega
%\newcommand{\bomega}{{\mbox{\boldmath$\omega$}}}
%\newcommand{\bOmega}{{\mbox{\boldmath$\Omega$}}}

%%%%%%%%%%%%%%
% 	Tilde English	%
%%%%%%%%%%%%%%
% a
\newcommand{\bara}{\bar{a}}
\newcommand{\barA}{\bar{A}}
% b
\newcommand{\barb}{\bar{b}}
\newcommand{\barB}{\bar{B}}
% c
\newcommand{\barc}{\bar{c}}
\newcommand{\barC}{\bar{C}}
% d
\newcommand{\bard}{\bar{d}}
\newcommand{\barD}{\bar{D}}
% e
\newcommand{\bare}{\bar{e}}
\newcommand{\barE}{\bar{E}}
% f
\newcommand{\barbf}{\bar{f}}
\newcommand{\barF}{\bar{F}}
% g
\newcommand{\barg}{\bar{g}}
\newcommand{\barG}{\bar{G}}
% h
\newcommand{\barh}{\bar{h}}
\newcommand{\barbh}{\bar\bh}
\newcommand{\barH}{\bar{H}}
% i
\newcommand{\bari}{\bar{i}}
\newcommand{\barI}{\bar{I}}
% i
\newcommand{\barj}{\bar{j}}
\newcommand{\barJ}{\bar{J}}
% k
\newcommand{\bark}{\bar{k}}
\newcommand{\barK}{\bar{K}}
% l
\newcommand{\barl}{\bar{l}}
\newcommand{\barL}{\bar{L}}
% m
\newcommand{\barm}{\bar{m}}
\newcommand{\barM}{\bar{M}}
% n
\newcommand{\barn}{\bar{n}}
\newcommand{\barN}{\bar{N}}
% o
\newcommand{\baro}{\bar{o}}
\newcommand{\barO}{\bar{O}}
% p
\newcommand{\barp}{\bar{p}}
\newcommand{\barP}{\bar{P}}
% q
\newcommand{\barq}{\bar{q}}
\newcommand{\barQ}{\bar{Q}}
% r
\newcommand{\barr}{\bar{r}}
\newcommand{\barR}{\bar{R}}
% s
\newcommand{\bars}{\bar{s}}
\newcommand{\barS}{\bar{S}}
% t
\newcommand{\bart}{\bar{t}}
\newcommand{\barT}{\bar{T}}
% u
\newcommand{\baru}{\bar{u}}
\newcommand{\barU}{\bar{U}}
% v
\newcommand{\barv}{\bar{v}}
\newcommand{\barV}{\bar{V}}
% w
\newcommand{\barw}{\bar{w}}
\newcommand{\barW}{\bar{W}}
\newcommand{\barbw}{\bar{\bw}}
% x
\newcommand{\barx}{\bar{x}}
\newcommand{\barX}{\bar{X}}
%y
\newcommand{\bary}{\bar{y}}
\newcommand{\barY}{\bar{Y}}
%z
\newcommand{\barz}{\bar{z}}
\newcommand{\barZ}{\bar{Z}}
%%%%%%%%%%%%%%
% 	Hat English	%
%%%%%%%%%%%%%%
% a
\newcommand{\hata}{\hat{a}}
\newcommand{\hatA}{\hat{A}}
% b
\newcommand{\hatb}{\hat{b}}
\newcommand{\hatB}{\hat{B}}
% c
\newcommand{\hatc}{\hat{c}}
\newcommand{\hatC}{\hat{C}}
\newcommand{\hatbc}{\hat{\bc}}
% d
\newcommand{\hatd}{\hat{d}}
\newcommand{\hatD}{\hat{D}}
% e
\newcommand{\hate}{\hat{e}}
\newcommand{\hatE}{\hat{E}}
% f
\newcommand{\hatbf}{\hat{f}}
\newcommand{\hatF}{\hat{F}}
% g
\newcommand{\hatg}{\hat{g}}
\newcommand{\hatG}{\hat{G}}
% h
\newcommand{\hath}{\hat{h}}
\newcommand{\hatbh}{\hat\bh}
\newcommand{\hatH}{\hat{H}}
% i
\newcommand{\hati}{\hat{i}}
\newcommand{\hatI}{\hat{I}}
% i
\newcommand{\hatj}{\hat{j}}
\newcommand{\hatJ}{\hat{J}}
% k
\newcommand{\hatk}{\hat{k}}
\newcommand{\hatK}{\hat{K}}
% l
\newcommand{\hatl}{\hat{l}}
\newcommand{\hatL}{\hat{L}}
% m
\newcommand{\hatm}{\hat{m}}
\newcommand{\hatM}{\hat{M}}
% n
\newcommand{\hatn}{\hat{n}}
\newcommand{\hatN}{\hat{N}}
% o
\newcommand{\hato}{\hat{o}}
\newcommand{\hatO}{\hat{O}}
% p
\newcommand{\hatp}{\hat{p}}
\newcommand{\hatP}{\hat{P}}
% q
\newcommand{\hatq}{\hat{q}}
\newcommand{\hatQ}{\hat{Q}}
% r
\newcommand{\hatr}{\hat{r}}
\newcommand{\hatR}{\hat{R}}
\newcommand{\hatbr}{\hat{\br}}
% s
\newcommand{\hats}{\hat{s}}
\newcommand{\hatS}{\hat{S}}
% t
\newcommand{\hatt}{\hat{t}}
\newcommand{\hatT}{\hat{T}}
% u
\newcommand{\hatu}{\hat{u}}
\newcommand{\hatU}{\hat{U}}
% v
\newcommand{\hatv}{\hat{v}}
\newcommand{\hatV}{\hat{V}}
% w
\newcommand{\hatw}{\hat{w}}
\newcommand{\hatW}{\hat{W}}
% x
\newcommand{\hatx}{\hat{x}}
\newcommand{\hatX}{\hat{X}}
\newcommand{\hatbX}{\hat{\bX}}
%y
\newcommand{\haty}{\hat{y}}
\newcommand{\hatY}{\hat{Y}}
%z
\newcommand{\hatz}{\hat{z}}
\newcommand{\hatZ}{\hat{Z}}
%%%%%%%%%%%%%%
% 	Tilde English	%
%%%%%%%%%%%%%%
% A
\newcommand{\mathall}{{\rm all}}
\newcommand{\mathand}{{\rm and}}
% B
\newcommand{\BER}{{\rm BER}}
% C
\newcommand{\Cov}{{\rm Cov}}
\newcommand{\conv}{{\rm conv}}
% D
\newcommand{\diver}{\mathrm{div~}}
% E
\newcommand{\eg}{\textit{e.g.}}
\newcommand{\etc}{\textit{etc.}}
\newcommand{\etal}{\textit{et al.}}
\newcommand{\iid}{\textit{i.i.d.}}
\newcommand{\ie}{\textit{i.e.}}
% F
\newcommand{\mathfor}{{\rm for}}
% G
\newcommand{\GF}{{\rm GF}}
\newcommand{\grad}{\mathrm{grad~}}
% I

\newcommand{\mathif}{{\rm if}}
% O
\newcommand{\opt}{{\rm opt}}
\newcommand{\mathor}{{\rm or}}
\newcommand{\mathotherwise}{{\rm otherwise}}
% R
\newcommand{\rot}{\mathrm{rot~}}
% S
\newcommand{\SER}{{\rm SER}}
\newcommand{\st}{~\mathrm{s.t.}~}
% T
%\newcommand{\th}{\mathrm{th}}
% V
\newcommand{\Var}{{\rm Var}}





% Environment
%%%%%%%%%%%%
% 	 Operations 		%
%%%%%%%%%%%%
\DeclareMathOperator{\diag}{diag}
\DeclareMathOperator*{\argmin}{arg~min}
\DeclareMathOperator*{\argmax}{arg~max}
\DeclareMathOperator*{\minmax}{min~max}
\DeclareMathOperator*{\maxmin}{max~min}
\DeclareMathOperator{\lcm}{lcm}
\DeclareMathOperator{\mtxop}{mtx}
\DeclareMathOperator{\vecop}{vec}
\DeclareMathOperator{\sinc}{sinc}
\DeclareMathOperator{\sgn}{sgn}
\DeclareMathOperator{\rank}{rank}
\DeclareMathOperator{\tr}{tr}
\DeclareMathOperator{\sig}{sig}
% New Environment
\newenvironment{MyDefinition}[1]
    {\begin{itembox}[l]{\bf #1}
    \begin{definition}}%
    {\end{definition}
    \end{itembox}}
    
\newenvironment{MyTheorem}
    {\begin{shadebox}
    \bigskip
    \begin{theorem}}%
    {\end{theorem}
    \smallskip
    \end{shadebox}}
 
\newenvironment{MyProof}
    {\begin{boxnote}
    	%\begin{quote}
    %\begin{proof}
    }
    {%\end{proof}
    %\end{quote}
\end{boxnote}}
 
 \newenvironment{MyExample}
 {\begin{shaded}
 		\begin{quote}
 		\begin{example}}%
 		{\end{example}
   \end{quote}
\end{shaded}}
  
\newenvironment{MyLemma}
    {\begin{doublebox}
    \begin{lemma}}%
    {\end{lemma}
    \end{doublebox}}
    

    
% References
%\newcommand{\IEEE_L_COM}{{IEEE} Commun. Lett.~}

\newcommand{\GCOM}{{IEEE} Global Telecommun. Conf.~}
\newcommand{\ICC}{{IEEE} Int. Conf. Commun.~}
\newcommand{\ISIT}{{IEEE} Int. Symp. Infor. Theory~}
\newcommand{\IWSDA}{{IEEE} Int. Workshop Signal Design and Its Applications in Commun.~}
\newcommand{\MILCOM}{{IEEE} Military Commun. Conf.~}
\newcommand{\PIMRC}{{IEEE} Int. Symp. Personal, Indoor and Mobile Radio Commun.~}
\newcommand{\VTC}{{IEEE} Veh. Tech. Conf.~}
\newcommand{\WCNC}{{IEEE} Wireless Commun. Networking Conf.~}
\bibliographystyle{IEEEtran}
\usepackage{graphicx}
\usepackage{lscape}
\usepackage{beamerthemesplit}
\setbeamertemplate{navigation symbols}{}
\setbeamertemplate{footline}[frame number]
\usetheme{Luebeck}

\usepackage[orientation=portrait,size=a0,scale=1.4]{beamerposter}
\mode<presentation>{%
  \usetheme{Frankfurt}%
}

\renewcommand{\maketitle}{%
  \vspace*{1ex}%
  \begin{center}%
    \Huge\inserttitle\\%
    \LARGE\insertauthor\\%
    \Large\insertinstitute%
  \end{center}%
  \vspace*{-1ex}%
}

\setbeamerfont{block title}{size=\LARGE}

\title[Construction of Z-CCC]{~~~A General Construction of \\$Z$-Concatenative Complete Complementary Codes~~~}
  \author[Han \& Hashimoto]{Chenggao~Han and Takeshi~Hashimoto}
  \institute[UEC]{Graduate School of Informatics and Engineering, The University of Electro-Communications, Japan\\Email:\{hana,~hashimoto\}@ee.uec.ac.jp}
\date{}

\begin{document}
\begin{frame}{\maketitle}
  \begin{columns}
    \begin{column}{.45\textwidth}
      \begin{block}{Background}
      \begin{itemize}
        \item For (quasi-) synchronous CDMA systems, complete complementary codes (CCCs)  and zero correlation zone (ZCZ) sequences provide co-channel and multi-path interference free communications. \\
        \item Comparing with ZCZ sequences, CCC leads a CDMA system which has lower implementation complexity but is lacking in spectral efficiency in general.
        \item As a hybrid of CCC and ZCZ sequence, we proposed the Z-concatenative CCC (Z-CCC) and presented two Z-CCCs consisting of rows of the discrete Fourier transform (DFT) and Hadamard matrices.
        \item  In this work, we generalize the previous work and propose a novel construction of Z-CCCs.
	\end{itemize}
      \end{block}
      \vspace*{1ex}
      
      %%%%%%%%%
      \begin{definition}
      \begin{itemize}
      \item {\bf zero-correlation zone sequences}
      
       A sequence set (SS) with $M$ length-$L$ sequence, denoted by $(M,L)$-SS, $\bbS$ is called the {\it zero correlation zone SS} or $(M,L;Z)$-ZCZ if their periodic correlations satisfying
\[
\tilR_{\bs_m,\bs_{m'}}(\tau) = \tilR_{\bs_m,\bs_{m'}}(0)\delta(m-{m'}),~{\rm for}~|\tau| \leq Z
\]

	\item complete complementary codes
	
	A sequence family $\left(M,N,L\right)$-SF $\cC$, consisting of $M$ $(N,L)$-SSs, is called to the {\it complete complementary code} (CCC) or $(M,N,L)$-CCC if the sum of the correlation between $\bs_n^m$ and $\bs_n^{m'}$ satisfying
\begin{eqnarray*}
\cR_{\cC^m,\cC^{m'}}(\tau) = \sum_{n = 0}^{N-1}R_{\bs_n^m,\bs_n^{m'}}(\tau)\delta(m-{m'},\tau)
\label{CCCfun}
\end{eqnarray*}
for all $0 \leq m,m' < M$.

\item $Z$-concatenative complete complementary codes

If the $(M,NL)$-SS $\bbS$ generated by connecting sequences in each SS of an $(M,N,L)$-CCC $\cC$, {\it i.e.}, $\bbS = \{\bs^m\}_{m=0}^{M-1} = \{({\bf c}_n^m)_{n=0}^{N-1}\}_{m=0}^{M-1}$,  is an $(M,LN;Z)$-ZCZ, then $\cC$ is called the Z-concatenative CCC or $(M,N,L;Z)$-CCC. 
      
      \item Kronecker's product
      
      For two matrix ${\bf A}^{(0)}$ and ${\bf A}^{(1)}$ of sizes $M_0\times N_0$ and $M_1 \times N_1$, respectively, $\otimes$ denotes the Kronecker's product which yields a size $M_0M_1 \times N_0N_1$ matrix by the rule
\begin{eqnarray*}
{\bf A}^{(1)}\otimes {\bf A}^{(0)} = \left[a^{(1,m)}_{n}{\bf A}^{(0)}\right]_{m=0,n=0}^{M_1-1,N_1-1}
\end{eqnarray*}
      \end{itemize}
      \end{definition}
       \vspace*{1ex}
       
       %%%%%%%
      \begin{block}{Construction Method}
      Let
\begin{eqnarray*}
{\bf U}=\bigotimes_{k=0}^{K-1}{\bf F}_{N_k} :={\bf F}_{N_{(K-1)}}\otimes {\bf F}_{N_{(K-2)}}\otimes \cdots \otimes {\bf F}_{N_0}
\end{eqnarray*}
where $\bF_N$ stands for the $N$-dimensional DFT matrix.
      Then, the $(N,N,N)$-SF, $N = \prod_{k=0}^{K-1}N_k$, constructed by entry-wise multiplication $\odot$ as
\begin{eqnarray*}
\cC = [{\bf c}_n^m]_{m=0,n=0}^{N-1,N-1} = [{\bf u}_N^m\odot{\bf u}_N^n]_{m=0,n=0}^{N-1,N-1}
\label{CCCC}
\end{eqnarray*}
is an $(N,N,N;Z)$-CCC with $Z = (N_{K-1}-1)\prod_{k=0}^{K-2}N_k$.
      \end{block}
       \vspace*{1ex}
      
      %%%%%%
            \begin{block}{Construction efficiency}


For the bound achieving CCCs, {\it i.e.}, $(N,N,L;Z)$-CCCs, we have theoretical bound $Z \leq L-1$. Therefore, under definition of the merit figure $\eta := (Z+1)/L$, the construction efficiency of the proposed Z-CCC can be evaluated as
\begin{eqnarray*}
\eta = \frac{(N_{K-1}-1)\prod_{k=0}^{K-2}N_k+1}{\prod_{k=0}^{K-1}N_k} \approx \frac{N_{K-1}-1}{N_{K-1}}
\label{MF}
\end{eqnarray*}

There is a tradeoff relationship between merit figure and alphabet size. To achieve high merit figure, a large $N_{K-1}$ is expected at expenses of increasing alphabet size. 
      \end{block}
      
    \end{column}
    \begin{column}{.45\textwidth}
    
    \begin{example}
    Let
    \begin{eqnarray*}
{\bf U} = {\bf F}_4\otimes {\bf F}_2 = \left[
\begin{array}{c}
{\bf u}^{0} \cr
{\bf u}^{1} \cr
{\bf u}^{2} \cr
{\bf u}^{3} \cr
{\bf u}^{4} \cr
{\bf u}^{5} \cr
{\bf u}^{6} \cr
{\bf u}^{7}
\end{array}
\right]=\left[
\begin{array}{rrrrrrrr}
1 & 1 & 1 & 1 & 1 & 1 & 1 & 1 \cr
1 & -1 & 1 & -1 & 1 & -1 & 1 & -1 \cr
1 & 1 & j & j & -1 & -1 & -j & -j \cr
1 & -1 & j & -j & -1 & 1 & -j & j \cr
1 & 1 & -1 & -1 & 1 & 1 & -1 & -1 \cr
1 & -1 & -1 & 1 & 1 & -1 & -1 & 1 \cr
1 & 1 & -j & -j & -1 & -1 & j & j \cr
1 & -1 & -j & j & -1 & 1 & j & -j
\end{array}
\right]
\end{eqnarray*}

Then, from the table
\begin{table}[htdp]
\caption{Parameters of the considering CCs}
\begin{center}
\begin{tabular}{c|cccccccc}
\hline
${\bf j}$ $\cdot$ ${\bf i}$ & $(00)$  & $(10)$ & $(01)$ & $(11)$ & $(02)$ & $(12)$ & $(03)$ & $(13)$ \\ \hline
$(00)$ & $(00)$  & $(10)$ & $(01)$ & $(11)$ & $(02)$ & $(12)$ & $(03)$ & $(13)$ \\
$(10)$ & $(10)$  & $(00)$ & $(11)$ & $(01)$ & $(12)$ & $(02)$ & $(13)$ & $(03)$ \\
$(01)$ & $(01)$  & $(11)$ & $(02)$ & $(12)$ & $(03)$ & $(13)$ & $(00)$ & $(10)$ \\
$(11)$ & $(11)$  & $(01)$ & $(12)$ & $(02)$ & $(13)$ & $(03)$ & $(10)$ & $(00)$ \\
$(02)$ & $(02)$  & $(12)$ & $(03)$ & $(13)$ & $(00)$ & $(10)$ & $(01)$ & $(11)$ \\
$(12)$ & $(12)$  & $(02)$ & $(13)$ & $(03)$ & $(10)$ & $(00)$ & $(11)$ & $(01)$ \\
$(03)$ & $(03)$  & $(13)$ & $(00)$ & $(10)$ & $(01)$ & $(11)$ & $(02)$ & $(12)$ \\
$(13)$ & $(13)$  & $(03)$ & $(10)$ & $(00)$ & $(11)$ & $(01)$ & $(12)$ & $(02)$ \\ \hline
 \end{tabular}
\end{center}
\label{CCTable}
\end{table}% 



the resulted SF is given by
 \begin{eqnarray*}
\mathcal{C} &=&\left[
 \begin{array}{cccccccc}
{\bf u}^{0} & {\bf u}^{1} & {\bf u}^{2} & {\bf u}^{3} & {\bf u}^{4} & {\bf u}^{5} & {\bf u}^{6} & {\bf u}^{7} \cr
{\bf u}^{1} & {\bf u}^{0} & {\bf u}^{3} & {\bf u}^{2} & {\bf u}^{5} & {\bf u}^{4} & {\bf u}^{7} & {\bf u}^{6} \cr
{\bf u}^{2} & {\bf u}^{3} & {\bf u}^{4} & {\bf u}^{5} & {\bf u}^{6} & {\bf u}^{7} & {\bf u}^{0} & {\bf u}^{1} \cr
{\bf u}^{3} & {\bf u}^{2} & {\bf u}^{5} & {\bf u}^{4} & {\bf u}^{7} & {\bf u}^{6} &  {\bf u}^{1} & {\bf u}^{0} \cr
{\bf u}^{4} & {\bf u}^{5} & {\bf u}^{6} & {\bf u}^{7} & {\bf u}^{0} & {\bf u}^{1} & {\bf u}^{2} & {\bf u}^{3} \cr
{\bf u}^{5} & {\bf u}^{4} & {\bf u}^{7} & {\bf u}^{6} & {\bf u}^{1} & {\bf u}^{0} & {\bf u}^{3} & {\bf u}^{2} \cr
{\bf u}^{6} & {\bf u}^{7} & {\bf u}^{0} & {\bf u}^{1} & {\bf u}^{2} & {\bf u}^{3} & {\bf u}^{4} & {\bf u}^{5} \cr
{\bf u}^{7} & {\bf u}^{6} & {\bf u}^{1} & {\bf u}^{0} & {\bf u}^{3} & {\bf u}^{2} & {\bf u}^{5} & {\bf u}^{4}  \end{array}
\right]
\label{CCC22}
\end{eqnarray*}

One can confirm that ${\cal C}$ is $(8,8,8)$-CCC and if we let
\begin{eqnarray*}
\mathbb{S} = \left[
\begin{array}{c}
{\bf s}^{0} \cr
{\bf s}^{1} \cr
{\bf s}^{2} \cr
{\bf s}^{3} \cr
{\bf s}^{4} \cr
{\bf s}^{5} \cr
{\bf s}^{6} \cr
{\bf s}^{7}
\end{array}
\right] = 
\left[
 \begin{array}{c}
\left({\bf u}^{0} ~ {\bf u}^{1} ~ {\bf u}^{2} ~ {\bf u}^{3} ~ {\bf u}^{4} ~ {\bf u}^{5} ~ {\bf u}^{6} ~ {\bf u}^{7} \right) \cr
\left({\bf u}^{1} ~ {\bf u}^{0} ~ {\bf u}^{3} ~ {\bf u}^{2} ~ {\bf u}^{5} ~ {\bf u}^{4} ~ {\bf u}^{7} ~ {\bf u}^{6} \right)\cr
\left({\bf u}^{2} ~ {\bf u}^{3} ~ {\bf u}^{4} ~ {\bf u}^{5} ~ {\bf u}^{6} ~ {\bf u}^{7} ~ {\bf u}^{0} ~ {\bf u}^{1} \right)\cr
\left({\bf u}^{3} ~ {\bf u}^{2} ~ {\bf u}^{5} ~ {\bf u}^{4} ~ {\bf u}^{7} ~ {\bf u}^{6} ~  {\bf u}^{1} ~ {\bf u}^{0} \right)\cr
\left({\bf u}^{4} ~ {\bf u}^{5} ~ {\bf u}^{6} ~ {\bf u}^{7} ~ {\bf u}^{0} ~ {\bf u}^{1} ~ {\bf u}^{2} ~ {\bf u}^{3} \right)\cr
\left({\bf u}^{5} ~ {\bf u}^{4} ~ {\bf u}^{7} ~ {\bf u}^{6} ~ {\bf u}^{1} ~ {\bf u}^{0} ~ {\bf u}^{3} ~ {\bf u}^{2} \right)\cr
\left({\bf u}^{6} ~ {\bf u}^{7} ~ {\bf u}^{0} ~ {\bf u}^{1} ~ {\bf u}^{2} ~ {\bf u}^{3} ~ {\bf u}^{4} ~ {\bf u}^{5} \right)\cr
\left({\bf u}^{7} ~ {\bf u}^{6} ~ {\bf u}^{1} ~ {\bf u}^{0} ~ {\bf u}^{3} ~ {\bf u}^{2} ~ {\bf u}^{5} ~ {\bf u}^{4}  \right)
\end{array}
\right]
\end{eqnarray*}
Then, it is an $(8,64; 6)$-ZCZ which merit factor is $\eta = 7/8$.
    \end{example}
      
       \vspace*{5ex}
      
      
      \begin{block}{References}
        \begin{thebibliography}{10}
\bibitem{Suehiro88-01J}
N.~Suehiro and M.~Hatori, ``$N$-shift cross-orthogonal sequences,'' {\it IEEE Trans. Info. Theory}, vol. IT-34, no. 1, pp. 143-146, Jan. 1988. 
\bibitem{Suehiro94-06J}
N. Suehiro, ``A signal design without co-channel interference for approximately synchronized CDMA systems," {\it IEEE Journal on Sel. Areas in Commun.}, vol. 12, no. 5, pp. 837-841, June 1994.
\bibitem{Han09-06C}
C. Han and T. Hashimoto, ``$Z$-connecteble complete complementary code and its application in CDMA, in {\it Proc. IEEE Int. Symp. Information Theory} (ISIT2009), June 2009, pp. 438-422.
\bibitem{Han11-09J}
C. Han, N. Suehiro, and T. Hashimoto, `` A systematic framework for the construction of optimal complete complementary codes, {\it IEEE Trans. Information Theory}, vol. 57, no. 9, pp. 6033- 6042, Sept. 2011.
\end{thebibliography}
      \end{block}
    \end{column}
  \end{columns}
\end{frame}
\end{document}